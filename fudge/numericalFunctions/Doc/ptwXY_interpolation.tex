\subsection{Interpolation}
This section decribes all the functions in the file ``ptwXY\_interpolation.c''.

\subsubsection{ptwXY\_interpolatePoint}
\setargumentNameLengths{interpolation}
This function interpolates an $x$ value between the points (x1,y1) and (x2,y2) to obtain its $y$ value
for the requested \highlight{interpolation}.
\CallingC{fnu\_status ptwXY\_interpolatePoint(}{statusMessageReporting *smr,
    \addArgument{ptwXY\_interpolation interpolation,}
    \addArgument{double x,}
    \addArgument{double *y,}
    \addArgument{double x1,}
    \addArgument{double y1,}
    \addArgument{double x2,}
    \addArgument{double y2 );}}
    \argumentBox{smr}{The \highlight{statusMessageReporting} instance to record errors.}
    \argumentBox{interpolation}{Type of interpolation to perform (see Section~\ref{interpolationSection}).}
    \argumentBox{x}{The $x$ value at which the $y$ value is desired.}
    \argumentBox{x1}{The $x$ value of the first point.}
    \argumentBox{y1}{The $y$ value of the first point.}
    \argumentBox{x2}{The $x$ value of the second point.}
    \argumentBox{y2}{The $y$ value of the second point.}
    \vskip 0.05 in \noindent
If the interpolation flag is invalid or ( x1 $>$ x2 ) then \highlight{nfu\_invalid\-Interpolation} is returned. 
If logarithm interpolation is requested for an axis, and one of the input values for that axis is less than or equal to 0., 
then \highlight{nfu\_invalid\-Interpolation} is also returned. If interpolation is \highlight{ptwXY\_interpolationOther} then
\highlight{nfu\_otherInterpolation} is returned.

\subsubsection{ptwXY\_flatInterpolationToLinear}
This function returns a linear-linear interpolated representation of \highlight{ptwXY}.
\setargumentNameLengths{upperEps}
\CallingC{ptwXYPoints *ptwXY\_flatInterpolationToLinear(}{statusMessageReporting *smr,
    \addArgument{ptwXYPoints *ptwXY,}
    \addArgument{double lowerEps,}
    \addArgument{double upperEps );}}
    \argumentBox{smr}{The \highlight{statusMessageReporting} instance to record errors.}
    \argumentBox{ptwXY}{A pointer to a \highlight{ptwXYPoints} object.}
    \argumentBox{lowerEps}{The amount to adjust every interior point down in x.}
    \argumentBox{upperEps}{The amount to adjust every interior point up in x}
    \vskip 0.05 in \noindent
For every interior point (i.e., $(x_i,y_i)$ for $0 < i < n - 1$ where n is the number of points), two points may be added.
The positions of these points depend on \highlight{lowerEps} and \highlight{upperEps} as follows:
\begin{description}
    \item[lowerEps ==  0 and upperEps ==  0:] This condition is not allowed. status is set to \highlight{nfu\_bad\-Input} and NULL is returned.
        This condition is also returned if either \highlight{lowerEps} or \highlight{upperEps} is negative.
    \item[lowerEps $>$ 0 and upperEps ==  0:] At each interior point $(x_i,y_i)$ the two points $(x_m,y_{i-1})$ and $(x_i,y_i)$ are set.
    \item[lowerEps ==  0 and upperEps $>$ 0:] At each interior point $(x_i,y_i)$ the two points $(x_i,y_{i-1})$ and $(x_p,y_i)$ are set.
    \item[lowerEps $>$ 0 and upperEps $>$ 0:] At each interior point $(x_i,y_i)$, this point is removed and the two 
        points $(x_m,y_{i-1})$ and $(x_p,y_i)$ are set.
\end{description}
where $x_m$ and $x_p$ are given in Table~\ref{flatInterpolationToLinear}.
\begin{table}
\begin{center}
\begin{tabular}{|c|l|l|}  \hline
                & $x_m$                    & $x_p$                          \\ \hline \hline
    $x_i <  0$  & $x_i ( 1 + \epsilon_l )$ & $x_p = x_i ( 1 - \epsilon_p )$ \\ \hline
    $x_i == 0$  & $ -\epsilon_l $          & $ \epsilon_p $                 \\ \hline
    $x_i >  0$  & $x_i ( 1 - \epsilon_l )$ & $x_p = x_i ( 1 + \epsilon_p )$ \\ \hline
\end{tabular}
\end{center}
\caption{The value of $x_m$ and $x_p$ used to adjust interior points in \highlight{ptwXY\_fla-Interpolation\-To\-Linear}. 
    Here, $ \epsilon_l = $ \highlight{lowerEps} and $ \epsilon_p = $ \highlight{upperEps}. \label{flatInterpolationToLinear}}
\end{table}

\subsubsection{ptwXY\_toOtherInterpolation}
This function returns \highlight{ptwXY} converted to interpolation \highlight{interpolation}.
\setargumentNameLengths{interpolation}
\CallingC{ptwXYPoints *ptwXY\_toOtherInterpolation(}{statusMessageReporting *smr,
    \addArgument{ptwXYPoints *ptwXY,}
    \addArgument{ptwXY\_interpolation interpolation,}
    \addArgument{double accuracy );}}
    \argumentBox{smr}{The \highlight{statusMessageReporting} instance to record errors.}
    \argumentBox{ptwXY}{A pointer to a \highlight{ptwXYPoints} object.}
    \argumentBox{interpolation}{The interpolation to convert to.}
    \argumentBox{accuracy}{The accuracy of the conversion.}
    \vskip 0.05 in \noindent
Currently, \highlight{interpolation} can only be \highlight{ptwXY\_\-interpolation\-LinLin}.

\subsubsection{ptwXY\_toUnitbase}
This function returns a unit-based version of \highlight{ptwXY}.
\setargumentNameLengths{scaleRange}
\CallingC{ptwXYPoints *ptwXY\_toUnitbase(}{statusMessageReporting *smr,
    \addArgument{ptwXYPoints *ptwXY,}
    \addArgument{int scaleRange );}}
    \argumentBox{smr}{The \highlight{statusMessageReporting} instance to record errors.}
    \argumentBox{ptwXY}{A pointer to the \highlight{ptwXYPoints} object.}
    \argumentBox{scaleRange}{The y-values are not scaled if this is 0.}
    \vskip 0.05 in \noindent
Unitbasing maps the domain to 0 to 1 by scaling each x-value as 
\begin{equation}
    x_i = ( x_i - x_0 ) / ( x_{n-1} - x_0 )
\end{equation}
and if \highlight{scaleRange} is not 0, scaling each y-value as
\begin{equation}
    y_i = y_i \times ( x_{n-1} - x_0 ) \ \ \ . 
\end{equation}
Unitbasing is most useful for pdf's.

\subsubsection{ptwXY\_fromUnitbase}
This function undoes the unit base mapping done by \highlight{ptwXY\_toUnitbase}.
\setargumentNameLengths{scaleRange}
\CallingC{ptwXYPoints *ptwXY\_fromUnitbase(}{statusMessageReporting *smr,
    \addArgument{ptwXYPoints *ptwXY,}
    \addArgument{double domainMin,}
    \addArgument{double domainMax,}
    \addArgument{int scaleRange );}}
    \argumentBox{smr}{The \highlight{statusMessageReporting} instance to record errors.}
    \argumentBox{ptwXY}{A pointer to the \highlight{ptwXYPoints} object.}
    \argumentBox{domainMin}{The lower domain for the returned \highlight{ptwXYPoints} instances.}
    \argumentBox{domainMax}{The upper domain for the returned \highlight{ptwXYPoints} instances.}
    \argumentBox{scaleRange}{The y-values are not scaled if this is 0.}
    \vskip 0.05 in \noindent
Each x-value is scaled as 
\begin{equation}
    x_i = ( {\rm domainMax} - {\rm domainMin} ) \times x_i + {\rm domainMin}
\end{equation}
and if \highlight{scaleRange} is not 0, each y-value is scaled as 
\begin{equation}
y_i = y_i / ( {\rm domainMax} - {\rm domainMin} ) \ \ \ \ 
\end{equation}

\subsubsection{ptwXY\_unitbaseInterpolate}
This function returns a \highlight{ptwXYPoints} instance that is the unit-base interpolation of \highlight{ptwXY1} at $w_1$
and \highlight{ptwXY2} at $w_2$ at the w-value $w$.
\setargumentNameLengths{interpolation}
\CallingC{ptwXYPoints *ptwXY\_unitbaseInterpolate(}{statusMessageReporting *smr,
    \addArgument{double w,}
    \addArgument{double w1,}
    \addArgument{ptwXYPoints *ptwXY1,}
    \addArgument{double w2,}
    \addArgument{ptwXYPoints *ptwXY2,}
    \addArgument{scaleRange );}}
    \argumentBox{smr}{The \highlight{statusMessageReporting} instance to record errors.}
    \argumentBox{w}{The w-value to interpole to.}
    \argumentBox{w1}{The lower w-value}
    \argumentBox{ptwXY1}{A pointer to a \highlight{ptwXYPoints} object at w1.}
    \argumentBox{w2}{The upper w-value}
    \argumentBox{ptwXY2}{A pointer to a \highlight{ptwXYPoints} object at w2.}
    \argumentBox{scaleRange}{The y-values are not scaled if this is 0.}
    \vskip 0.05 in \noindent
