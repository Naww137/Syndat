\documentclass[11pt,twoside]{article}
\usepackage{verbatim}
\pagestyle{headings}
\setlength{\textwidth}{6.75 in}
\setlength{\textheight}{8.5 in}
\setlength{\oddsidemargin}{0. in}
\setlength{\evensidemargin}{\oddsidemargin}
\markboth{Cross section thermal broadening}{Cross section thermal broadening}

\def\CRoutine{crossSectionAdjustForHeatedTarget}
\def\CRoutineS{\CRoutine \ }
\def\CLibrary{libcrossSectionAdjustForHeatedTarget.a}
\setcounter{secnumdepth}{3}
\setcounter{tocdepth}{3}

\title{Correcting a cross section for a thermal target\footnote{
This work was performed under the auspices of the U.S. Department of Energy by
Lawrence Livermore National Laboratory under contract \#W-7405-ENG-48.} \\
--------------------------- \\
{\it Documentation for LLNL's routines that thermally broaden a cross section
for a heated target
}}
\author{{Bret R. Beck}\\Lawrence Livermore National Laboratory}

\begin{document}

\maketitle

\pagebreak

\section{Introduction}

Most cross section data are given as a function of the incident particle's velocity or energy (e.g., $\sigma(E)$) assuming that the target is
at rest. Most targets, while at rest on average, have a thermal motion which must be included to obtain the actual reaction rate between incident
particle and target.  This article explains how to correct - often called
thermal broadening - a cross section for the thermal motion of the target (see section~\ref{Theory}) and 
describes~\cite{Sigma1} routines that can be used to thermally broaden a cross section (see section~\ref{Routines}).

\section{Theory}     \label{Theory}

For a material at temperature T, the effective cross section ${\left<\sigma(v_{\rm n},T)\right>}$ for a reaction is defined 
as the cross section at incident velocity $v_{\rm n}$ (or energy $E_{\rm n}$) that gives the same reaction rate for a stationary target as the real 
cross section $\sigma(v_{\rm n})$ gives for a thermal target. For a target
of density $\rho$ and incident particle velocity ${\bf v}_{\rm n}$, or speed $v_{\rm n}$, the reaction rate is, 
\begin{equation}	\label{rrEq}
	\rho \, v_{\rm n} \, \left<\sigma(v_{\rm n},T)\right> = \int d{\bf v}_{\rm t} \rho \, v_{\rm r} \, 
		\sigma( v_{\rm r}) \, M({\rm v}_{\rm t}) \ \ \ \ .
\end{equation}
Here, $M(v_{\rm t})$ is the target velocity distribution function and $v_{\bf r} = |{\bf v}_{\rm n} - {\bf v}_{\rm t}|$ 
is the relative speed between the incident particle and a target with velocity ${\bf v}_{\rm t}$.
The relative speed can also be written as,
\begin{equation}    \label{eqRelSpeed}
    v_{\rm r}^2 = v_{\rm n}^2 + v_{\rm t}^2 - 2 \, \mu \, v_{\rm n} v_{\rm t} \ \ \ \ ,
\end{equation}
where $\mu$ is the angle between the incident particle's velocity and the target's velocity.
In Eq.~\ref{rrEq} we can drop the factor $\rho$ appearing on both sides of the equation,
divide by $v_{\rm n}$
and replace $d\vec{v}_{\rm t}$ with $d\theta \, d\mu \, v_{\rm t}^2 \, dv_{\rm t}$ to obtain,
\begin{equation} \label{eqSigmaAverage}
    \left<\sigma(E_{\rm n},T)\right>
    = {1 \over v_{\rm n}} \int_{\vec{v}_{\rm t}} v_{\rm r} \, \sigma(E_{\rm r}) \, M(v_{\rm t}) \, d\vec{v}_{\rm t}
    = {1 \over v_{\rm n}} \int_0^{2 \pi} d\theta \int_{-1}^1 d\mu \int_0^\infty dv_{\rm t} v_{\rm t}^2 v_{\rm r}
        \, \sigma(E_{\rm r}) \, M(v_{\rm t}) \ \ \ \ .
\end{equation}
Here, $\sigma(E_{\rm r})$ is written in terms of the incident particle's energy when its velocity is $v_{\rm r}$, 
$E_{\rm r} = m_{\rm n} v_{\rm r}^2 / 2$, as that is how the data are typically stored in nuclear databases.
The $\theta$ integral yields $2 \pi$ as no factor in the integrand depends on it.  The $\mu$ integration
is performed by using Eq.~\ref{eqRelSpeed} to replace $d\mu$ by $dv_{\rm r}$,
(i.e., $v_{\rm r} \, dv_{\rm r} = -v_{\rm t} \, v_{\rm n} \, d\mu$). Thus, the integral becomes,
\begin{equation}
    \int_0^{2 \pi} d\theta \int_{-1}^1 d\mu \int_0^\infty dv_{\rm t} v_{\rm t}^2 v_{\rm r} \, \sigma(E_{\rm r}) \, M(v_{\rm t})
    = { 2 \pi \over v_{\rm n} } \int_0^\infty dv_{\rm r} \, v_{\rm r}^2 \, \sigma(E_{\rm r}) \,
        \int_{v_{\rm r} - v_{\rm n}}^{v_{\rm r} + v_{\rm n}} dv_{\rm t} \, v_{\rm t} \, M(v_{\rm t})
\end{equation}
where the integration limits for $v_{\rm t}$ have changed to be consistent with Eq.~\ref{eqRelSpeed}.
Using a Maxwellian for $M(v_{\rm t})$,
\begin{equation}
    M(v_{\rm t}) = \left({ m_{\rm t} \over 2 \pi T }\right)^{3/2} \, \exp( - m_{\rm t} v_{\rm t}^2 / ( 2 T ) ) \ \ \ \ ,
\end{equation}
the integral over $v_{\rm t}$ yields,
\begin{equation}	\label{continuousEq}
    \left<\sigma(E_{\rm n},T)\right> = { 1 \over v_{\rm n}^2 v_{\rm T} \sqrt{\pi} } \ \int_0^\infty dv_{\rm r} \,
        v_{\rm r}^2 \, \sigma(E_{\rm r}) \left[ \exp\left( -\left( { v_{\rm r} - v_{\rm n} \over v_{\rm T} } \right)^2 \right)
        - \exp\left( -\left( { v_{\rm r} + v_{\rm n} \over v_{\rm T } } \right)^2 \right) \right]
\end{equation}
where $v_{\rm T} = \sqrt{2T / m_{\rm t}}$ and $m_{\rm t}$ is the target mass.
This can be broken up into two terms, $\sigma_-$ and $\sigma_+$. The $\sigma_-$ term contains the factor
$\exp( -( ( v_{\rm r} - v_{\rm n} ) / v_{\rm T } )^2 )$
and the other contains the factor $\exp( -( ( v_{\rm r} + v_{\rm n} ) / v_{\rm T } )^2 )$.
The main contribution to $\sigma_-$ is from the region $ | v_{\rm r} - v_{\rm n} | < 5 v_{\rm T }$ and
the main contribution to $\sigma_+$ is from the region $ v_{\rm r} < 5 v_{\rm T }$.

\section{Numerical algorithm}

This section describes how to numerically integrate Eq.~\ref{continuousEq}
for a cross section given as a function of the incident particle's energy.

\subsection{Linear interpolated, point-wise cross section data} \label{lin_pointxsec}
In nuclear databases, $\sigma(E_{\rm r})$ is often given at $N$ points (i.e., $E_i, \sigma_i$) and intermediate values are
interpolated using linear-linear interpolation. In other words, for $E_i < E < E_{i+1}$, $\sigma(E)$ is,
\begin{equation}
	\sigma(E) = \left( { \sigma_{i+1} - \sigma_i \over E_{i+1} - E_i } \right) \, \left( E - E_i \right) + \sigma_i 
		= \left( { \sigma_{i+1} - \sigma_i \over v_{i+1}^2 - v_i^2 } \right) \, \left( v^2 - v_i^2 \right) + \sigma_i
		\equiv s_i \, \left( v^2 - v_i^2 \right) + \sigma_i \ \ \ \ .
\end{equation}
Substituting this into Eq.~\ref{continuousEq} yields,
\begin{eqnarray}    \label{discreteEq}
    \left<\sigma(E_{\rm n},T)\right> & = & { 1 \over v_{\rm n}^2 v_{\rm T} \sqrt{\pi} } \ \sum_{i=1}^{N-1} \ 
		\int_{v_i}^{v_{i+1}}dv_{\rm r} \, v_{\rm r}^2 \, 
        \left( s_i \left( v_{\rm r}^2 - v_i^2 \right) + \sigma_i \right) \times \nonumber \\
		& & \ \ \ \ \ \ \ \ \left[ \exp\left( -\left( { v_{\rm r} - v_{\rm n} \over v_{\rm T} } \right)^2 \right)
        - \exp\left( -\left( { v_{\rm r} + v_{\rm n} \over v_{\rm T } } \right)^2 \right) \right] \ \ \ \ .
\end{eqnarray}
Breaking this into two parts, $\sigma_-$ for the exponential with the term
$v_{\rm r} - v_{\rm n}$ and $\sigma_+$ for the $v_{\rm r} + v_{\rm n}$ term,
and substituting $x_- = ( v_{\rm r} - v_{\rm n} ) / v_{\rm T}$ into the expression for $\sigma_-$,
with a similar substitution for $\sigma_+$, produces
\begin{equation}	\label{discreteEq2}
	\left<\sigma_\mp(E_{\rm n},T)\right> = { 1 \over v_{\rm n}^2 \sqrt{\pi} } \ \sum_{i=1}^{N-1} \
	\int_{( v_i \mp v_{\rm n} ) / v_{\rm T}}^{( v_{i+1} \mp v_{\rm n} ) / v_{\rm T}} dx \, e^{-x^2} \, g_\mp(x)
\end{equation}
where
\begin{eqnarray}
    g_\mp(x) & = & s_i \, v_{\rm T}^4 \, x^4 \pm 4 \, s_i \, v_{\rm n} \, v_{\rm T}^3 \, x^3 + 
	\left( \sigma_i + s_i \, ( 6 v_{\rm n}^2 - v_i^2 ) \right) \, v_{\rm T}^2 \, x^2 \nonumber \\
	& & \pm 2 \, v_{\rm n} \, \left( \sigma_i + s_i \, ( 2 v_{\rm n}^2 - v_i^2 ) \right) \, v_{\rm T} \, x +
	v_{\rm n}^2 \, \left( \sigma_i + s_i \, ( v_{\rm n}^2 - v_i^2 ) \right) \ \ \ \ .
\end{eqnarray}
Defining $s_{E,i} = ( \sigma_{i+1} - \sigma_i ) / ( E_{i+1} - E_i )$ and $K = T \, m_{\rm n} / ( E_{\rm n} m_{\rm t} )$
the expression for $g_\mp(x) / v_{\rm n}^2$ becomes,
\begin{eqnarray}
    { g_\mp(x) \over v_{\rm n}^2 } & = & s_{E,i} \, K^2 \, E_{\rm n} \, x^4 \pm 4 \, s_{E,i} \, K^{3/2} \, E_{\rm n} \, x^3 +
    \left( \sigma_i + s_{E,i} \, ( 6 E_{\rm n} - E_i ) \right) \, K \, x^2 \nonumber \\
    & & \pm 2 \, \left( \sigma_i + s_{E,i} \, ( 2 E_{\rm n} - E_i ) \right) \, K^{1/2} \, x +
    \left( \sigma_i + s_{E,i} \, ( E_{\rm n} - E_i ) \right) \ \ \ \ .
\end{eqnarray}

\subsubsection{Dealing with end points.} \label{EndPoints}
In equation~\ref{discreteEq} the thermally corrected cross sections for the last points (e.g., $E_N$, $E_{N-1}$
or any point within $5 \, v_{\rm T}$ of $E_N$)
will be incorrect unless the integral is extended beyond the last given point $E_N$. There are three trivial
extensions one can make (and maybe more). The simplest extension is to do nothing. 
For this extension it is easy to show that the last point will be low by about a factor-of-two, since the
integral in Eq.~\ref{discreteEq} ignores the contribution for $E > E_N$. 

A second possible extension is to assume that the cross section is flat for $E \ge E_N$ and to use
the cross section at $E_N$ as its value. In this case the extension yields,
\begin{eqnarray}
	\sigma_{\rm extension} & = & { \sigma(E_N) \over v_n^2 \, \sqrt{\pi} }
            \left[ 2 \, v_{\rm T} \, v_{\rm n} \, e^{-x_2^2} +
            \int_{x1}^{x2} dx e^{-x^2} \, \Big( v_{\rm T}^2 \, x^2 + 2 \, v_{\rm T} \, v_{\rm n} \, x + v_{\rm n}^2 \Big) \right] \nonumber \\
        & = & { \sigma(E_N) \over 4 \, v_{\rm n}^2 \, \sqrt{\pi} } 
		    \Bigg[ \Big( 4 \, v_{\rm T} \, v_{\rm n} + 2 \, v_{\rm T}^2 x_1 \Big) \, e^{-x_1^2} + 
            \Big( 4 \, v_{\rm T} \, v_{\rm n} + 2 \, v_{\rm T}^2 x_2 \Big) \, e^{-x_2^2} \nonumber \\
        & & + \ \sqrt{\pi} \Big( 2 \, v_{\rm n}^2 + v_{\rm T}^2 \Big) \Big( {\rm erf}(x_2) - {\rm erf}(x_1) \Big) \Bigg]
\end{eqnarray}
where $x_1 = ( v_N - v_{\rm n} ) / v_{\rm T}$, $x_2 = ( v_N + v_{\rm n} ) / v_{\rm T}$ and $v_N$ 
is the incident particle's speed evaluated at energy $E_N$.

A third possible extension is to assume that the cross section is $ 1/v$ for $E \ge E_N$ and to use
$\sigma(E > E_N) = \sigma(E_N) \, v_N / v(E)$.  In this case the extension yields,
\begin{eqnarray}
	\sigma_{\rm extension} & = & { \sigma(E_N) \, v_N \over v_{\rm n}^2 \, \sqrt{\pi} }
            \left[ v_{\rm T} \, e^{-x_2^2} + \int_{x1}^{x2} dx \, e^{-x^2} \, ( v_{\rm T} \, x + v_{\rm n} ) \right] \nonumber \\
        & = & { \sigma(E_N) \, v_N \over 2 \, v_{\rm n}^2 \sqrt{\pi} } 
		    \left[ v_{\rm T} \left( e^{-x_2^2 } + e^{-x_1^2} \right) + \sqrt{\pi} \, v_{\rm n} \Big( {\rm erf}(x_2) - {\rm erf}(x_1) \Big) \right]
\end{eqnarray}

\subsection{Multi-group cross section data} \label{mgroupxsec}
In deterministic transport, the cross section data are divided into $G$ groups with energy boundaries
$E_i$ and the cross section $\sigma_g$ for group $g$ is a constant.
Using $v_i = \sqrt{2 E_i / m_{\rm n}}$, Eq.~\ref{continuousEq} becomes,
\begin{equation}
    \left<\sigma_\mp(E_{\rm n},T)\right> = { 1 \over v_{\rm n}^2 \sqrt{\pi} } \ \sum_{i=1}^{G} \ \sigma_g \ 
    \int_{( v_i \mp v_{\rm n} ) / v_{\rm T}}^{( v_{i+1} \mp v_{\rm n} ) / v_{\rm T}} dx \, e^{-x^2} \, 
	\left( v_{\rm T}^2 \, x^2 \pm 2 v_{\rm n} \, v_{\rm T} \, x + v_{\rm n}^2 \right) \ \ \ \ .
\end{equation}
This equation is the same as Eq.~\ref{discreteEq2} with $s_i = 0$. Thus, $g_\mp(x) / v_{\rm n}^2$ becomes,
\begin{equation}
    { g_\mp(x) \over v_{\rm n}^2 } = \sigma_i K \, x^2 \pm 2 \, \sigma_i \, K^{1/2} \, x + \sigma_i \ \ \ \ .
\end{equation}

\subsection{Evaluating integrals of the form $\int dx \, x^n \exp(-x^2)$}
In sections~\ref{lin_pointxsec} and~\ref{mgroupxsec}, integrals of the form
\begin{equation}
	h_{\rm n}(a, b) = { 1 \over \sqrt{\pi} } \int_a^b dx \, x^n e^{-x^2}
\end{equation}
are to be evaluated. These integrals are
\begin{eqnarray}  \label{integrals1}
	{ 1 \over \sqrt{\pi} } \int dx \,     e^{-x^2} & = & { {\rm erf}(x) \over 2 }   \nonumber \\
	{ 1 \over \sqrt{\pi} } \int dx \, x   e^{-x^2} & = & { -e^{-x^2} \over 2 \sqrt{\pi} } \nonumber \\
	{ 1 \over \sqrt{\pi} } \int dx \, x^2 e^{-x^2} & = & { {\rm erf}(x) \over 4 } - 
		{ x \, e^{-x^2} \over 2 \sqrt{\pi} } \nonumber \\
	{ 1 \over \sqrt{\pi} } \int dx \, x^3 e^{-x^2} & = & { -1 \over 2 \sqrt{\pi} } \left( x^2 + 1 \right) e^{-x^2} \nonumber \\
	{ 1 \over \sqrt{\pi} } \int dx \, x^4 e^{-x^2} & = & { 3 \, {\rm erf}(x) \over 8 } - { x \over 4 \sqrt{\pi} } 
		\left( 2 \, x^2 + 3 \right) e^{-x^2} \ \ \ \ .
\end{eqnarray}
When using a computer to evaluate these expressions, the forms in Eq.~\ref{integrals1} will not produce
good results when $b - a$ is small. For small $b - a$ it is better to Taylor series expand the integrands
about the mid point $x_0 = ( b + a ) / 2$. Let $C_{n,i}(x_0)$ be the coefficient for the term
$\exp(-x_0^2) \, ( x - x_0 )^i / i!$ and $\epsilon = b - a$, then the Taylor series is,
\begin{eqnarray}
	\int_a^b dx \sum_{i=0}^{\infty} { C_{n,i}(x_0) \over i! } \, \exp(-x_0^2) \, ( x - x_0 )^i
		& = & \exp(-x_0^2) \; \sum_{i=0}^{\infty} { C_{n,i}(x_0) \over i! } 
		\int_{x_0 + \epsilon / 2}^{x_0 + \epsilon / 2} dx ( x - x_0 )^i \nonumber \\
		& = & \exp(-x_0^2) \; \sum_{i=0 {\rm\;by\;2}}^{\infty} { C_{n,i}(x_0) \, \epsilon^{i+1} \over 2^i \, (i + 1)! } 
\end{eqnarray}
For even $i$ to 6 the $C_{n,i}(x)$ are, \\
for n = 0,
\begin{eqnarray}  \label{C0is}
	C_{0,0} & = & 1 \nonumber \\
	C_{0,2} & = & 2 ( 2 x^2 - 1 ) \nonumber \\
	C_{0,4} & = & 4 ( 4 x^4 - 12 x^2 + 3 ) \nonumber \\
	C_{0,6} & = & 8 ( 8 x^6 - 60 x^4 + 90 x^2 - 15 )                \hspace{1.55in}
\end{eqnarray}
for n = 1,
\begin{eqnarray}  \label{C1is}
    C_{1,0} & = &   x \nonumber \\
    C_{1,2} & = & 2 x ( 2 x^2 - 3 ) \nonumber \\
    C_{1,4} & = & 4 x ( 4 x^4 - 20 x^2 + 15 ) \nonumber \\
    C_{1,6} & = & 8 x ( 8 x^6 - 84 x^4 + 210 x^2 - 105 )            \hspace{1.3in}
\end{eqnarray}
for n = 2,
\begin{eqnarray}  \label{C2is}
    C_{2,0} & = & x^2 \nonumber \\
    C_{2,2} & = & 2 ( 2 x^4 - 5 x^2 + 1 ) \nonumber \\
    C_{2,4} & = & 4 ( 4 x^6 - 28 x^4 + 39 x^2 - 6 ) \nonumber \\
    C_{2,6} & = & 8 ( 8 x^8 - 108 x^6 + 390 x^4 - 375 x^2 + 45 )    \hspace{0.85in}
\end{eqnarray}
for n = 3,
\begin{eqnarray}  \label{C3is}
    C_{3,0} & = & x^3 \nonumber \\
    C_{3,2} & = & 2 x ( 2 x^4 - 7 x^2 + 3 ) \nonumber \\
    C_{3,4} & = & 4 x ( 4 x^6 - 36 x^4 + 75 x^2 - 30 ) \nonumber \\
    C_{3,6} & = & 8 x ( 8 x^8 - 132 x^6 + 630 x^4 - 945 x^2 + 315 ) \hspace{.65in}
\end{eqnarray}
for n = 4,
\begin{eqnarray}  \label{C4is}
    C_{4,0} & = & x^4 \nonumber \\
    C_{4,2} & = & 2 x^2 ( 2 x^4 - 9 x^2 + 6 ) \nonumber \\
    C_{4,4} & = & 4 ( 4 x^8 - 44 x^6 + 123 x^4 - 84 x^2 + 6 ) \nonumber \\
    C_{4,6} & = & 8 ( 8 x^{10} - 156 x^8 + 930 x^6 - 1935 x^4 + 1170 x^2 - 90 )
\end{eqnarray}

\section{Heating a heated cross section} \label{HeatingHeated}
Often the cross section to be heated to temperature $T$ as already been heated to temperature $T_1$. As long as the $T \ge T_1$,
Eq.~\ref{continuousEq} is still valid provided~\cite{RedC} the temperature used in Eq.~\ref{continuousEq} is $T_2 = T - T_1$. 
This sections goes through the proof of this statement.

If a cross section heated to temperature $T_1$ is inserted into the right-hand-side of Eq.~\ref{continuousEq}, we have
\begin{eqnarray}	\label{continuousEq2}
    \left<\sigma(v_{\rm n},T)\right> 
        & = & { 1 \over v_{\rm n}^2 v_{{\rm T}_2} \sqrt{\pi} } \ \int_0^\infty dv_{\rm r} \,
            v_{\rm r}^2 \, \left<\sigma(v_{\rm r},T_1)\right> \left[ e^{-( ( v_{\rm r} - v_{\rm n} ) / v_{{\rm T}_2} )^2 } 
            - e^{-( ( { v_{\rm r} + v_{\rm n} ) / v_{{\rm T}_2} } )^2 } \right] \nonumber \\
        & = & { 1 \over v_{\rm n}^2 v_{{\rm T}_2} \sqrt{\pi} } \ \int_0^\infty dv_{\rm r} \, v_{\rm r}^2 \, 
            \Bigg\{ { 1 \over v_{\rm r}^2 v_{{\rm T}_1} \sqrt{\pi} } \ \int_0^\infty dv_{\rm r'} \, v_{\rm r'}^2 \, \sigma(v_{\rm r'}) \nonumber \\
        & & \times \left[ e^{-( ( v_{\rm r'} - v_{\rm r} ) / v_{{\rm T}_1})^2 } - e^{-(( v_{\rm r'} + v_{\rm r} ) / v_{{\rm T}_1} )^2 } \right] \Bigg\} 
            \left[ e^{-( ( v_{\rm r} - v_{\rm n} ) / v_{{\rm T}_2} )^2 }
            - e^{-( ( v_{\rm r} + v_{\rm n} ) / v_{{\rm T}_2 } )^2 } \right] \nonumber \\
        & = & { 1 \over v_{\rm n}^2 \, v_{{\rm T}_2} \, v_{{\rm T}_1} \, \pi } \ \int_0^\infty dv_{\rm r'} \, v_{\rm r'}^2 \, \sigma(v_{\rm r'}) \, 
            \int_0^\infty dv_{\rm r} \nonumber \\
        & & \times \left[ e^{-( ( v_{\rm r'} - v_{\rm r} ) / v_{{\rm T}_1} )^2 } - e^{-( ( v_{\rm r'} + v_{\rm r} ) / v_{{\rm T}_1} )^2 } \right]
            \left[ e^{-( ( v_{\rm r} - v_{\rm n} ) / v_{{\rm T}_2} )^2} - e^{-( ( v_{\rm r} + v_{\rm n} ) / v_{{\rm T}_2 } )^2 } \right]
\end{eqnarray}
The integration over $v_{\rm r}$ is
\begin{equation}
    \int_0^\infty dv_{\rm r} \left[ e^{-( ( v_{\rm r'} - v_{\rm r} ) / v_{{\rm T}_1} )^2 } - e^{-( ( v_{\rm r'} + v_{\rm r} ) / v_{{\rm T}_1} )^2 } \right]
        \left[ e^{-( ( v_{\rm r} - v_{\rm n} ) / v_{{\rm T}_2} )^2} - e^{-( ( v_{\rm r} + v_{\rm n} ) / v_{{\rm T}_2 } )^2 } \right]
\end{equation}
After substituting $v_{\rm r} \rightarrow -v_{\rm r}$ into two of the terms, one each of the positive and negative terms, this equation simplifies to
\begin{equation} \label{vrIntegral}
    \int_{-\infty}^\infty dv_{\rm r} \left\{
    \left[ e^{-( ( v_{\rm r'} - v_{\rm r} ) / v_{{\rm T}_1} )^2 } \, e^{-( ( v_{\rm r} - v_{\rm n} ) / v_{{\rm T}_2} )^2 } \right] -
    \left[ e^{-( ( v_{\rm r'} + v_{\rm r} ) / v_{{\rm T}_1} )^2 } \, e^{-( ( v_{\rm r} - v_{\rm n} ) / v_{{\rm T}_2} )^2 } \right] \right\}
\end{equation}
Using the equation for the convolution of two Gaussians,\footnote{The equation for the convolutions of two Gaussians is
$$
    \int_{-\infty}^\infty dx' \,
    { e^{-(x' - x_1)^2/(2 \, \sigma_1^2)} \over \sqrt{2 \pi} \ \sigma_1 } \, 
    { e^{-(x - x')^2/(2 \, \sigma_2^2)} \over \sqrt{2 \pi} \ \sigma_2 } \, 
    = { e^{-(x - x_1)^2/( 2 \, ( \sigma_1^2 + \sigma_2^2 ) )} \over \sqrt{ 2 \pi \, ( \sigma_1^2 + \sigma_2^2 ) } }
$$} Eq.~\ref{vrIntegral} integrates to
\begin{equation} 
    { \sqrt{ \pi } \, v_{{\rm T}_1} \, v_{{\rm T}_2} \over v_{\rm T} } \left[ e^{-( ( v_{\rm r'} - v_{\rm n} ) / v_{{\rm T}} )^2 } -
        \, e^{-( ( v_{\rm r'} + v_{\rm n} ) / v_{{\rm T}} )^2 } \right]
\end{equation}
where $v_{\rm T} = \sqrt{v_{{\rm T}_1}^2 + v_{{\rm T}_2}^2} = \sqrt{ 2 \, ( T_1 + T_2 ) / m_{\rm n} } = \sqrt{ 2 \, T / m_{\rm n}}$.
Therefore, Eq.~\ref{continuousEq2} becomes
\begin{equation}
    \left<\sigma(v_{\rm n},T)\right> = { 1 \over v_{\rm n}^2 v_{\rm T} \sqrt{\pi} } \ \int_0^\infty dv_{\rm r'} \,
        v_{\rm r'}^2 \, \sigma(v_{\rm r'}) \left[ \exp\left( -\left( { v_{\rm r'} - v_{\rm n} \over v_{\rm T} } \right)^2 \right)
        - \exp\left( -\left( { v_{\rm r'} + v_{\rm n} \over v_{\rm T } } \right)^2 \right) \right]
\end{equation}
and completes the proof.

\section{Routine to thermally broaden a cross section}  \label{Routines}

A C routine called \CRoutineS has been written to thermally broaden a piecewise, linear-linear cross section that
is a function of the incident particle's energy.  This section describes it and a python interface to it.

\subsection{\CRoutineS C library}

A library called \CLibrary \ contains\footnote{This library also contains the external routine 
\CRoutine\_\-integrate\_\-xn\_\-qauss. This routine is only for testing and should not, in general, be used.}
the C-routine \CRoutineS for heating cross sections. The file \CRoutine.h is its header file. The routine \CRoutineS
has the following C declaration:

\begin{verbatim}
int crossSectionAdjustForHeatedTarget( 
        crossSectionAdjustForHeatedTarget_limit lowerlimit,     /* Input */
        crossSectionAdjustForHeatedTarget_limit upperlimit,     /* Input */
        crossSectionAdjustForHeatedTarget_info *info,           /* Input/Output */
        double EMin,                                            /* Input */ 
        double massRatio,                                       /* Input */
        double T,                                               /* Input */
        double fInterpolation,                                  /* Input */
        int nPairs,                                             /* Input */
        double *E_cs_in,                                        /* Input */
        double **E_cs_out );                                    /* Output */
\end{verbatim}
Here, {\bf Input} ({\bf Output}) means the argument is used for input (output).
The cross section to be heated (see section~\ref{HeatingHeated} if it has already been heated) contains {\bf nPairs} of piecewise, 
linear-linear (energy,cross section) points given by {\bf E\_cs\_in}. For example, if at energies 1e-11, 1e-10, 1e-4, 1. and 20
the cross section is 12.3, 9.2, 9.1, 4.3 and 11.9 respectively, then a C calling routine could define {\bf nPairs} and {\bf E\_cs\_in} as:
\label{EcsinExample}
\begin{verbatim}
    int nPairs = 5;
    double E_cs_in[10] = { 1e-11, 12.3, 1e-10, 9.2, 1e-4, 9.1, 1., 4.3, 20., 11.9 };
\end{verbatim}
{\bf T} is the temperature of the target (see section~\ref{HeatingHeated}), {\bf massRatio} is the ratio of the target's mass to the incident particle's mass
and {\bf fInterpolation} is the desired accuracy of the heated, piecewise, linear-linear cross section (e.g., {\bf fInterpolation} = 1e-3 means that
the heated cross section data should be piecewise, linear-linear to 0.1\%). Currenlty, {\bf fInterpolation} is restricted to be between
$10^{-6}$ and 0.1.
The arguments {\bf EMin}, {\bf T} and the energy values in {\bf E\_cs\_in} must all be in the same energy unit (e.g., MeV).
If the return value from \CRoutineS is positive, then it is the number of (energy, cross section) data pairs return in {\bf E\_cs\_out}.
The memory for {\bf E\_cs\_out} is allocated by \CRoutineS and must be freed by the calling routine. If the return value from \CRoutineS
is negative, then an has error occurred. The following table list the error values and their meaning:

\begin{verbatim}
    -1      nPair is less than 2.
    -2      massRatio is less than or equal to 0.
    -3      The first energy point is less than or equal to 0.
    -4      The temperature is less than or equal to 0.
    -5      The i^th energy value is greater than the i^th + 1 energy value.
    -6      Could not allocate internal memory.
    -7      Could not allocate memory for E_cs_out.
   -11      Could not allocate internal memory.
\end{verbatim}

The inputted cross section has domain $E_{\rm min}$ = E\_cs\_in[0] to $E_{\rm max}$ = E\_cs\_in[2 $\times$ nPairs - 2]. The arguments {\bf lowerlimit}
and {\bf upperlimit} define how the cross section data should be extended below $E_{\rm min}$ and above $E_{\rm max}$, respectively (see 
section~\ref{EndPoints}). The type crossSectionAdjustForHeatedTarget\_limit is a C enum with the following defined values:
\begin{description}
\item[crossSectionAdjustForHeatedTarget\_limit\_one\_over\_v:] The cross section is extended as $1/v$.

\item[crossSectionAdjustForHeatedTarget\_limit\_constant:] The first (last) cross section point is used as the value below (above) the first (last)
    point.

\item[crossSectionAdjustForHeatedTarget\_limit\_threshold:] The cross section is zero outside the domain. For example, the reaction has
    a threshold at $E_{\rm threshold}$. 
\end{description}
The heated data domain is never extended above $E_{\rm max}$. However, if {\bf lowerlimit} is 
{\bf crossSectionAdjustForHeatedTarget\_\-limit\_\-threshold} the domain will be extended below $E_{\rm min}$ to the
greater of {\bf EMin} and $E_{\rm cutoff}$ where $E_{\rm cutoff} = m_{\rm n} v_{\rm cutoff}^2 / 2$ and $v_{\rm cutoff}$ is
the incident particle's velocity at $E_{\rm min}$ minus 5 times the target's thermal velocity $v_{\rm T}$. If the heated
domain must not extend below the unheated domain, set {\bf EMin} = $E_{\rm min}$. Setting {\bf EMin} $>$ $E_{\rm min}$ is the
same as setting {\bf EMin} = $E_{\rm min}$.

The argument {\bf info} is a C struct with the following elements:
\begin{verbatim}
        int mode;                   /* Input */
        int verbose;                /* Input */
        int InfoStats;              /* Currently not used. */
        int WarningStats;           /* Output */
        int ErrorStats;             /* Currently not used. */
        int bytes;                  /* Output */
        int evaluations;            /* Output */
        double fInterpolation;      /* Output */
\end{verbatim}
Bits in the element {\bf mode} are used to control how \CRoutineS choose the energy points to heat and whether the
heated cross section data should be thinned. Currently, only two bits are used and they can be set with the following
macros:
\begin{description}
\item[ crossSectionAdjustForHeatedTarget\_mode\_all:] By default, \CRoutineS does not heated every point in the input energy domain.
    Instead, it attempts to judiciously pick the points to obtain the desired requested accuracy via {\bf fInterpolation} (e.g., for energies
    less than T / massRatio the heated cross section will vary approximately as 1/v, and for a given {\bf fInterpolation} the required step size
    can be calculated). By judiciously picking points, the code can be much faster (a-factor-of 100 as been observed) if the unheated cross section
    contains many resonances (i.e., points). Or-ing this macro with {\bf mode} causes every point in the unheated data to be heated.
    Independent of how the points are picked, \CRoutineS always test if a point should be added between to consecutive energy points
    based on {\bf fInterpolation}. And it iterates this until one is not required.

\item[crossSectionAdjustForHeatedTarget\_mode\_do\_not\_thin:] By default, the returned cross section is thinned. Or-ing this macro 
    with {\bf mode} results in the data not being thinned.
\end{description}

This routines as 3-levels\footnote{Currently, only warning is implemented.} of messages (informative, warning and error). The number of occurrences
of each is returned by {\bf mode} elements {\bf InfoStats}, {\bf WarningStats} and {\bf ErrorStats}. In addition, if {\bf mode} element {\bf verbose}
is greater than 0, then a message is printed for each error. If it is greater than 1, a message is printed for each error and warning. And, if
it is greater than 2, a message is printed for each error, warning and informative. If memory allocation fails, then {\bf mode} element {\bf bytes}
is set to the number of bytes that were requested.

The {\bf mode} element {\bf evaluations} is set to the number of points heated and the element {\bf fInterpolation} is set to the actual fInterpolation
used.

\subsection{Python interface to the \CRoutineS library}
The routine \CRoutineS can be called from python via the module crossSectionAdjustForHeatedTarget. This
module has one function, also called crossSectionAdjustForHeatedTarget, which has the effective\footnote{Effective
because it is also a C-routine implementing python functions.} python definition of:

\begin{verbatim}
def crossSectionAdjustForHeatedTarget( massRatio, T, E_cs_in, 
    lowerlimit = 'constant', upperlimit = 'oneOverV', fInterpolation = 2e-3,
    heatAllPoints = False, doNotThin = False, EMin = 1e-11 )
\end{verbatim}
Currently, E\_cs\_in must be a python list of [E,$\sigma(E)$] points. Using the example on page~\pageref{EcsinExample}, E\_cs\_in would be:
\begin{verbatim}
    E_cs_in = [ [ 1e-11, 12.3 ], [ 1e-10, 9.2 ], [ 1e-4, 9.1 ],
                [ 1., 4.3 ], [ 20., 11.9 ] ]
\end{verbatim}
The arguments {\bf massRatio}, {\bf T}, {\bf lowerlimit}, {\bf upperlimit}, {\bf fInterpolation} and {\bf EMin}
are as per the C-routine's \CRoutineS arguments. However, the arguments {\bf lowerlimit} and {\bf upperlimit} require string values, and
they must be one of the following: 'constant', 'oneOverV' or 'threshold'. Finally, the arguments {\bf heatAllPoints}
and {\bf doNotThin} are used to set \CRoutine's {\bf mode} argument.

The heated data is returned as a list of [E,$\sigma(E)$] points. If \CRoutineS returns an error, as raise is executed.

\begin{thebibliography}{99}

% \bibitem{OmegaDoc} R.J.\ Howerton, R.E.\ Dye, P.C.\ Giles, J.R.\ Kimlinger, S.T.\ Perkins
%    and E.F.\ Plechaty, {\it Omega Documentation}, UCRL-50400 Vol 25 (1983)

\bibitem{Sigma1} There is another heating package called SIGMA1 written by D. Cullen that heats ENDF formatted files. It is not described here but 
can be found at http://www-nds.iaea.org as part of the PREPRO package.

\bibitem{RedC} Private communications with Dermott (Red) Cullen.

\end{thebibliography}

\end{document}
