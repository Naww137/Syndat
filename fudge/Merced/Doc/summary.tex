\chapter{Summary}
\pagenumbering{arabic}
The \gettransfer\ code is one of the computer programs
used in the conversion of reaction
data from the \xendl\ library~\cite{GND} of evaluated nuclear data
to input for
deterministic particle transport codes.  This data conversion
is managed by the \xndfgen\ python script~\cite{xndfgen}, while the
\gettransfer\ code performs the computation of transfer
matrices used to approximate
the kernel in the integral operator of the Boltzmann
equation.

This document is organized a follows.  Section~\ref{Sec:transfer} explains how the
transfer matrix is used in the discretization of the Boltzmann
equation.  Section~\ref{Sec:interpolate} examines the methods used for interpolation
of data in \xendl.  The remainder of the document is devoted to
a discussion of the considerations involved in computing 
transfer matrices based on the various data formats used in the \xendl\ library.

For discrete 2-body reactions, the processing of angular probability
density data given in the center-of-mass frame is discussed in
Section~\ref{Sec:2-body}.  The treatment here is Newtonian, with a relativistic
version presented in Appendix~\ref{Appendix-relativity}.

Section~\ref{Sec:isotropic-lab} discusses the treatment of the data in \xendl\ used for
isotropic energy probability densities given in the laboratory frame.

Sections \ref{Sec:uncorrelated-lab} through~\ref{Sec:double-diff-formula}
deal with double-differential, energy-angle
probability density data.  Uncorrelated energy-angle probability
density data is presented in Section~\ref{Sec:uncorrelated-lab}.  
One option for energy-angle probability
density data is  as coefficients of Legendre expansions.
This option is discussed in Section~\ref{Sec:Legendre-lab} for data given in the laboratory frame
and in Section~\ref{Ch:Legendre-cm} for center-of-mass data.  The proof of a mathematical
detail used in analysis of the boost for such data is given in Appendix~\ref{Sec:Appendix-B}.
Energy-angle probability densities may also be presented as
tabulated data as discussed in Section~\ref{Sec:joint-table}.  The final form of
energy-angle probability density data is in the form of parameters
of mathematical formulas, and these are taken up in Section~\ref{Sec:double-diff-formula}.

Section~\ref{Sec:gamma-in} deals with special data for incident gammas,
specifically, coherent scattering and Compton scattering.

Finally, the document closes in Section~\ref{Sec:usage} with instructions on how to run
\gettransfer, along with an explanation of the input parameters.
