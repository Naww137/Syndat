{ % localize the newcommands
\newcommand{\Tlabin}{T_{i,\textrm{lab}}}
\newcommand{\plabin}{p_{i,\textrm{lab}}}
\newcommand{\plaba}{p_{\textrm{lab},1}}
\newcommand{\plabb}{p_{\textrm{lab},2}}
\newcommand{\plabc}{p_{\textrm{lab},3}}
\newcommand{\Tlabe}{T_{e,\textrm{lab}}}
\newcommand{\plabe}{p_{e,\textrm{lab}}}
\newcommand{\plabea}{p_{e,\textrm{lab},1}}
\newcommand{\plabeb}{p_{e,\textrm{lab},2}}
\newcommand{\plabec}{p_{e,\textrm{lab},3}}
\newcommand{\Tcmi}{T_{i,\textrm{cm}}}
\newcommand{\Tcmt}{T_{t,\textrm{cm}}}
\newcommand{\Tcme}{T_{e,\textrm{cm}}}
\newcommand{\Tcmr}{T_{r,\textrm{cm}}}
\newcommand{\pcmi}{p_{i,\textrm{cm}}}
\newcommand{\pcma}{p_{\textrm{cm},1}}
\newcommand{\pcmb}{p_{\textrm{cm},2}}
\newcommand{\pcmc}{p_{\textrm{cm},3}}
\newcommand{\pcme}{p_{e,\textrm{cm}}}
\newcommand{\pcmea}{p_{e,\textrm{cm},1}}
\newcommand{\pcmeb}{p_{e,\textrm{cm},2}}
\newcommand{\pcmec}{p_{e,\textrm{cm},3}}
\newcommand{\pcmta}{p_{t,\textrm{cm},1}}
\newcommand{\pcmia}{p_{i,\textrm{cm},1}}
%\newcommand{\mucm}{\mu_{\textrm{cm}}}
%\newcommand{\mulab}{\mu_{\textrm{lab}}}

\chapter{Relativistic 2-body problems}
\label{Appendix-relativity}
In this appendix, relativistic 2-body mechanics is examined from the point
of view of computational physics.  That is, the subtraction
nearly equal numbers is avoided as much as is possible.
The analysis starts with a collision of an incident particle with
a stationary target.  This determines the mapping between
the laboratory frame and the center-of-mass frame.  
The appendix closes with a discussion of
emission after the reaction. 

As is customary in discussions of relativity, the units are
such that the speed of light has the value~$c = 1$.

\section{Initial collision}
For this appendix, $E$ is the total energy of
a system
and $p$ its total momentum.  Thus, for a particle with rest mass~$m_0$
and kinetic energy~$T$, it follows that $E = m_0 + T$. 
\textit{The convention $c = 1$ implies that the data must be such that
particle rest masses and kinetic energies must be given in the same units.}
The analysis makes repeated use of the invariance under Lorentz
transformations of the quantity
\begin{equation}
  S_0 = E^2 - p^2.
  \label{spacetime}
\end{equation}

If the system is a single particle
in a frame in which the particle is stationary, then $S_0 = m_0^2$.
Consequently, for a single particle in
any frame Eq.~(\ref{spacetime}) takes the form
\begin{equation}
  m_0^2 = (m_0 + T)^2 - p^2,
 \label{spacetime-1}
\end{equation}
or
\begin{equation}
  p^2  = 2m_0 T + T^2.
  \label{one-p}
\end{equation}
When it is desired to solve Eq.~(\ref{one-p}) for~$T$ corresponding
to a known value of~$p^2$, it is recommended to use the formula
\begin{equation}
  T = \frac{p^2 }
          {m_0 + \sqrt{ m_0^2 + p^2 }}.
 \label{good-T}
\end{equation}
The relation Eq.~(\ref{good-T}) is computationally more reliable
than the more obvious solution of the quadratic equation Eq.~(\ref{one-p})
$$
  T = -m_0 + \sqrt{ m_0^2 + p^2 }.
$$

Consider the application of Eq.~(\ref{spacetime}) to the system
consisting of a moving incident particle and a target 
at rest in
the laboratory frame.  Suppose that the incident
particle has rest mass $\myi$ and kinetic
energy $\Tlabin$, and let $\mtarg$ be the rest mass of the
target.  Then, for a target at rest it follows from Eq.~(\ref{one-p})
that the initial laboratory-frame momentum is given by
\begin{equation}
  \plabin^2  = 2\myi \Tlabin + \Tlabin^2.
 \label{p-lab-in}
\end{equation}
Consequently, for the system of consisting of a particle
incident on a stationary target
in the laboratory frame, the energy-momentum invariant is
$$
  S = (\mtarg + \myi + \Tlabin)^2 -
      ( 2\myi \Tlabin + \Tlabin^2 ),
$$
This expression simplifies to
\begin{equation}
  S = (\myi + \mtarg )^2 + 2\mtarg \Tlabin.
  \label{S-lab}
\end{equation}

The value of $S$ must be the same when this
system of two particles is considered in the center-of-mass
frame.  Denote the center-of-mass kinetic energy of
the incident particle by $\Tcmi$ and its momentum by~$\pcmi$.
Similarly, let the target have center-of-mass kinetic energy $\Tcmt$, 
and its momentum is~$-\pcmi$.  The energy-momentum invariant
for the system is therefore
\begin{equation}
  S = (\myi + \Tcmi + \mtarg + \Tcmt)^2,
 \label{spacetime-cm}
\end{equation}
the square of the total energy of the system in the center-of-mass frame.
By using Eq.~(\ref{spacetime-1}) on each of the particles, it
is possible to rewrite this as
$$
  S = \left(
    \sqrt{\myi^2 + \pcmi^2 } +
       \sqrt{\mtarg^2 + \pcmi^2 }
   \right)^2.
$$
Upon solving this equation for $\pcmi^2$, it is found that
\begin{equation}
  \pcmi^2 = \frac{ \left[S - (\myi^2 + \mtarg^2)\right]^2 -
                 4 \myi^2 \mtarg^2 }
                {4 S }.
  \label{p-cm}
\end{equation}
An expression for $\pcmi^2$ in terms of the
laboratory incident kinetic energy $\Tlabin$ is obtained by substituting
in Eq.~(\ref{p-cm}) the value of $S$ given by Eq.~(\ref{S-lab}),
\begin{equation}
  \pcmi^2 = \frac{\mtarg^2( 2 \myi \Tlabin + \Tlabin^2)}
              {(\mtarg + \myi )^2 + 2\mtarg \Tlabin}.
  \label{p-cm-1}
\end{equation}
It follows from Eq.~(\ref{one-p}) that this equation may also
be written as
$$
  \pcmi^2 = \frac{\mtarg^2 \plabin^2 }
                {(\mtarg + \myi )^2 + 2\mtarg \Tlabin}.
$$

\section{Mapping between frames}
Consider a coordinate system in which the momentum
$\plabin$ of the incident particle is in the direction of
the first spatial axis.  The boost of the energy-momentum
vector from
the laboratory to the center-of-mass frame then takes the
form
\begin{equation}
  \begin{bmatrix}
    E_{\text{cm}} \\
    \pcma \\
    \pcmb \\
    \pcmc
  \end{bmatrix}
   =
    \begin{bmatrix}
     \cosh \chi & -\sinh \chi & 0 & 0 \\
     -\sinh \chi & \cosh \chi & 0 & 0 \\
     0  & 0 & 1 & 0 \\
     0  & 0 & 0 & 1
  \end{bmatrix}
  \begin{bmatrix}
    E_{\text{lab}} \\
    \plaba \\
    \plabb \\
    \plabc
  \end{bmatrix}.
  \label{R-3}
\end{equation}

Upon applying the second component of the boost Eq.~(\ref{R-3}) to the 
target,
it is found that
$$
  \pcmta = - \mtarg  \sinh \chi.
$$
With the notation that $ | p | $
is the length of the vector~$p$, 
it follows that
$$
   \pcmta =  -\pcmia = - | \pcmi |,
$$
so that
\begin{equation}
   \sinh \chi = \frac{|\pcmi|}{\mtarg}.
 \label{lab-to-cm}
\end{equation}
By using Eq.~(\ref{p-cm-1}), one may conclude that
\begin{equation}
  \sinh \chi = \frac{\sqrt{ 2 \myi \Tlabin + \Tlabin^2}}
    {\sqrt{( \mtarg + \myi )^2 + 2 \mtarg \Tlabin }}.
  \label{lab-to-cm-2}
\end{equation}
Note that except for incident gammas, $\Tlabin$ is much smaller
than the rest mass $\myi$, so that $\chi$ is a
small, positive number.

In the next section of this appendix, for 2-body problems
the center-of-mass energy and momentum of the
emitted particle and residual are determined.  In order to boost these 4-vectors
to the laboratory frame, one may use the inverse of the matrix in
Eq.~(\ref{R-3}), so that
\begin{equation}
  \begin{bmatrix}
    E_{\text{lab}} \\
    \plaba \\
    \plabb \\
    \plabc
  \end{bmatrix}
   =
    \begin{bmatrix}
     \cosh \chi & \sinh \chi & 0 & 0 \\
     \sinh \chi & \cosh \chi & 0 & 0 \\
     0  & 0 & 1 & 0 \\
     0  & 0 & 0 & 1
  \end{bmatrix}
  \begin{bmatrix}
    E_{\text{cm}} \\
    \pcma \\
    \pcmb \\
    \pcmc
  \end{bmatrix}.
  \label{R-3-inv}
\end{equation}

\subsection{Incident photons}
\label{Sec:photon-in}
When the incident particle is a photon, the boost from the 
center-of-mass frame to the laboratory frame must be determined
relativistically, because the mass of the incident particle is zero
but its momentum is nonzero.

In this case, Eq.~(\ref{p-lab-in}) simplifies to
$$
  | \plabin |  =  \Tlabin,
$$
and Eq.~(\ref{lab-to-cm-2}) becomes
$$
  \sinh \chi = \frac{ \Tlabin}
    {\sqrt{ \mtarg^2 + 2 \mtarg \Tlabin }}.
$$
It follows that
$$
  \cosh \chi = \frac{ \mtarg + \Tlabin}
    {\sqrt{ \mtarg^2 + 2 \mtarg \Tlabin }}.
$$



\section{Outgoing particles}
Denote by $\myo$ the rest mass of the
emitted particle and
$\Tcme$ its kinetic energy in the center-of-mass frame.
The convention in \xendl\ is that the energy~$Q$ of the reaction
is specified by the data, and the rest mass  $\mres$ of the residual
is calculated from
\begin{equation}
  \mres = \mtarg + ( \myi - \myo ) - Q.
 \label{Q-2body}
\end{equation}
Let $\Tcmr$ be the kinetic energy of the residual in the center-of-mass frame.
In terms of these variables, the energy-momentum invariant
for the system is the square of the total energy
$$
    S = (\myo + \Tcme + \mres + \Tcmr)^2,
$$
with the same value of $S$ as in Eq.~(\ref{spacetime-cm}).
The argument leading to Eq.~(\ref{p-cm}) shows that the
momentum~$\pcme$ of the emitted particle in the center-of-mass frame
has magnitude given by
\begin{equation}
  \pcme^2 = \frac{ \left[S - (\mres^2 + \myo^2)\right]^2 -
                 4 \mres^2 \myo^2 }
                {4 S }.
  \label{p-cm-e}
\end{equation}

It is not a good idea to use Eq.~(\ref{p-cm-e}) in a
computation, because of its subtraction of nearly equal
numbers.  It is therefore desirable to do some algebraic manipulation
in order to mitigate this problem as much as possible.
As a first step, Eq.~(\ref{p-cm-e}) is rewritten in the
form
\begin{equation}
  4 S \pcme^2 = \left[S - (\mres + \myo)^2\right]
              \left[S - (\mres - \myo)^2\right].
 \label{p-cm-e-1}
\end{equation}
In this expression, the subtraction of nearly equal
numbers is confined to the first factor on the right-hand
side.  For photon emission the two factors are identical.
An analysis of photon emission later, because it offers some
simplifications.

By using the expression for~$S$ in Eq.~(\ref{S-lab}), one
obtains the relation
$$
  S - (\mres + \myo)^2 =
  (\mtarg + \myi)^2 - (\mres + \myo)^2  + 2\mtarg \Tlabin.
$$
In terms of the energy $Q$ of the discrete 2-body reaction
and the parameter
\begin{equation}
  M_T = \mtarg + \mres + \myi + \myo,
 \label{def-M}
\end{equation}
it follows that
$$
    S - (\mres + \myo)^2 =
     M_T Q + 2\mtarg \Tlabin.
$$
Consequently, it is seen that Eq.~(\ref{p-cm-e}) may be
replaced by
\begin{equation}
  \pcme^2 = \frac{ (M_T Q +  2\mtarg \Tlabin )
                 ( M_T Q +  2\mtarg \Tlabin + 4\mres \myo)}
                {4 S }.
  \label{p-cm-e-OK}
\end{equation}

\textbf{Remark.}
It is clear from Eq.~(\ref{p-cm-e-OK}) that for endothermic
reactions ($Q < 0$), the threshold occurs when the incident
particle has kinetic energy
$$
  \Tlabin = \frac{-M_T Q}{2\mtarg}.
$$

In Eq.~(\ref{p-cm-e-OK}) there is subtraction of nearly equal
numbers when the kinetic energy $\Tlabin$ of the incident
particle is just above the threshold in endothermic reactions.
That operation is unavoidable in
the analysis of nuclear reactions. 

Now that $\pcme^2$ has been obtained in Eq.~(\ref{p-cm-e-OK}), one may use
Eq.~(\ref{good-T}) to determine the kinetic energy of the
emitted particle in the center-of-mass frame as
\begin{equation}
  \Tcme = \frac{\pcme^2 }
          {\myo + \sqrt{ \myo^2 + \pcme^2 }}.
 \label{T-cm-e}
\end{equation}

\subsection{The boost to the laboratory frame}
It is often desired to determine the kinetic energy~$\Tlabe$
and momentum~$\plabe$ of the emitted particle in the laboratory frame
for given direction cosine~$\mucm$ in the center-of-mass frame.
It is possible to use the boost 
Eq.~(\ref{R-3-inv}) to determine $\plabe$ as follows.  Recall that
the form of Eq.~(\ref{R-3-inv}) is determined by the requirement that
the first axis of the coordinate system was chosen parallel to~$\plabin$.
Consequently, one has
$$
  \pcmea = \mucm |\pcme|.
$$
If the orientation of the coordinate system is such that
$$
  \pcmec = 0
  \quad \textrm{and} \quad
  \pcmeb \ge 0,
$$
then
$$
  \pcmeb =  |\pcme| \sqrt{ 1 - \mucm^2}.
$$
The momentum components of the boost Eq.~(\ref{R-3-inv}) then
take the form
\begin{equation*}
\begin{split}
  \plabea &= (\myo + \Tcme)\sinh \chi +  \mucm |\pcme| \cosh \chi, \\
  \plabeb &= |\pcme| \sqrt{ 1 - \mucm^2}, \\
  \plabec & = 0.\\
\end{split}
\end{equation*}

The magnitude of the momentum in the laboratory frame is
$$
  |\plabe| = \sqrt{ \plabea^2 + \plabeb^2 + \plabec^2 }.
$$
If $|\plabe| = 0$, the direction cosine $\mulab$ in the laboratory
frame is undetermined.  Otherwise, it is given by
$$
  \mulab = \frac{\plabea}{|\plabe|}.
$$
The kinetic energy $\Tlabe$ is calculated from $|\plabe|$
by using Eq.~(\ref{good-T}).

\subsection{Photon emission}
When the emitted particle is a photon, because $\myo = 0$,
Eqs.~(\ref{p-cm-e-OK}) and (\ref{T-cm-e}) take the simpler form
$$
  E_{e, \textrm{cm}} = \Tcme = |\pcme| =
   \frac{ M_T Q +  2\mtarg \Tlabin }
                {2 \sqrt{S} }.
$$
For given direction cosine~$\mucm$ in the center-of-mass frame,
the energy component of the boost Eq.~(\ref{R-3-inv}) gives the
Doppler shift
$$
   E_{e, \textrm{lab}} = E_{e, \textrm{cm}}
     \left( \cosh \chi + \mucm \sinh \chi \right).
$$
The first component of the momentum of the photon in the laboratory
frame is
$$
  \plabea =  E_{e, \textrm{cm}}
     \left( \sinh \chi + \mucm \cosh \chi \right),
$$
so the direction cosine is
$$
  \mulab = \frac{\sinh \chi + \mucm \cosh \chi}
                {\cosh \chi + \mucm \sinh \chi}.
$$

\newcommand{\mayo}{m_{1,e}}
\newcommand{\mares}{m_{1,r}}
\newcommand{\mbyo}{m_{2,e}}
\newcommand{\mbres}{m_{2,r}}

\newcommand{\muacm}{\mu_{\text{1,cm}}}
\newcommand{\mubcm}{\mu_{\text{2,cm}}}
\newcommand{\mualab}{\mu_{\text{1,lab}}}
\newcommand{\mublab}{\mu_{\text{2,lab}}}

\newcommand{\Vacm}{\textbf{V}_{\text{1,cm}}}
\newcommand{\Valab}{\textbf{V}_{\text{1,lab}}}

\newcommand{\Talabe}{T_{1,e,\textrm{lab}}}
\newcommand{\Tacme}{T_{1,e,\textrm{cm}}}
\newcommand{\Tacmr}{T_{1,r,\textrm{cm}}}
\newcommand{\pacmi}{p_{1,i,\textrm{cm}}}
\newcommand{\pacme}{p_{1,e,\textrm{cm}}}
\newcommand{\pacmea}{p_{1,e,\textrm{cm},1}}
\newcommand{\pacmeb}{p_{1,e,\textrm{cm},2}}
\newcommand{\pacmec}{p_{1,e,\textrm{cm},3}}
\newcommand{\plabi}{p_{i,\textrm{lab}}}
\newcommand{\palabe}{p_{1,e,\textrm{lab}}}
\newcommand{\palabea}{p_{1,e,\textrm{lab},1}}
\newcommand{\palabeb}{p_{1,e,\textrm{lab},2}}
\newcommand{\palabec}{p_{1,e,\textrm{lab},3}}

\newcommand{\Tblabe}{T_{2,e,\textrm{lab}}}
\newcommand{\Tbcme}{T_{2,e,\textrm{cm}}}
\newcommand{\Tbcmr}{T_{2,r,\textrm{cm}}}
\newcommand{\pbcmi}{p_{2,i,\textrm{cm}}}
\newcommand{\pbcme}{p_{2,e,\textrm{cm}}}
\newcommand{\pbcmea}{p_{2,e,\textrm{cm},1}}
\newcommand{\pbcmeb}{p_{2,e,\textrm{cm},2}}
\newcommand{\pbcmec}{p_{2,e,\textrm{cm},3}}
\newcommand{\pbThetae}{p_{2,e,{\Theta_1}}}
\newcommand{\pbThetaea}{p_{2,e,{\Theta_1},1}}
\newcommand{\pbThetaeb}{p_{2,e,{\Theta_1},2}}
\newcommand{\pbThetaec}{p_{2,e,{\Theta_1},3}}
\newcommand{\pblabe}{p_{2,e,\textrm{lab}}}
\newcommand{\pblabea}{p_{2,e,\textrm{lab},1}}
\newcommand{\pblabeb}{p_{2,e,\textrm{lab},2}}
\newcommand{\pblabec}{p_{2,e,\textrm{lab},3}}

\section{Two-step 2-body reactions}
\label{Sec:2-step-2-body-rel}
This section presents a relativistic treatment of the 2-step
discrete 2-body reaction discussed in Section~\ref{Sec:2-step-2-body}.
The first step of the reaction is inelastic scattering, with an
excited outgoing particle.  This excited particle then splits into a final outgoing particle
and a residual.  The notation used for this reaction is as in
Section~\ref{Sec:2-step-2-body} so that for the first step, $\mayo$ is the rest mass of
the excited outgoing particle, $\mares$ is the residual, and $Q_1$ is
the $Q$-value.  For the second step of the reaction, $\mbyo$ denotes the rest mass of
the final outgoing particle, $\mbres$ is the residual, and $Q_2$ is
the $Q$-value.

For the first step of the reaction, the energy-momentum invariant $S_1$
is as in Eq.~(\ref{spacetime-cm}),
\begin{equation}
  S_1 = (\myi + \Tcmi + \mtarg + \Tcmt)^2.
 \label{spacetime-cm-step1}
\end{equation}
As mentioned in Eq.~(\ref{2-step-mayo}), the rest mass of
the excited outgoing particle from the first step is given by
$$
  \mayo = \myi + \mtarg - \mares - Q_1.
$$

For the excited particle emitted in the first step of the
reaction, the boost Eq.~(\ref{R-3-inv}) from the center-of-mass frame to the
laboratory frame takes the form
\begin{equation}
  \begin{bmatrix}
    \mayo + \Talabe \\
    \palabea \\
    \palabeb \\
    \palabec
  \end{bmatrix}
   =
    \begin{bmatrix}
     \cosh \chi_1 & \sinh \chi_1 & 0 & 0 \\
     \sinh \chi_1 & \cosh \chi_1 & 0 & 0 \\
     0  & 0 & 1 & 0 \\
     0  & 0 & 0 & 1
  \end{bmatrix}
  \begin{bmatrix}
    \mayo + \Tacme \\
    \pacmea \\
    \pacmeb \\
    \pacmec
  \end{bmatrix}.
  \label{R-3-inv-step1}
\end{equation}
As in Eq.~(\ref{lab-to-cm-2}),
the parameter~$\chi_1$ for this boost
is is given by
\begin{equation}
  \sinh \chi_1 = \frac{\sqrt{ 2 \myi \Tlabin + \Tlabin^2}}
    {\sqrt{( \mtarg + \myi )^2 + 2 \mtarg \Tlabin }}.
  \label{lab-to-cm-2-step1}
\end{equation}

For step~1 of the reaction, the total mass of the particles
in Eq.~(\ref{def-M}) is given by
\begin{equation}
  M_{1,T} = \mtarg + \mares + \myi + \mayo,
 \label{def-M-step1}
\end{equation}
and for the momentum of the excited outgoing particle in
the frame of the center of mass of the initial collision the
relation Eq.~(\ref{p-cm-e-OK}) takes the form
\begin{equation}
  |\pacme|^2 = \frac{ (M_{1,T} Q_1 +  2\mtarg \Tlabin )
                 ( M_{1,T} Q_1 +  2\mtarg \Tlabin + 4\mayo \mares)}
                {4 S_1 }.
  \label{p-cm-e-OK-step1}
\end{equation}
Then by Eq.~(\ref{T-cm-e}), the kinetic energy of the excited
particle emitted in the first step is
\begin{equation}
  \Tacme = \frac{|\pacme|^2 }
          {\mayo + \sqrt{ \mayo^2 + |\pacme|^2 }}.
 \label{T-cm-e-step1}
\end{equation}

The remaining items to be determined on the right-hand
side of Eq.~(\ref{R-3-inv-step1}) are the components of
the momentum~$\pacme$.
Let $\theta_1$ denote the angle between $\pacme$ and the
center-of-mass momentum~$\pacmi$ of the incident particle.
The coordinate system may be
chosen so that $\pacmec = 0$ and so that~$\theta_1$
satisfies the condition
$$
  0 \le \theta_1 \le \pi.
$$
Then with $\muacm = \cos \theta_1$, the components 
of~$\pacme$ are
\begin{equation*}
\begin{split}
  \pacmea &= | \pacme | \, \muacm, \\
  \pacmeb &=     |   \pacme | \sqrt{ 1 - \muacm^2 }, \\
  \pacmec &=     0.
\end{split}
%\label{pcme-abc}
\end{equation*}

Now that all of the variables on the  right-hand
side of Eq.~(\ref{R-3-inv-step1}) are identified, those on
the left-hand side are determined as well.  In particular,
the magnitude of the momentum~$\palabe$ in the laboratory
frame of the excited
outgoing particle from step~1 of the reaction is given by
\begin{equation}
  | \palabe | = \sqrt{ \palabea^2 + \palabeb^2 + \palabec^2 }\,.
 \label{plab-rel-step1}
\end{equation}
The direction cosine~$\mualab$ for step~1 of the reaction is
defined as
\begin{equation}
  \mualab = 
  \begin{cases}
     \palabea / | \palabe | &
       \text{for $| \palabe | \ne 0$,} \\
     1 & \text{otherwise.}
   \end{cases}
 \label{mu-lab-step1}
\end{equation}
Consequently, for $| \palabe | \ne 0$ the angle between
$\palabe$ and the laboratory-frame momentum~$\plabi$
of the incident particle is
\begin{equation}
   \Theta_1 = \cos^{-1} \mualab.
 \label{def-Theta1}
\end{equation}

For the breakup of the excited particle emitted
by step~1, the initial analysis takes place in the 
frame in which this particle is stationary.  In this frame
the energy-momentum invariant is
$$
  S_2 = \mayo^2.
$$
The total mass for the breakup is
$$
  M_{2,T} = \mayo + \mbres + \mbyo.
$$
For step~2 of the reaction, Eq.~(\ref{p-cm-e-OK}) for the
magnitude of the momentum takes the form
\begin{equation}
  |\pbcme|^2 = \frac{ M_{2,T} Q_2 
                 ( M_{2,T} Q_2 + 4\mbyo \mbres)}
                {4 S_2 }.
  \label{p-cm-e-OK-step2}
\end{equation}
In this frame, the
kinetic energy of the final emitted particle is
\begin{equation}
  \Tbcme = \frac{|\pbcme|^2 }
          {\mbyo + \sqrt{ \mbyo^2 + |\pbcme|^2 }}.
  \label{T-cm-e-OK-step2}
\end{equation}

For the relativistic treatment of this 2-step reaction,
the elements $\Inum_{g,h,\ell}$ of the transfer matrix are calculated
as in the Newtonian case Eq.~(\ref{2-step-muEint}), namely by
\begin{equation}
    \Inum_{g,h,\ell} =
     \int_{\calE_g} dE \, \sigma ( E ) w(E) \widetilde \phi_\ell(E)
    \int_{\muacm} d\muacm  \, g(\muacm \mid E)
    \int_{\Sigma_{1,h}} d\sigma_1 \, P_\ell( \mulab ).
  \label{2-step-muEint-2step}
\end{equation}
One difference from the Newtonian case is that, here
the isotropic emission in the breakup step is reflected
in the uniform distribution of the momentum
$$
   \pbcme =
  \begin{bmatrix}
    \pbcmea \\
    \pbcmeb \\
    \pbcmec
  \end{bmatrix}
$$
over the hemisphere~$\Sigma_1$ given by the relation
\begin{equation}
  \pbcmec =
  \sqrt{ |  \pbcme |^2 -  \pbcmea^2 -  \pbcmeb^2 }
  \label{2-step-def-Sigma0}
\end{equation}
for
$$
  \pbcmea^2 +  \pbcmeb^2 \le |  \pbcme |^2.
$$
The differential $ d\sigma_1$ in Eq.~(\ref{2-step-muEint-2step})
is the differential surface area
on the hemisphere~$\Sigma_1$ normalized so that
$$
  \int_{\Sigma_1} d\sigma_1 = 1.
$$

Another difference in Eq.~(\ref{2-step-muEint-2step}) from the 
Newtonian case is the need to construct a boost of the
energy-momentum vector
\begin{equation}
  \begin{bmatrix}
    \mbyo + \Tbcme \\
    \pbcmea \\
    \pbcmeb \\
    \pbcmec
  \end{bmatrix}
  \label{2-step-final-Ep-cm}
\end{equation}
to the laboratory frame of the initial collision
\begin{equation}
  \begin{bmatrix}
    \mbyo + \Tblabe \\
    \pblabea \\
    \pblabeb \\
    \pblabec
  \end{bmatrix}.
  \label{2-step-final-Ep-lab}
\end{equation}
This boost will be constructed in such a way that the $\pblabea$
momentum axis in Eq.~(\ref{2-step-final-Ep-lab}) is parallel to
the momentum $\plabin$ of the initial incident particle.
In the integral in Eq.~(\ref{2-step-muEint-2step}), $\Sigma_{1,h}$
denotes the portion of $\Sigma_1$ on which the kinetic energy
$\Tbcme$ of the final emitted particle in the laboratory frame
is in the outgoing energy bin~$\calE_h'$.  

The direction cosine~$\mulab$
in Eq.~(\ref{2-step-muEint-2step}) is taken to be
\begin{equation}
  \mulab =
  \begin{cases}
    \pblabea / | \pblabe | &
       \text {if $ | \pblabe | > 0$,} \\
     1 & \text{otherwise.}
  \end{cases}
  \label{2-step-final-mu-lab}
\end{equation}

The boost of the energy-momentum vector for the final
emitted particle from the frame of Eq~(\ref{2-step-final-Ep-cm})
to the laboratory frame of Eq.~(\ref{2-step-final-Ep-lab})
will be constructed as an initial boost followed by a rotation
of the momentum subspace.

For the excited particle emitted in step~1, the boost
\begin{equation}
  \begin{bmatrix}
    \mayo + \Talabe \\
    |\palabe|  \\
    0 \\
    0
  \end{bmatrix}
   =
    \begin{bmatrix}
     \cosh \chi_2 & \sinh \chi_2 & 0 & 0 \\
     \sinh \chi_2 & \cosh \chi_2 & 0 & 0 \\
     0  & 0 & 1 & 0 \\
     0  & 0 & 0 & 1
  \end{bmatrix}
  \begin{bmatrix}
    \mayo \\
    0 \\
    0 \\
    0
  \end{bmatrix}
  \label{boost-chi2}
\end{equation}
maps the energy and momentum from the center-of-mass
frame of the breakup reaction to a laboratory frame.
The value of~$\chi_2$ in the boost Eq.~(\ref{boost-chi2}) 
is determined by
$$
  \sinh \chi_2 = \frac{  |\palabe| }{ \mayo}
$$
with~$|\palabe|$ given by Eq.~(\ref{plab-rel-step1}).

An application of the boost Eq.~(\ref{boost-chi2}) to the
energy-momentum vector for the
particle emitted in step~2 of the reaction gives the result
\begin{equation}
  \begin{bmatrix}
    \mbyo + \Tblabe \\
    \pbThetaea \\
    \pbThetaeb \\
    \pbThetaec
  \end{bmatrix}
   =
    \begin{bmatrix}
     \cosh \chi_2 & \sinh \chi_2 & 0 & 0 \\
     \sinh \chi_2 & \cosh \chi_2 & 0 & 0 \\
     0  & 0 & 1 & 0 \\
     0  & 0 & 0 & 1
  \end{bmatrix}
  \begin{bmatrix}
    \mbyo + \Tbcme \\
    \pbcmea \\
    \pbcmeb \\
    \pbcmec
  \end{bmatrix}.
  \label{R-3-inv-step2}
\end{equation}

The mapping Eq.~(\ref{R-3-inv-step2}) takes the
center-of-mass energy-$\pbcmea$ plane onto the laboratory
energy-$\pbThetaea$ plane.  The construction Eq.~(\ref{boost-chi2})
ensures that this plane contains the vector~$\palabe$.  Both
$\palabe$ and the momentum~$\plabin$ of the initial incident particle
lie in the original laboratory momentum subspace, but Eqs.~(\ref{mu-lab-step1})
and~(\ref{def-Theta1}) show that these vectors are separated by
the angle~$\Theta_1$.  It is therefore desirable to rotate the
$$
  \pbThetae =
   \begin{bmatrix}
     \pbThetaea \\
    \pbThetaeb \\
    \pbThetaec
   \end{bmatrix}
$$
momentum subspace by the mapping
\begin{equation}
  \begin{bmatrix}
    \mbyo + \Tblabe \\
    \pblabea \\
    \pblabeb \\
    \pblabec
  \end{bmatrix}
   =
    \begin{bmatrix}
     1 & 0 & 0 & 0 \\
     0 & \cos \Theta_1 & -\sin \Theta_1 & 0 \\
     0  & \sin \Theta_1 & \cos \Theta_1 & 0 \\
     0  & 0 & 0 & 1
  \end{bmatrix}
  \begin{bmatrix}
    \mbyo + \Tblabe \\
    \pbThetaea \\
    \pbThetaeb \\
    \pbThetaec
   \end{bmatrix}.
  \label{R-3-inv-rotate}
\end{equation}
Recall that $\Theta_1$ is set equal to~$0$ if $|\palabe | = 0$, so that 
Eq.~(\ref{R-3-inv-rotate}) is also valid in
this case.  For step~2 of the reaction, the mapping of the
energy-momentum vector from the center-of-mass frame
to the initial laboratory frame is accomplished by the boost
Eq.~(\ref{R-3-inv-step2}) followed by the rotation Eq.~(\ref{R-3-inv-rotate}).

One should not compute the kinetic energy~$\Tblabe$ by subtracting
the rest mass~$\mbyo$ from the total energy in the laboratory frame.
It is better to first calculate the square of the length of the momentum
$$
  |\pblabe|^2 = \pblabea^2 + \pblabeb^2 + \pblabec^2
$$
and to use
\begin{equation*}
  \Tblabe = \frac{|\pblabe|^2 }
          {\mbyo + \sqrt{ \mbyo^2 + |\pblabe|^2 }}.
  %\label{T-lab-e-OK-step2}
\end{equation*}

It remains to describe the details of the computation of
the integral
$$
    \int_{\Sigma_{1,h}} d\sigma_1 \, P_\ell( \mulab ),
$$
which appears in Eq.~(\ref{2-step-muEint-2step}).
This is an integral over a subset of the hemisphere
Eq.~(\ref{2-step-def-Sigma0}) of the center-of-mass
momentum subspace for the breakup step of the reaction.
In the Newtonian analysis of this reaction in Section~\ref{Sec:2-step-2-body},
a uniform measure~$d\sigma_0$ was derived for the
hemisphere~$\Sigma_0$ of Eq.~(\ref{2-step-def-zeta}) in
terms of the variables $\mubcm$ and~$w$ of Eqs.~(\ref{2-step-def-xi})
and~(\ref{2-step-def-eta}).  The analogous uniform measure~$d\sigma_1$
on the hemisphere~$\Sigma_1$ of Eq.~(\ref{2-step-def-Sigma0})
is given by the relations
\begin{alignat}{2}
  \pbcmea &= |\pbcme| \, \mubcm
    \quad &\text{for $-1 \le \mubcm \le 1$,} 
      \label{2-step-def-xi-rel} \\
  \pbcmeb &= |\pbcme| \sqrt{ 1 - \mubcm^2 } \, \sin w     
    \quad &\text{for $-\pi/2 \le w \le \pi/2$}.
   \label{2-step-def-eta-rel}
\end{alignat}
With this notation, the expression
$$
  d\sigma_1 = \frac{ d \mubcm \, dw}{2 \pi}
$$
gives a uniform measure on~$\Sigma_1$.

\begin{figure}
% mapping to laboratory coordinates, 2-step, 2-body reactions
\begin{center}

\begin{tikzpicture}
% the axes
 \draw (-5, 0) -- ( 5, 0);
 \draw (-3.536, -3.536) -- (3.536, 3.536);
 \draw( 0,-4) -- (0, 5);
% sphere
 \draw (4, 0) arc (0: 180: 4);
 \draw (0, 0) ellipse ( 4cm and 2cm );
 \draw (-3.666, 0.8) -- (3.666, 0.8);
 \draw (0, 0) -- (3.666, 0.8);
  \draw[-{>[scale=2.5,
          length=5,
          width=3]},line width=0.4pt] (0, 0) -- (1.511,  3.435);
 \draw [dashed] (1.511, 0.8) -- (1.511,  3.435);
% elliptic arcs
 \draw  ( 3.666,  0.800) --
 ( 3.618,  0.996) --
 ( 3.560,  1.186) --
 ( 3.493,  1.370) --
 ( 3.417,  1.549) --
 ( 3.333,  1.723) --
 ( 3.241,  1.891) --
 ( 3.140,  2.054) --
 ( 3.032,  2.211) --
 ( 2.915,  2.362) --
 ( 2.791,  2.508) --
 ( 2.659,  2.648) --
 ( 2.519,  2.781) --
 ( 2.371,  2.909) --
 ( 2.215,  3.029) --
 ( 2.051,  3.142) --
 ( 1.879,  3.248) --
 ( 1.699,  3.346) --
 ( 1.511,  3.435) --
 ( 1.315,  3.514) --
 ( 1.112,  3.584) --
 ( 0.902,  3.642) --
 ( 0.685,  3.689) --
 ( 0.462,  3.724) --
 ( 0.234,  3.745) --
 (0.000,  3.752);
 \draw (-3.666,  0.800) --
 (-3.618,  0.996) --
 (-3.560,  1.186) --
 (-3.493,  1.370) --
 (-3.417,  1.549) --
 (-3.333,  1.723) --
 (-3.241,  1.891) --
 (-3.140,  2.054) --
 (-3.032,  2.211) --
 (-2.915,  2.362) --
 (-2.791,  2.508) --
 (-2.659,  2.648) --
 (-2.519,  2.781) --
 (-2.371,  2.909) --
 (-2.215,  3.029) --
 (-2.051,  3.142) --
 (-1.879,  3.248) --
 (-1.699,  3.346) --
 (-1.511,  3.435) --
 (-1.315,  3.514) --
 (-1.112,  3.584) --
 (-0.902,  3.642) --
 (-0.685,  3.689) --
 (-0.462,  3.724) --
 (-0.234,  3.745) --
 ( 0.000,  3.752);
 \draw (-0.358, -0.358) --
 (-0.322, -0.366) --
 (-0.286, -0.374) --
 (-0.249, -0.380) --
 (-0.212, -0.386) --
 (-0.174, -0.390) --
 (-0.135, -0.394) --
 (-0.097, -0.397) --
 (-0.058, -0.399) --
 (-0.019, -0.400) --
 ( 0.020, -0.400) --
 ( 0.060, -0.399) --
 ( 0.098, -0.397) --
 ( 0.137, -0.394) --
 ( 0.176, -0.390) --
 ( 0.213, -0.385) --
 ( 0.251, -0.380) --
 ( 0.288, -0.373) --
 ( 0.324, -0.366) --
 ( 0.359, -0.357) --
 ( 0.394, -0.348) --
 ( 0.427, -0.338) --
 ( 0.460, -0.327) --
 ( 0.491, -0.316) --
 ( 0.522, -0.303) --
 ( 0.551, -0.290) --
 ( 0.578, -0.276) --
 ( 0.605, -0.262) --
 ( 0.630, -0.247) --
 ( 0.653, -0.231) --
 ( 0.675, -0.215) --
 ( 0.695, -0.198) --
 ( 0.714, -0.181) --
 ( 0.730, -0.163) --
 ( 0.746, -0.145) --
 ( 0.759, -0.127) --
 ( 0.770, -0.108) --
 ( 0.780, -0.089) --
 ( 0.788, -0.070) --
 ( 0.794, -0.051) --
 ( 0.798, -0.031) --
 ( 0.800, -0.011) --
 ( 0.800,  0.008) --
 ( 0.798,  0.028) --
 ( 0.794,  0.047) --
 ( 0.789,  0.066) --
 ( 0.781,  0.086) --
 ( 0.772,  0.105) --
 ( 0.761,  0.123) --
 ( 0.748,  0.142) --
 ( 0.733,  0.160);
% labels
 \node [above] at (-2.5, 3.3) {$\Sigma_1$};
  \node [below] at (-3.536, -3.536) {$\pbcmea$};
  \node [right] at (5, 0) {$\pbcmeb$};
  \node [left] at (0, 5) {$\pbcmec$};
  \node [below] at (3.55, 0.8) {$A$};
  \node [below] at (-3.55, 0.8) {$B$};
  \node [left] at (1.35, 3) {$\pbcme$};
  \node [below] at (0.6, -0.3) {$\theta_2$};
  \node [right] at (3, -2.5) {$\mubcm = \cos \theta_2$};
\end{tikzpicture}
\caption{Step 2 for a 2-step 2-body reaction.  In this figure
$\pbcmea = |\pbcme| \mubcm$.}
\label{Fig:2-step-2-body-step2-rel}
\end{center} 

\end{figure}

It is clear from the mappings Eqs.~(\ref{R-3-inv-step2}) and~(\ref{R-3-inv-rotate})
that for fixed values of the kinetic energy~$\Tlabin$ of the incident
particle and for fixed direction cosines $\muacm$ and~$\mubcm$,
the kinetic energy~$\Tblabe$ of the final emitted particle in the initial
laboratory frame is independent of the value of~$\pbcmeb$.
That is, $\Tblabe$ is constant for $\pbcmea$ and~$\pbcmeb$ on
the segment from $A$ to~$B$ in the $(\pbcmea, \pbcmeb)$-plane
in Figure~\ref{Fig:2-step-2-body-step2-rel}.
Note also that for fixed $\Tlabin$ and~$\muacm$, the relations
Eqs.~(\ref{R-3-inv-step2}) and~(\ref{2-step-def-xi-rel}) imply that
$\Tblabe$ is an increasing function of~$\mubcm$.  It follows
that if $\Tblabe$ lies in an energy bin~$\calE_h'$ for given
$\Tlabin$ and~$\muacm$, then it does so for $\mubcm$ on
an interval $[a_h, b_h]$ with
$$
  -1 \le a_h \le \mubcm \le b_h \le 1.
$$
This situation is of interest only if $a_h < b_h$.  In conclusion,
it has been shown that the integral over~$\Sigma_{1,h}$ in
Eq.~(\ref{2-step-muEint-2step}) may be written as
\begin{equation*}
   \int_{\Sigma_{1,h}} d\sigma_1 \, P_\ell( \mulab ) =
   \frac{1}{2 \pi }
   \int_{ a_h}^{b_h} d\mubcm \int_{- \pi/2}
                           ^{\pi/2} dw \,
          P_\ell( \mulab ) .
% \label{2-step-int-sigma-rel}
\end{equation*}




} % end of this appendix
