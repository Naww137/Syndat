\chapter{Proof of Assertion~\ref{Sec:assertion8}}
\label{Sec:Appendix-B}
It is proved in this appendix that for a Newtonian boost,
for the function $G_0$ defined in Eq.~(\ref{def-G0}),
it is true that arcs $\Elab' = \Ebin$ and
$\Ecm' = \text{const}$ in Figure~\ref{Fig:boost-regions} intersect if and only if
$G_0( \Ebin, \Ecm', E ) \ge 0$.

The clearest way to prove this assertion is to
argue four cases directly:
 \begin{align}
  G_0( \Ebin',  \Ecm', E ) \ge 0 & \quad \text{and}
    \quad \Etrans' + \Ecm' \ge \Ebin',
    \label{first-G0}\\
  G_0( \Ebin',  \Ecm', E ) \ge 0 & \quad \text{and}
    \quad \Etrans' + \Ecm' < \Ebin', \\
  G_0( \Ebin',  \Ecm', E ) < 0 & \quad \text{and}
    \quad \Etrans' + \Ecm' \ge \Ebin', \\
   G_0( \Ebin',  \Ecm', E ) < 0 & \quad \text{and}
    \quad \Etrans' + \Ecm' < \Ebin'.
 \end{align}
 In these inequalities $\Etrans'$ is as defined in Eq.~(\ref{E_trans}).

A geometric condition for the intersection of
the two arcs is presented first.  It is then shown that this geometric condition is equivalent
to the non-negativity of~$G_0$.  

\section{An equivalent geometric condition}
The geometric condition is that for given values of $\Ebin'$, $\Ecm'$ and~$E$,
the arcs $\Elab' = \Ebin'$ and
$\Ecm' = \text{const}$ in Figure~\ref{Fig:boost-regions} intersect if and only if
\begin{equation}
  \left( \sqrt{\Etrans'} - \sqrt{\Ecm'} \right)^2 \le \Ebin' \le
    \left( \sqrt{\Etrans'} + \sqrt{\Ecm'} \right)^2.
  \label{Elab-range}
\end{equation}

For the purposes of this argument, it is
convenient to use units of mass such that the mass of the outgoing particle
is $\myo = 2$.  Thus, its speed in the center-of-mass frame is
$\Vcm' = \sqrt{\Ecm'}$.  The arcs in
Figure~\ref{Fig:boost-regions} may be viewed either as curves of constant energy or constant speed.  For given
energy $E$ of the incident particle, the speed $\vtrans = \sqrt{\Etrans'}$ of
the center of mass is determined.  In terms of the speeds with
$\vbin' = \sqrt{\Ebin'}$, the condition
Eq.~(\ref{Elab-range}) is equivalent to
\begin{equation}
  \vtrans^2 + {\vcm'}^2 - 2 \vtrans \vcm' \le {\vbin'}^2 \le
  \vtrans^2 + {\vcm'}^2 + 2 \vtrans \vcm'.
 \label{Elab-range-speed}
\end{equation}

For emission in the forward direction, the speed of the outgoing particle
in the laboratory frame is
$$
  \vlab' = \vtrans + \vcm',
$$
so that its energy in the laboratory frame is
$$
  {\vlab'}^2 = \vtrans^2 + {\vcm'}^2 + 2 \vtrans \vcm'.
$$
In backward emission, the speed of the outgoing particle
in the laboratory frame is
$$
  \vlab' = \left| \vtrans - \vcm' \right|,
$$
and its energy in the laboratory frame is
$$
  {\vlab'}^2 = \vtrans^2 + {\vcm'}^2 - 2 \vtrans \vcm'.
$$
It follows that if condition Eq.~(\ref{Elab-range-speed}) is true,
then there exists a center-of-mass direction cosine $\mucm$ with
$-1 \le \mucm \le 1$ for which the emitted particle has the
desired laboratory energy
$$
 {\vbin'}^2 =
  \vtrans^2 + {\vcm'}^2 + 2 \mucm \vtrans \vcm'.
$$
The two arcs $\Elab' = \Ebin'$ and $\Ecm' = \text{const}$ intersect at this
value of~$\mucm$.  It is seen that if the geometric condition Eq.~(\ref{Elab-range})
is satisfied, then the arcs $\Elab' = \Ebin'$ and $\Ecm' = \text{const}$ 
do intersect.

It is now shown that if Eq.~(\ref{Elab-range-speed}) is false, then then arcs 
$\Elab' = \Ebin'$ and $\Ecm' = \text{const}$ do not intersect.
One way for Eq.~(\ref{Elab-range-speed}) to be false is that
\begin{equation}
  \vbin' > \vtrans + \vcm'.
  \label{big-Elab}
\end{equation}
In this case, forward emission has insufficient energy in the laboratory frame,
and  the arc $\Ecm' = \text{const}$ in Figure~\ref{Fig:boost-regions}
is entirely enclosed within the arc $\Elab' = \Ebin'$.

If
\begin{equation}
   \vbin' < \left| \vtrans - \vcm' \right|,
 \label{small-Elab}
\end{equation}
there are two more ways for Eq.~(\ref{Elab-range-speed}) to be false,
depending on whether
\begin{equation}
  \vcm' < \vtrans
 \label{small-Vcm}
\end{equation}
or
\begin{equation}
  \vcm' > \vtrans.
 \label{small-Vtrans}
\end{equation}

Under the conditions in Eq.~(\ref{small-Vcm}),
backward emission in the center-of-mass frame boosts to forward
emission in the laboratory frame.  The condition Eq.~(\ref{small-Elab})
implies that 
$$
   \vbin' <  \vtrans - \vcm',
$$
so that the arc $\Elab' = \Ebin'$ is completely to the left of
the arc $\Ecm' = \text{const}$ in Figure~\ref{Fig:boost-regions}.  (In fact, one pair of such
arcs is shown in Figure~\ref{Fig:boost-regions}.)

The final way for Eq.~(\ref{Elab-range-speed}) to be false is that
conditions Eqs.~(\ref{small-Elab}) and~(\ref{small-Vtrans}) be
valid.  In this case, backward emission in the center-of-mass
frame produces backward emission in the laboratory frame with
$$
  \vbin' < \vcm' - \vtrans.
$$
In this case, the arc $\Elab' = \Ebin'$ is completely contained
within the arc $\Ecm' = \text{const}$ in Figure~\ref{Fig:boost-regions}.  This finishes the proof
of the assertion that the arcs $\Elab' = \Ebin'$ and $\Ecm' = \text{const}$ 
in Figure~\ref{Fig:boost-regions} intersect if ans only if Eq.~(\ref{Elab-range}) is true.

\section{Proof of the assertion}

Consider the case Eq.~(\ref{first-G0}) above.  That is, suppose that
\begin{equation}
   G_0( \Ebin',  \Ecm', E ) \ge 0
 \label{B-G0-positive}
\end{equation}
and
\begin{equation}
 \Etrans' + \Ecm' \ge \Ebin'.
 \label{B-Ebin-small}
\end{equation}
It is now shown that these two inequalities lead to the geometric
condition Eq.~(\ref{Elab-range}) for intersection of the two arcs.
The inequality Eq.~(\ref{B-G0-positive}) may be rewritten in the form
$$
  4 \Ecm' \Etrans' - ( \Etrans' + \Ecm' - \Ebin' )^2 \ge 0.
$$
Because of the fact that $\Etrans' + \Ecm' - \Ebin' \ge 0$, 
it is possible to take positive square roots
to obtain the relation
$$
  2 \sqrt{ \Ecm' \Etrans' } \ge \Etrans' + \Ecm' - \Ebin',
$$
which may be rearranged as
$$
  \Ebin' \ge \left( \sqrt{\Etrans'} - \sqrt{\Ecm'} \right)^2.
$$
The first of the inequalities Eq.~(\ref{Elab-range}) is now verified.

The second inequality Eq.~(\ref{Elab-range}) follows trivially
from the assumption Eq.~(\ref{B-Ebin-small}),
$$
  \Ebin' \le \Etrans' + \Ecm' \le
  \Etrans' + \Ecm' +  2 \sqrt{ \Ecm' \Etrans' }.
$$

The other three cases may be analyzed in a similar fashion.
