\chapter{Formulas for double-differential energy-angle data}
\label{Sec:double-diff-formula}
This section explains the coding used to treat two representations
by formula for double-differential energy-angle data in the \xendl\ library,
the Kalbach-Mann formula and the phase-space model.  The
Kalbach-Mann model is described first, because it is used so often in \xendl.

\section{The Kalbach-Mann model for double-differential data}
In the Kalbach-Mann representation \cite{Kalbach} the
double differential probability density is of the form
\begin{equation}
  \pi(\Ecm', \mucm \mid E) =
  \pi_E( \Ecm' \mid E) \pi_\mu( \mucm \mid \Ecm', E ),
  \label{Kalbach-prob}
\end{equation}
where $E$ is the energy of the incident particle in laboratory
coordinates and $\Ecm'$ and $\mucm$ are the energy and cosine of
the outgoing particle in center-of-mass coordinates.
The values of the probability density $\pi_E( \Ecm' \mid E)$ for outgoing 
energy $\Ecm'$
are given as a table with normalization
$$
  \int_0^\infty d\Ecm' \, \pi_E( \Ecm' \mid E) = 1.
$$
In Eq.~(\ref{Kalbach-prob}) the function
$\pi_\mu( \mucm \mid \Ecm', E )$ is an exponential in
$\mucm$ depending on parameters~$a$ and~$r$~\cite{Kalbach},
\begin{equation}
  \pi_\mu( \mucm \mid \Ecm', E ) =
    \frac{1}{C}
    [ \cosh(a\mucm) + r \sinh(a\mucm)].
  \label{Kalbach-eta}
\end{equation}
The value of $r$ in Eq.~(\ref{Kalbach-eta}) depends on the incident and outgoing energies 
$E$ and $\Ecm'$ and is given in a data table.  The formula Eq.~(\ref{Kalbach-eta})
represents a pre-equilibrium model, with $r=0$ representing complete
equilibrium and $r = 1$ no equilibrium at all.  It is therefore always true
that
$$
  0 \le r \le 1.
$$

The value of $C$ in Eq.~(\ref{Kalbach-eta}) is chosen to ensure the normalization
$$
  \int_{-1}^1 d\mucm \, \pi_\mu( \mucm \mid \Ecm', E ) = 1.
$$
That is, take
$$
  C = \frac{ 2 \sinh{a}}{a}.
$$

\subsection{The Kalbach-Mann $a$ parameter}
The values of the parameter~$a$ in Eq.~(\ref{Kalbach-eta}) may be given
as a table depending on the incident
energy $E$ and on $\Ecm'$, the center-of-mass kinetic energy of
the outgoing particle.  It is more common, however, to use the formula
for $a$ as a function of $E$ as found in the references \cite{Kalbach} and~\cite{ENDFB}.
The details are repeated here for the sake of completeness.

{% begin special notation
\newcommand{\Ealab}{E_{a,\text{lab}}}
\newcommand{\Eacm}{E_{a,\text{cm}}}
\newcommand{\EAcm}{E_{A,\text{cm}}}
\newcommand{\EaAcm}{E_{aA,\text{cm}}}
\newcommand{\Ebcm}{E_{b,\text{cm}}}
\newcommand{\EBcm}{E_{B,\text{cm}}}
\newcommand{\EbBcm}{E_{bB,\text{cm}}}
Some special notation is used in this subsection.  The reaction is of the form
\begin{equation}
  A + a \to C \to B + b,
 \label{Kalbach-reaction}
\end{equation}
where
\begin{align*}
  A:\quad &\text{the target with mass $\mtarg$, assumed to be at rest in the laboratory frame,} \\
  a:\quad &\text{the incident particle with mass $\myi$,} \\
  C:\quad &\text{the compound nucleus,} \\
  B:\quad &\text{the residual nucleus with mass $\mres$,} \\
  b:\quad &\text{the emitted particle with mass $\myo$.}
\end{align*}


Several energies are needed, all measured in MeV,
\begin{align*}
  \Ealab:\quad &\text{energy of the incident particle in the laboratory frame,} \\
  \Eacm:\quad &\text{energy of the incident particle in the center-of-mass frame,} \\
  \EAcm:\quad &\text{energy of the target in the center-of-mass frame,} \\
  \EaAcm:\quad &\Eacm + \EAcm = \mtarg \Ealab/( \mtarg + \myi), \\
  \Ebcm:\quad &\text{energy of the outgoing particle in the center-of-mass frame,} \\
  \EbBcm:\quad & (\mres + \myo) \Ebcm / \myo.
\end{align*}
Note that the quantity $\EbBcm$ is the total kinetic energy of $B$ and $b$ if
the breakup of $C$ is a discrete 2-body reaction with the excitation level
of $B$ unspecified.

For a reaction with several outgoing particles, $b$ in Eq.~(\ref{Kalbach-reaction})
is the particle corresponding to the current data, and $B$ is the residual
following the emission of $b$ from the compound nucleus~$C$.
Thus, for the
$$
   {}^{78}\text{Kr} (n, np) {}^{77}\text{Br}
$$
reaction, one uses
$$
   B =  {}^{78}\text{Kr}
$$
in the computation of $a(E, \Ebcm)$ with Kalbach-Mann data for the
outgoing neutron, while
$$
   B =  {}^{78}\text{Br}
$$
with outgoing proton data.  Analogously, use
$$
     B =  {}^{78}\text{Kr}
$$
in the computation of $a(E, \Ebcm)$ with Kalbach-Mann neutron data for the
$$
   {}^{78}\text{Kr} (n, 2n) {}^{77}\text{Kr}
$$
reaction.

For massive incident particles, the value of $a(E, \Ebcm)$ is given 
by the expression
\begin{equation}
  a(E, \Ebcm) = C_1 X_1 + C_2 X_1^3 + C_3 M_a m_b X_3^4
 \label{Kalbach-a}
\end{equation}
with terms explained below.

The coefficients in Eq.~(\ref{Kalbach-a}) are
$$
  C_1 = 0.04\,\text{MeV}^{-1}, \quad
  C_2 = 1.8 \times 10^{-6}\,\text{MeV}^{-3}, \quad
  C_3 = 6.7 \times 10^{-7} \,\text{MeV}^{-4}.
$$

The values of $X_1$ and $X_3$ in Eq.~(\ref{Kalbach-a}) depend on
the energies $S_a$ and $S_b$ of the capture and breakup  reactions
in Eq.~(\ref{Kalbach-reaction}).  For the target define
\begin{align*}
  Z_A: \quad & \text{number of protons in the target nucleus,} \\
  N_A: \quad & \text{number of neutrons in the target nucleus,} \\
  A_A: \quad & Z_A + N_A.
\end{align*}
Corresponding $Z_C$, $N_C$, and $A_C$ are defined  
for the compound nucleus~$C$ and
$Z_B$, $N_B$, and $A_B$ for the residual nucleus~$B$.
For the capture reaction,
$S_a$ is taken as
\begin{multline}
  S_a = 15.68(A_C - A_A) -
    28.07\left(
      \frac{(N_C - Z_C)^2}{A_C} - \frac{(N_A - Z_A)^2}{A_A }
    \right) - {} \\
    18.56( A_C^{2/3} - A_A^{2/3}) +
      33.22\left(
        \frac{(N_C - Z_C)^2}{A_C^{4/3}} - \frac{(N_A - Z_A)^2}{A_A^{4/3} }
    \right) - {} \\
    0.717\left(
      \frac{Z_C^2}{A_C^{1/3}} - \frac{Z_A^2}{A_A^{1/3}}
    \right) + 1.211\left(
      \frac{Z_C^2}{A_C} - \frac{Z_A^2}{A_A}
    \right) - I_a.
 \label{def-Sa}
\end{multline}
Here, $I_a$ is the breakup energy for the incident particle
as given in Table~10.1.  The energy $S_b$ corresponding to the second
reaction in Eq.~(\ref{Kalbach-reaction}) is obtained from Eq.~(\ref{def-Sa})
with $Z_A$, $N_A$, $A_A$, and $I_a$ replaced, respectively by
$Z_B$, $N_B$, $A_B$, and $I_b$.

\begin{table}
\caption{Breakup energies for incident and outgoing particles in MeV}
$$
 \vbox{ \offinterlineskip \tabskip = 0.0cm
  \halign{
   \mystrut#\hfil  \tabskip = 0.5cm &
   \hfil#\hfil&
   \hfil#\hfil&  \tabskip = 0cm
   \hfil\vrule#\cr
   \noalign{\hrule}
   height 12pt depth 6pt& particle&  $I_a$ or $I_b$& \cr
   \noalign{\hrule}
   height 12pt& $n$& 0& \cr
   &$p$& 0& \cr
   &$d$& 2.22& \cr
   &$t$& 8.48& \cr
   &${}^3\text{He}$& 7.72& \cr
   depth 6pt &$\alpha$& 28.3& \cr
   \noalign{\hrule}
  }
 }
$$
\end{table}

The quantities $X_1$ and $X_3$ in Eq.~(\ref{Kalbach-a}) are
obtained by setting
\begin{alignat*}{2}
   \textbf{E}_a &= \EaAcm + S_a, \qquad
    & \textbf{E}_b &= \EbBcm + S_b, \\
   E_{t1} &= 130 \text{ MeV}, \qquad
    & E_{t3} &= 41 \text{ MeV}, \\
   R_1 &= \min( \textbf{E}_a, E_{t1}), \qquad
    & R_3 &= \min( \textbf{E}_a, E_{t3}), \\
   X_1 &= R_1 \textbf{E}_b / \textbf{E}_a, \qquad
    & X_3 &= R_3 \textbf{E}_b / \textbf{E}_a.
\end{alignat*}

Finally the values $M_a$ for the incident particle and $m_b$ for the
outgoing particle in the last term of Eq.~(\ref{Kalbach-a}) are
given in Table~10.2.  Note that $M_a$ is not defined for incident
tritons or for incident helium-3 nuclei, so that the Kalbach-Mann
model is not applicable when the incident energy of such 
particles is so large that $\textbf{E}_a > E_{t3}$.

\begin{table}
\caption{Values of $M_a$ and $m_b$ in Eq.~(\ref{Kalbach-a})}
$$
 \vbox{ \offinterlineskip \tabskip = 0.0cm
  \halign{
   \mystrut#\hfil  \tabskip = 0.5cm &
   \hfil#\hfil&
   \hfil#\hfil&
   \hfil#\hfil&  \tabskip = 0cm
   \hfil\vrule#\cr
   \noalign{\hrule}
   height 12pt depth 6pt& particle&  $M_a$& $m_b$& \cr
   \noalign{\hrule}
   height 12pt& $n$& 1& 1/2 &\cr
   &$p$& 1& 1&\cr
   &$d$& 1& 1& \cr
   &$t$& ---& 1& \cr
   &${}^3\text{He}$& ---& 1&\cr
   depth 6pt &$\alpha$& 0& 2& \cr
   \noalign{\hrule}
  }
 }
$$
\end{table}

\subsection{Photo-nuclear reactions}
When Kalbach-Mann data are given for photo-nuclear reactions, 
the parameter $a(E, \Ebcm)$ in Eq.~(\ref{Kalbach-a}) and the
angular probability density $\pi_\mu( \mucm \mid \Ecm', E )$
in Eq.~(\ref{Kalbach-eta}) are modified as in the paper~\cite{photo-nuc}.

One begins by computing $a_n(E, \Ebcm)$ in Eq.~(\ref{Kalbach-a}) using a neutron as
incident particle.  Then, for the incident photon one takes
\begin{equation}
  a(E, \Ebcm) = a_n(E, \Ebcm)
    \sqrt{ \frac{E}{2 m_n} } \,
    \min\left(
      4, \max\left(
        1, \frac{9.3}{\sqrt{\Ebcm}}
      \right)
    \right).
 \label{Kalbach-photo}
\end{equation}
Here, $m_n$ is the mass of the neutron in MeV.

For incident photons the angular probability density takes the form
$$
  \pi_\mu( \mucm \mid \Ecm', E ) =
  \frac{1}{2} \left[
    (1 - r) + \left(
      \frac{ar}{\sinh(a)}
    \right) \expon{a \mucm}
  \right].
$$


\subsection{Interpolation of Kalbach-Mann data}\label{Sec:Kalbach-Mann-interp}

In the \xendl\ library, the Kalbach-Mann data are given
as a table of the probability density $\pi_E( \Ecm' \mid E)$ of outgoing
energy~$\Ecm'$ for an incident particle with energy~$E$,
along with a table of values of the parameter~$r$ in Eq.~(\ref{Kalbach-eta})
as a function of $E$ and~$\Ecm'$.  It is also permitted to include a
table of values $a(E,  \Ebcm)$ to be used in place of the expression
in Eq.~(\ref{Kalbach-a}).
}% end special notation

Because $\pi_E( \Ecm' \mid E)$ is a probability
density, it is to be interpolated with respect to $E$ by one of the methods
of Section~\ref{Sec:2d-interp}.  
The interpolated values of $r$, however, must 
maintain the physical constraints that $0 \le r \le 1$,
so the Kalbach-Mann $r$ parameter is interpolated by the unscaled methods of
Section~\ref{Sec:Kalbach-r-interp}.  If the values of $a$ are also
given as a table, they are also interpolated as in Section~\ref{Sec:Kalbach-r-interp}.

For unit-base interpolation the method is as follows.
The energy probability density
$\pi_E( \Ecm' \mid E)$ is first mapped to unit base as defined in equations
Eqs.~(\ref{unit-base-map}) and~(\ref{unitbaseMap}), so that
\begin{equation}
  \widehat \pi_E( \widehat \Ecm' \mid E) =
    ( E'_{\text{cm}, \text{max}} - E'_{\text{cm}, \text{min}} ) 
     \pi_E( \Ecm' \mid E)
 \label{Kalb-pi-unit-base}
\end{equation}
for $0 \le \widehat \Ecm' \le 1$.
The scale factor in Eq.~(\ref{Kalb-pi-unit-base}) is chosen so as to 
normalize the function $\widehat \pi_E( \widehat \Ecm' \mid E) $,
$$
  \int_0^1 d\widehat \Ecm' \, \pi_E( \widehat \Ecm' \mid E) = 1.
$$
The values of $\widehat \pi_E( \widehat \Ecm' \mid E)$ are
interpolated linearly with respect to $E$.

For the values of the parameter $r$, the energy
of the outgoing particle to is mapped $0 \le \widehat \Ecm' \le 1$
using Eq.~(\ref{Kalb-unit-base}) in the form of
$$
  \widehat \Ecm' = \frac{
  \Ecm' - E'_{\text{cm}, \text{min}} }
     { E'_{\text{cm}, \text{max}} - E'_{\text{cm}, \text{min}} }.
$$
Because of the
restriction that $0 \le r \le 1$, the parameter $r$ is mapped according to
\begin{equation}
  \widetilde r( \widehat \Ecm', E ) = r( \Ecm', E ).
  \label{map-Kalb-r}
\end{equation}

With these transformations, the number-preserving integral Eq.~(\ref{Inum}) 
takes the form
\begin{multline}
   \Inum_{gh,\ell} =
         \int_{\calE_g} dE \, \sigma ( E ) M(E) w(E) \widetilde \phi_\ell(E) 
       \, \int_{\widehat \Ecm' } d\widehat \Ecm' \,
           \widehat \pi_E( \widehat \Ecm' \mid E) \\
        \int_{\mucm} d\mucm \,  P_\ell( \mulab ) \pi_\mu( \mucm \mid \Ecm', E ),
  \label{Inum-Kalb-corr}
\end{multline}
and the energy-preserving integral Eq.~(\ref{Ien}) becomes
\begin{multline}
  \Ien_{gh,\ell} =
        \int_{\calE_g} dE \, \sigma ( E ) M(E) w(E) \widetilde \phi_\ell(E) 
       \, \int_{\widehat \Ecm' } d\widehat \Ecm' \,
           \widehat \pi_E( \widehat \Ecm' \mid E) \\
        \int_{\mucm} d\mucm \,  P_\ell( \mulab ) \pi_\mu( \mucm \mid \Ecm', E ) \Elab' .
  \label{Ien-Kalb-corr}
\end{multline}
The subscripts on $\mu$ serve to emphasize the facts
that the argument $\mulab$ of the Legendre polynomial
$P_\ell( \mulab )$ in Eqs.~(\ref{Inum-Kalb-corr}) and (\ref{Ien-Kalb-corr}) is
the direction cosine of the outgoing particle in laboratory coordinates,
while the integration variable $\mucm$ is the direction cosine in
center-of-mass coordinates.
Specifically, $\Elab' $ depends on $E$ and $\mucm$ according to equation Eq.~(\ref{E_lab}),
and $\mulab$ is given by Eq.~(\ref{get_mu}).

Because the energy probability density $\pi_E( \Ecm' \mid E)$ data are given
in the center-of-mass frame, the identification of the region of integration
over $\widehat \Ecm'$ and~$\mucm$ in Eqs.~(\ref{Inum-Kalb-corr})
and~(\ref{Ien-Kalb-corr}) involves the geometric considerations
presented for tabular center-of-mass data in Section~\ref{Sec:boost-geometry}.
For a given incident energy $E$ in bin $\calE_g$,
the regions of integration over
$\widehat \Ecm'$ and $\mucm$ in Eqs.~(\ref{Inum-Kalb-corr}) 
and~(\ref{Ien-Kalb-corr}) depend on how the domains for data interpolation
$ \widehat E'_{\text{cm},j-1} \le \widehat \Ecm' \le \widehat E'_{\text{cm},j}$
 intersect
the  $\calE_h'$ outgoing laboratory energy bin.   
The situation for a fixed incident
energy~$E$ is illustrated in Figure~\ref{Fig:boost-regions}.  
The half annulus
$$
  E'_{\text{cm}, j-1} \le \Ecm' \le  E'_{\text{cm}, j}
$$
is derived from the Kalbach-Mann data.  The region of integration 
over $\mucm$ and $\Ecm'$ for fixed incident energy $E$ is the intersection of
these two half annuli, and it is shaded dark gray in Figure~\ref{Fig:boost-regions}.

\subsection{The input file for the Kalbach-Mann model}
The data identifier in Section~\ref{data-model} for the
Kalbach-Mann model is\\
  \Input{Process: Kalbach spectrum}{}\\
and the data are always in the center-of-mass frame\\
  \Input{Product Frame: CenterOfMass}{}\\
Currently, only a Newtonian boost to the laboratory frame is
implemented.

The masses of the particles $a$, $A$, $C$, $b$, and~$B$ in
the reaction Eq.~(\ref{Kalbach-reaction}) are input in Section~\ref{model-info}
of the input file\\
  \Input{Projectile's mass:}{$\myi$} \\
 \Input{Target's mass:}{$\mtarg$} \\
 \Input{Compound's mass:}{$m_C$} \\
 \Input{Product's mass:}{$\myo$} \\
 \Input{Residual's mass:}{$\mres$}\\
The units used for these masses are arbitrary, but they must be the
same for all particles.

The number of protons $Z_A$ and the atomic number $A_A$ of the
target are needed for the computation of $S_a$ in Eq.~(\ref{def-Sa}).
This information is entered into the input file as
$$
  \textsf{ZA}_A = 1000Z_A + A_A,
$$
from which $A_A$, $Z_A$, and the number of neutrons $N_A = A_A - Z_A$
are easily computed.  Corresponding numbers $\textsf{ZA}_a$ for the projectile and
$\textsf{ZA}_b$ for the emitted particle are also given.  The numbers $\textsf{ZA}_C$ for
the compound nucleus and $\textsf{ZA}_B$ for the residual may be calculated
using
\begin{align*}
  \textsf{ZA}_C &= \textsf{ZA}_A + \textsf{ZA}_a,\\
  \textsf{ZA}_B &= \textsf{ZA}_C - \textsf{ZA}_b.
\end{align*}
This section of the input file is therefore\\
  \Input{Projectile's ZA:}{$\textsf{ZA}_a$} \\
 \Input{Target's ZA:}{$\textsf{ZA}_A$} \\
 \Input{Product's ZA:}{$\textsf{ZA}_b$}
 
 The remainder of the input file consists of tables of $\pi_E( \Ecm' \mid E)$
 and the parameter~$r$ in Eq.~(\ref{Kalbach-eta}) as functions of $E$ and~$\Ecm'$.
 There may also be a table of values of $a$ to be used in place of the expression
 Eq.~(\ref{Kalbach-a}).  
 
 The format for the probability density
 $\pi_E( \Ecm' \mid E)$ is\\
   \Input{Kalbach probabilities:}{$n = K$}\\
  \Input{Incident energy interpolation:}{probability interpolation flag}\\
  \Input{Outgoing energy interpolation:}{list interpolation flag}\\
followed by $K$ blocks of the form\\
 \Input{Ein: $E$:}{$n = J$}\\
with $J$ pairs of values of $\Ecm'$ and~$\pi_E( \Ecm' \mid E)$.
The flag for interpolation with respect to incident energy $E$ is
one of those for probability densities in Section~\ref{interp-flags-probability}, 
while that for the outgoing energy
is one for simple lists.

The table for the $r$ parameter is of the form\\
   \Input{Kalbach r parameter:}{$n = K$}\\
  \Input{Incident energy interpolation:}{unscaled interpolation flag}\\
  \Input{Outgoing energy interpolation:}{list interpolation flag}\\
followed by $K$ blocks of the form\\
 \Input{Ein: $E$:}{$n = J$}\\
with $J$ pairs of values of $\Ecm'$ and~$r( \Ecm',  E)$.
The flag for interpolation with respect to incident energy $E$ is
one of those for unscaled Kalbach-Mann data in Section~\ref{interp-flags-Kalbach-r}, 
while that for the outgoing energy
is one for simple lists.

The format for the Kalbach-Mann $a$ parameter is the same as that for $r$,
with ``r'' replaced by ``a''.  \textit{The tables for $\pi_E( \Ecm' \mid E)$, $r$,
and~$a$ must be given at the same incident energies, and at each
incident energy~$E$, the ranges of outgoing energies~$E'$ must also agree.}

An example of the content of Section~\ref{model-info} of the input
file for Kalbach-Mann data is as follows.  All energies are in MeV.\\
 \Input{Product Frame: centerOfMass}{}\\
 \Input{\# masses}{}\\
  \Input{Projectile's mass: 1.008665}{}\\
 \Input{Target's mass: 56.935394}{}\\
 \Input{Compound's mass: 57.933276}{}\\
 \Input{Product's mass: 1.008665}{}\\
 \Input{Residual's mass: 56.935394}{}\\
 \Input{\# ZA numbers}{}\\
  \Input{Projectile's ZA: 1}{}\\
 \Input{Target's ZA: 26057}{}\\
 \Input{Product's ZA: 1}{}\\
 \Input{\# Kalbach-Mann probability data}{}\\
 \Input{Kalbach probabilities:  n = 12}{}\\
 \Input{Incident energy interpolation: lin-lin unitbase}{}\\
 \Input{Outgoing energy interpolation: flat}{}\\
 \Input{ Ein: 7.781480e+00:   n = 2}{}\\
 \Input{ \indent 0.000000e+00   1.000000e+06}{}\\
 \Input{ \indent 1.000000e-06   0.000000e+00}{}\\
 \Input{ Ein: 7.800000e+00:   n = 7}{}\\
 \Input{ \indent 0.000000e+00   7.375605e+00}{}\\
 \Input{ \indent 1.473426e-03   1.472617e+01}{}\\
 \Input{ \indent 3.437994e-03   3.676034e+01}{}\\
 \Input{ \indent 7.367130e-03   5.717426e+01}{}\\
   \Input{ \indent 1.473426e-02  8.186549e+00}{}\\
  \Input{ \indent 3.437994e-02  5.948511e+00}{}\\
  \Input{ \indent 7.367130e-02  1.000000e-30}{}\\
 \Input{ $\cdots$}{}\\
 \Input{ Ein: 2.000000e+01:  n = 53}{}\\
  \Input{ \indent 0.000000e+00  2.063824e-03}{}\\
  \Input{ \indent 1.473426e-03  3.887721e-03}{}\\
  \Input{ \indent 3.437994e-03  9.245874e-03}{}\\
  \Input{ \indent 7.367130e-03  1.805939e-02}{}\\
  \Input{ \indent 1.473426e-02  3.554576e-02}{}\\
  \Input{ \indent 3.437994e-02  8.408804e-02}{}\\
  \Input{ \indent 7.367130e-02  1.261293e-01}{}\\
  \Input{ \indent }{ $\cdots$}\\
  \Input{ \indent 1.154184e+01  1.604784e-03}{}\\
  \Input{ \indent 1.203298e+01  1.000000e-30}{}\\
 \Input{\# Kalbach-Mann r data}{}\\
  \Input{Kalbach r parameter:  n = 12}{}\\
 \Input{Incident energy interpolation: lin-lin unscaledunitbase}{}\\
 \Input{Outgoing energy interpolation: flat}{}\\
 \Input{ Ein: 7.781480e+00:   n = 2}{}\\
 \Input{ \indent 0.000000e+00   0.000000e+00}{}\\
 \Input{ \indent 1.000000e-06  0.000000e+00}{}\\
 \Input{ Ein: 7.800000e+00:   n = 7}{}\\
 \Input{ \indent 0.000000e+00   4.272290e-02}{}\\
 \Input{ \indent 1.473426e-03   2.992310e-02}{}\\
 \Input{ \indent 3.437994e-03   1.833870e-02}{}\\
 \Input{ \indent 7.367130e-03   1.427320e-02}{}\\
   \Input{ \indent 1.473426e-02  1.829320e-02}{}\\
  \Input{ \indent 3.437994e-02  1.611740e-02}{}\\
  \Input{ \indent 7.367130e-02  1.590910e-02}{}\\
 \Input{ $\cdots$}{}\\
 \Input{ Ein: 2.000000e+01:  n = 53}{}\\
  \Input{ \indent 0.000000e+00  7.037570e-02}{}\\
  \Input{ \indent 1.473426e-03  4.957320e-02}{}\\
  \Input{ \indent 3.437994e-03  3.056740e-02}{}\\
  \Input{ \indent 7.367130e-03  2.555550e-02}{}\\
  \Input{ \indent 1.473426e-02  1.908500e-02}{}\\
 \Input{ \indent  $\cdots$}{}\\
 \Input{ \indent 1.154184e+01  9.548000e-01}{}\\
  \Input{ \indent 1.203298e+01  9.656400e-01}{}

\section{The $n$-body phase space model}
The $n$-body phase space model gives the probability density for
the energy of an outgoing particle in center-of-mass coordinates.
The formula is derived from the volume in phase space occupied
by the particles, subject to the constraints of conservation
of energy and momentum.  The model uses Newtonian mechanics.

In the \ENDF\ manual~\cite{ENDFB}
there are two scenarios for this model: (1) a discrete 2-body reaction
followed by break-up of the excited residual, and (2) break-up
induced by the collision.  In the first case, the $n$-body phase space
model treats only the particles emitted in the break-up of the excited
residual, not the one from the initial collision.  The total kinetic energy $\Estar$
of the outgoing particles treated by the model therefore depends on
the scenario.  For both scenarios, the number~$n$ of outgoing particles
ought to be greater than~2, because the breakup into 2 particles may
be treated as a discrete 2-body reaction.

In the case of break-up following a discrete 2-body
reaction, the analysis is in the frame in which the residual from
the initial collision is stationary.  The total kinetic energy of the 
outgoing particles involved is then
$$
  \Estar = Q_{\textrm{res}},
$$
where $Q_{\textrm{res}}$ is
the energy of the break-up of the excited residual.  The \gettransfer\ 
code currently does not implement this scenario, because no data
in the \ENDFdata\ library~\cite{ENDFdata} currently use it.
The \ENDFdata\ library does contain one reaction for which the data are marked
as an $n$-body reaction following a knock-on collision.  But since the subsequent
breakup is into only 2 particles, the \gettransfer\ code treats the reaction
as a sequence of two discrete 2-body reactions as discussed in
Section~\ref{Sec:2-step-2-body}.

For the break-up of a compound nucleus following the collision of a projectile
with a stationary target in the laboratory frame, the total kinetic energy $\Estar$
of the outgoing particles in the center-of-mass frame is the sum of two
components, the $Q$ of the reaction plus the energy of the initial collision in the
center-of-mass frame.  For an incident
particle of mass $m_i$ and energy $E$ in the laboratory frame hitting a
stationary target of mass $\mtarg$, this collision energy in the center-of-mass frame is
$$
  \frac{\mtarg E}{\myi + \mtarg}.
$$
Consequently, in this scenario the total center-of-mass kinetic energy for all
outgoing particles is
\begin{equation}
  \Estar = Q + \frac{\mtarg E}{\myi + \mtarg}.
  \label{def-Estar}
\end{equation}

The details of the $n$-body phase space model are as follows.
Consider a particular outgoing particle, and suppose that its mass is~$\myo$.
Then conservation of energy and
momentum implies that the maximum kinetic energy of this particle
in the center-of-mass frame is given by
\begin{equation}
  \Emax = \frac{(M_t - \myo) \Estar}{M_t},
  \label{def-Emax}
\end{equation}
where $M_t$ is the total mass of the outgoing particles covered by
the $n$-body phase space model.

Suppose that $n$ is the number of particles resulting from the break-up
reaction.  For an outgoing particle with mass $\myo$,
let $\Emax$ be as in Eq.~(\ref{def-Emax}).
Then in the $n$-body phase space model, the energy probability density 
that this outgoing particle will have energy $\Ecm'$ with $0 \le \Ecm' \le \Emax$
is given by
\begin{equation}
  \picm(\Ecm' \mid E ) = C_n 
     \sqrt{\Ecm'}\,
     (\Emax - \Ecm')^{(3n - 8)/2}.
  \label{n-body-probability}
\end{equation}
Note that this probability density is isotropic in the center-of-mass frame.
Furthermore, the relation Eq.~(\ref{n-body-probability}) was derived using Newtonian
mechanics.

The normalization constant $C_n$ in Eq.~(\ref{n-body-probability}) is best
represented in terms of the beta function
\begin{equation}
  B(\alpha, \beta) =
   \int_0^1 dt \, t^{\alpha - 1} (1 - t)^{\beta - 1} =
   \frac{\Gamma(\alpha) \Gamma(\beta)}{\Gamma( \alpha + \beta)}.
  \label{beta-func}
\end{equation}
With this notation, it is seen that
\begin{equation}
  \frac{1}{C_n} = B\left(\frac{3}{2}, \frac{3n - 6}{2} \right)
    {\Emax}^{(3n - 5)/2}.
  \label{norm}
\end{equation}

\subsection{Geometry of the $n$-body phase space model}
The construction of Figure~\ref{Fig:boost-regions}
made use of the fact that the tabular data required the
consideration of ranges of energy $\Ecm'$ of the outgoing particle
between the tabulated values,
\begin{equation}
  E'_{\text{cm},j-1} \le \Ecm' \le E'_{\text{cm},j}.
  \label{KalbachEcm}
\end{equation}
Here, the limiting values $E'_{\text{cm},j-1}$ and $E'_{\text{cm},j}$
depend on the energy $E$ of the incident particle according to the
principles of unit-base interpolation in Eq.~(\ref{Kalb-unit-base}).

For the $n$-body phase space model, however, the range of
center-of-mass energies of the outgoing particle is
\begin{equation}
  0 \le \Ecm' \le \Emax,
  \label{nBodyEcm}
\end{equation}
where $\Emax$ is as in Eq.~(\ref{def-Emax}).  
That is, for the $n$-body phase space
the annular ring Eq.~(\ref{KalbachEcm}) in Figure~8.1 is replaced by the
interior of the semicircle Eq.~(\ref{nBodyEcm}).

\subsection{Quadrature for the $n$-body phase space model}
\label{Sec:n-body-quadrature}
It is clear from Eq.~(\ref{n-body-probability})
that in the integrals in Eqs.~(\ref{Inum}) and~(\ref{Ien}) the
integrands with respect to outgoing energy~$\Ecm'$ have a
$\sqrt{\Ecm'}$ singularity.  The \gettransfer\ code has a parameter~$\delta_s$
such that if integration of Eqs.~(\ref{Inum}) or~(\ref{Ien}) with respect
to~$\Ecm'$ is over an interval $a \le \Ecm' \le b$ with $b \le \delta_s  \Emax$,
then first-order Gaussian integration with weight~$\sqrt{\Ecm' }$ is used.
Furthermore, if $a < \delta_s  \Emax < b \le (1 - \delta_s ) \Emax$ for such an integral, then
$\sqrt{\Ecm' }$ weighted Gaussian integration is used on
the subinterval $a \le \Ecm'  \le \delta_s  \Emax$ and standard
Gaussian integration on $\delta_s  \Emax \le \Ecm'  \le b$.
See Section~\ref{Sec:sqrt-quad} for instructions on setting the
value of~$\delta_s$.  See Section~\ref{Sec:QuadratureMethods}
for control over whether or not adaptive Gaussian quadrature is used.

When this model is used with $n = 3$, the singularity 
$\sqrt{\Emax - \Ecm'}$ is handled in a similar fashion.  In this case,
first-order Gaussian quadrature with weight $\sqrt{\Emax - \Ecm'}$
is used for integrals with respect to $\Ecm'$ which overlap with
the interval $(1 - \delta_s ) \Emax < \Ecm' \le \Emax$.

\subsection{Input file for the $n$-body phase space model}
The data identifier in Section~\ref{data-model} for the
$n$-body phase space model is\\
  \Input{Process: phase space spectrum}{}\\
and the data are always in the center-of-mass frame\\
  \Input{Product Frame: CenterOfMass}{}\\
Currently, only a Newtonian boost to the laboratory frame is
implemented.

In the model-dependent Section~\ref{model-info} of the input file,
the computation of $\Estar$ in Eq.~(\ref{def-Estar}) requires the
reaction's $Q$ value, as well as the masses $\myi$ of the projectile
and $\mtarg$ of the target.  The units used for the masses are arbitrary,
but the same units must be used for all particles.  The $Q$ value must be
in the same units as the energy bins.  This information is input using
the commands\\
 \Input{Q value:}{$Q$} \\
  \Input{Projectile's mass:}{$\myi$} \\
 \Input{Target's mass:}{$\mtarg$}
 
For the calculation of $\Emax$ in Eq.~(\ref{def-Emax}), the mass $\myo$
is needed, along with the total mass $M_t$ of the outgoing particles covered
by the $n$-body phase space model.  This information is input using\\
 \Input{Product's mass:}{$\myo$} \\
 \Input{Total mass:}{$M_t$}\\
The units used for these masses must be the
same as is used for the other particles.

Finally, the probability density in Eq.~(\ref{n-body-probability}) requires the number
of particles $n$ in the model, and this is given by\\
 \Input{Number of particles:}{$n$}
 
A sample Section~\ref{model-info} of the input file for the
$n$-body phase space model is\\
 \Input{Product Frame: CenterOfMass}{}\\
 \Input{Q value: -2.225002}{}\\
 \Input{Projectile's mass: 1.008665}{}\\
 \Input{Target's mass: 2.014102}{}\\
 \Input{Product's mass: 1.008665}{}\\
 \Input{Total mass: 3.0246030}{}\\
 \Input{Number of particles: 3}{}


