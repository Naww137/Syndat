\chapter{Joint energy-angle probability density tables}
\label{Sec:joint-table}
It is also possible to give energy-angle probability densities
as tables in \xendl.  Traditionally, the data for these tables 
had to be in the laboratory coordinate system, but \xendl\ also
permits center-of-mass data.  For such data in the laboratory frame,
the \ENDL~\cite{Omega} and \ENDF\ 
\texttt{MF = 6, LAW = 7}~\cite{ENDFB} forms 
differ slightly, and \gettransfer\ supports both formats.

The \ENDF\ manual~\cite{ENDFB} includes the format 
\texttt{MF = 6, LAW = 1}, which
may be used for energy-angle probability density tables in either
the laboratory or center-of-mass frames, but the
\ENDFdata~\cite{ENDFdata} library contains no data of this form.
The \xendl\ tables of energy-angle probability density in the
center-of-mass frame are normalized differently from the
\ENDF\  \texttt{MF = 6, LAW = 1} tables, and \gettransfer\ supports only the
\xendl\ version.

The \ENDF\ \texttt{MF = 6, LAW = 7} format for tables
of values of $\pi(\Elab', \mulab \mid E)$ in the laboratory frame is as arrays
\begin{equation}
  \{ E, \{ \mulab, \{ \Elab', \pi(\Elab', \mulab \mid E) \} \} \}.
 \label{ENDF-E-mu-table}
\end{equation}
The data for the lowest incident energy $E$ are given first,
and data for a given incident energy are ordered by increasing direction cosine $\mulab$.
For fixed $E$ and $\mulab$, the data consist of pairs $\{ \Elab', \pi(\Elab', \mulab \mid E) \}$
for values of the energy $\Elab'$ of the outgoing particle.
The normalization of the data $\pi(\Elab', \mulab \mid E)$ is such that 
for each incident energy $E$
the total probability is
$$
  \int_0^\infty d\Elab' \, \int_{-1}^1 d\mulab \, \pi(\Elab', \mulab \mid E) = 1.
$$

The \ENDL\ energy-angle probability density data tables
are given in the form of the product
\begin{equation}
  \pi(\Elab', \mulab \mid E) =
  \pi_\mu(\mulab \mid E)\pi_{E'}(\Elab' \mid E, \mulab),
   \label{correlated}
\end{equation}
in which $\pi_{E'}(\Elab' \mid E, \mulab)$ is normalized so that
$$
  \int_0^\infty d\Elab' \, \pi_{E'}(\Elab' \mid E, \mulab) = 1
$$
for each of the tabulated values of $E$ and~$\mulab$.

The \xendl\ format for joint energy-angle probability tables in the
center-or-mass frame is of the form
\begin{equation}
  \{ E, \{ \Ecm', \{ \mucm, \pi(\Ecm', \mucm \mid E) \} \} \}.
 \label{ENDF-E-mu-cm-table}
\end{equation}
For each incident energy~$E$, the data in Eq.~(\ref{ENDF-E-mu-cm-table})
are normalized so that
$$
  \int_0^\infty d\Ecm' \, \int_{-1}^1 d\mucm \, \pi(\Ecm', \mucm \mid E) = 1.
$$

For the sake of completeness, a description of the
\ENDF\  \texttt{MF = 6, LAW = 1} format energy-angle probability density
data is given here.  In this case, $\pi(\Ecm', \mucm \mid E)$
is represented as a product
$$
  \pi(\Ecm', \mucm \mid E) = \pi_{E'}(\Ecm' \mid E)
   \pi_\mu(\mucm \mid E, \Ecm', ).
$$
The values of $\pi_{E'}(\Ecm' \mid E)$ are given separately.
The order for the data $\pi_\mu( \mucm \mid E, \Ecm')$ is as
in Eq.~(\ref{ENDF-E-mu-cm-table}), and the normalization is
such that
$$
  \int_{-1}^1 d\mucm \, \pi_\mu(\mucm \mid E, \Ecm') = 1
$$
for each value of $E$ and $\Ecm'$.

\section{Processing of energy-angle probability density tables}
\label{Sec:joint-table-process}
The processing of joint energy-angle probability density tabulated
data by the \gettransfer\ code depends on the format of the data.
The treatment of data given in  the laboratory frame is discussed first.
In \gettransfer, energy-angle probability density
tables in the format of Eq.~(\ref{ENDF-E-mu-table}) are converted to
the format of Eq.~(\ref{correlated}) via the formulas
$$
  \pi_\mu(\mulab \mid E) = \int_0^\infty d\Elab' \, \pi(\Elab', \mulab \mid E)
$$
and
$$
  \pi_E(\Elab' \mid E, \mulab) = \frac
    {\pi(\Elab', \mulab \mid E)} {\pi_\mu(\mulab \mid E)}.
$$

With the correlated energy-angle probability density
Eq.~(\ref{correlated}), the number-preserving integral Eq.~(\ref{Inum}) is
\begin{multline}
   \Inum_{gh,\ell} =
        \int_{\calE_g} dE \, \sigma ( E ) M(E) w(E) \widetilde \phi_\ell(E) 
       \, \int_{\calE_h' } d\Elab' \, \\
       \, \int_{\mulab} d\mulab \,  P_\ell( \mulab ) \pi_\mu(\mulab \mid E)\pi_E(\Elab' \mid E, \mulab),
  \label{Inum-corr}
\end{multline}
and the energy-preserving integral Eq.~(\ref{Ien}) becomes
\begin{multline}
  \Ien_{gh,\ell} =
     \int_{\calE_g} dE \, \sigma ( E ) M(E) w(E) \widetilde \phi_\ell(E) 
     \, \int_{\calE_h' } d\Elab' \,  \Elab' \\
     \, \int_{\mulab} d\mulab  \,  P_\ell ( \mulab ) \pi_\mu(\mulab \mid E)\pi_E(\Elab' \mid E, \mulab).
  \label{Ien-corr}
\end{multline}

The method used by \gettransfer\ to evaluate the integrals Eqs.~(\ref{Inum-corr})
and~(\ref{Ien-corr}) is to first compute the Legendre coefficients
\begin{equation}
  \pi_\ell(\Elab' \mid E) =
  \int_{-1}^1 d\mulab  \,  P_\ell ( \mulab ) \pi_\mu(\mulab \mid E)\pi_E(\Elab' \mid E, \mulab).
  \label{get-I4}
\end{equation}
In the evaluation of this integral, the functions~$\pi_\mu(\mulab \mid E)$
and~$\pi_E(\Elab' \mid E, \mulab)$ are interpolated separately with respect to~$\mulab$
before being multiplied.
The coding for the integration of Eqs.~(\ref{InumI4})
and~(\ref{IenI4}) is then applied to obtain the transfer matrix.

The treatment of center-of-mass energy-angle probability tables
of the form Eq.~(\ref{ENDF-E-mu-cm-table}) is as described in
Section~\ref{Ch:Legendre-cm}.

\section{Input of tables of energy-angle probability densities}
The form of the input files for energy-angle probability tables
depends on whether the data is as in Eqs.~(\ref{ENDF-E-mu-table}),
(\ref{correlated}), or~(\ref{ENDF-E-mu-cm-table}).

\subsection{Input of $\pi(\Elab', \mulab \mid E)$ in the form of Eq.~(\ref{ENDF-E-mu-table})}
For tables of the energy-angle probability density $\pi(\Elab', \mulab \mid E)$
in the format Eq.~(\ref{ENDF-E-mu-table}), the identification line in
Section~\ref{model-info} is\\
      \Input{Process: ENDF Double differential EMuEpP data}{}\\
These data are always in the laboratory frame,\\
  \Input{Product Frame: lab}{}

The first lines in the data for Section~\ref{model-info} give the number~$K$
of incident energies along with the interpolation rules\\
  \Input{EMuEpPData:}{$n = K$}\\
  \Input{Incident energy interpolation:}{probability interpolation flag}\\
  \Input{Outgoing cosine interpolation:}{probability interpolation flag}\\
  \Input{Outgoing energy interpolation:}{list interpolation flag}\\
The flags for interpolation with respect to incident energy~$E$ and
direction cosine~$\mulab$ are those for probability density tables in
Section~\ref{interp-flags-probability}, and that for outgoing energy~$E'$ is
one for simple lists.

For each incident energy~$E$ there is a data section of the form\\
    \Input{ Ein:}{$E$: \quad \texttt{n = $N$}}\\
indicating that data are given for $N$ values of $\mulab$.  The block of
data corresponding to a value of $\mulab$ is of the form\\
    \Input{ mu:}{$\mulab$: \quad \texttt{n = $J$}}\\
followed by $J$ pairs of values of outgoing energy $\Elab'$ and
probability density~$\pi(\Elab', \mulab \mid E)$ .

An example of such data with energy in MeV is\\
  \Input{EMuEpPData: n = 18}{}\\
  \Input{Incident energy interpolation: lin-lin unitbase}{}\\
  \Input{Outgoing cosine interpolation: lin-lin unibase}{}\\
  \Input{Outgoing energy interpolation: lin-lin }{}\\
  \Input{ Ein: 1.748830e+00:  n = 21}{}\\
  \Input{  mu: -1.000000e+00:  n = 15}{}\\
  \Input{ \indent   1.092990e-03  0.000000e+00}{}\\
  \Input{ \indent   1.093000e-03  7.406740e-01}{}\\
  \Input{ \indent   3.278900e-03  1.166140e+00}{}\\
  \Input{ \indent   7.650800e-03  1.466540e+00}{}\\
  \Input{ \indent   1.202300e-02  1.585880e+00}{}\\
  \Input{ \indent   2.076600e-02  1.610940e+00}{}\\
   \Input{ \indent 2.951000e-02  1.546240e+00}{}\\
\Input{ \indent  5.574100e-02  1.071950e+00}{}\\
\Input{ \indent  7.104300e-02  7.097100e-01}{}\\
\Input{ \indent  8.197300e-02  4.021720e-01}{}\\
 \Input{ \indent 9.071600e-02  1.795810e-01}{}\\
\Input{ \indent  9.508800e-02  9.526480e-02}{}\\
\Input{ \indent  9.946000e-02  2.867760e-02}{}\\
 \Input{ \indent 1.016500e-01  4.692750e-03}{}\\
\Input{ \indent  1.016510e-01  0.000000e+00}{}\\
\Input{ $\cdots$}{}\\
\Input{ Ein: 2.000000e+01:  n = 21}{}\\
 \Input{  mu: -1.000000e+00:  n = 76}{}\\
\Input{ \indent  4.606790e-02  0.000000e+00}{}\\
\Input{ \indent  4.606800e-02  3.837140e-02}{}\\
\Input{ \indent  9.213400e-02  4.393050e-02}{}\\
\Input{ \indent  1.842700e-01  4.977660e-02}{}\\
\Input{ \indent  2.764100e-01  4.806820e-02}{}\\
\Input{ \indent  3.685400e-01  4.385540e-02}{}\\
\Input{ \indent  6.449500e-01  2.695920e-02}{}\\
\Input{ \indent  7.370900e-01  2.255450e-02}{}\\
 \Input{ \indent }{ etc.}

\subsection{Input of $\pi(\Elab', \mulab \mid E)$ as a product, Eq.~(\ref{correlated})}
For tables of the energy-angle probability density $\pi(\Elab', \mulab \mid E)$
given as the product in Eq.~(\ref{correlated}), the identification line in
Section~\ref{model-info} is\\
      \Input{Process: Double differential EMuEpP data transfer matrix}{}\\
These data are always in the laboratory frame,\\
  \Input{Product Frame: lab}{}

The model-dependent portion of the input file in Section~\ref{model-info}
contains a section for the angular probability density $\pi_\mu(\mulab \mid E)$
and another for the conditional probability density $\pi_E(\Elab' \mid E, \mulab)$.

The section for angular probability density starts with the lines\\
  \Input{Angular data:}{$n = K$}\\
  \Input{Incident energy interpolation:}{probability interpolation flag}\\
  \Input{Outgoing cosine interpolation:}{list interpolation flag}\\
where $K$ is the number of incident energies~$E$.  The flag for
interpolation with respect to incident energy is one of those for
probability density tables in Section~\ref{interp-flags}, and that for 
the direction cosine $\mulab$ is one of those 
for simple lists.  There follows $K$ blocks of data,
one for each incident energy\\
    \Input{ Ein:}{$E$: \quad \texttt{n = $N$}}\\
indicating that data are given for $N$ pairs of values of $\mulab$
and $\pi_\mu(\mulab \mid E)$.

The section for conditional probability density of outgoing energy
$\pi_E(\Elab' \mid E, \mulab)$ gives the number~$K$
of incident energies along with the interpolation rules\\
  \Input{EMuEpPData:}{$n = K$}\\
  \Input{Incident energy interpolation:}{probability interpolation flag}\\
  \Input{Outgoing cosine interpolation:}{probability interpolation flag}\\
  \Input{Outgoing energy interpolation:}{list interpolation flag}\\
The flags for interpolation with respect to incident energy~$E$ and
direction cosine~$\mulab$ are those for probability density tables in
Section~\ref{interp-flags-probability}, and that for outgoing energy~$E$ is one of those
for simple lists.

For each incident energy~$E$ there is a data section of the form\\
    \Input{ Ein:}{$E$: \quad \texttt{n = $N$}}\\
indicating that data are given for $N$ values of $\mulab$.  The block of
data corresponding to a value of $\mulab$ is of the form\\
    \Input{ mu:}{$\mulab$: \quad \texttt{n = $J$}}\\
followed by $J$ pairs of values of outgoing energy $E$ and
probability density~$\pi_E(\Elab' \mid E, \mulab)$.

An example of this type of data with energy in MeV is given by\\
  \Input{Angular data: n = 13}{}\\
  \Input{Incident energy interpolation: lin-lin unitbase}{}\\
  \Input{Outgoing cosine interpolation: lin-lin}{}\\
  \Input{ Ein:   7.78148000e+00:  n = 5}{}\\
  \Input{ \indent   9.99788143e-01   5.88016882e+01}{}\\
  \Input{ \indent   9.99841107e-01   1.03998708e+03}{}\\
  \Input{ \indent   9.99894071e-01   1.70086214e+03}{}\\
  \Input{ \indent   9.99947036e-01   2.60922780e+03}{}\\
  \Input{ \indent   1.00000000e+00   2.70023193e+04}{}\\
  \Input{ $\cdots$}{}\\
   \Input{ Ein: 2.00000000e+02: n = 5}{}\\
 \Input{ \indent -1.00000000e+00  3.26136085e-01}{}\\
 \Input{ \indent -5.00000000e-01  3.82892835e-01}{}\\
  \Input{ \indent 0.00000000e+00  4.64096868e-01}{}\\
  \Input{ \indent 5.00000000e-01  5.89499334e-01}{}\\
  \Input{ \indent 1.00000000e+00  8.00885838e-01}{}\\
   \Input{EMuEpPData: n = 13}{}\\
  \Input{Incident energy interpolation: lin-lin unitbase}{}\\
  \Input{Outgoing cosine interpolation: lin-lin unitbase}{}\\
  \Input{Outgoing energy interpolation: lin-lin}{}\\
  \Input{ Ein:   7.78148000e+00:  n = 5}{}\\
  \Input{ mu:   9.99788143e-01:  n = 4}{}\\
  \Input{ \indent   2.35390141e-03  1.62759930e+05}{}\\
  \Input{ \indent   2.35697064e-03  1.62877493e+05}{}\\
  \Input{ \indent   2.35697074e-03  1.62877496e+05}{}\\
  \Input{ \indent   2.36004196e-03  1.62892892e+05}{}\\
  \Input{ mu:   9.99841107e-01:  n = 16}{}\\
  \Input{ \indent   2.30884914e-03  9.56657064e+03}{}\\
   \Input{ \indent  2.32094195e-03  9.99310479e+03}{}\\
   \Input{ \indent}{ etc.}\\
 \Input{ Ein:  2.00000000e+02: n = 5}{}\\
\Input{  mu: -1.00000000e+00: n = 501}{}\\
  \Input{ \indent 1.00000000e-18  5.38736174e-10}{}\\
  \Input{ \indent 1.00563208e-17  1.70842412e-09}{}\\
  \Input{ \indent 1.91126417e-17  2.35524717e-09}{}\\
  \Input{ \indent 2.81689625e-17  2.85931206e-09}{}\\
  \Input{ \indent 3.72252834e-17  3.28696542e-09}{}\\
  \Input{ $\cdots$}{}\\
  \Input{ mu: 1.00000000e+00: n = 993}{}\\
  \Input{ \indent 1.00000000e-18  2.19383712e-10}{}\\
  \Input{ \indent 7.55831305e-18  6.03138181e-10}{}\\
  \Input{ \indent }{ $\cdots$}\\
  \Input{ \indent 1.32751551e+01  1.88436981e-03}{}\\
  \Input{ \indent 1.38015128e+01  1.90038969e-03}{}
 
 \subsection{Input of $\pi(\Ecm', \mucm \mid E)$ in the form of Eq.~(\ref{ENDF-E-mu-cm-table})}
For tables of the energy-angle probability density $\pi(\Ecm', \mucm \mid E)$
in the format Eq.~(\ref{ENDF-E-mu-cm-table}), the identification line in
Section~\ref{model-info} is\\
      \Input{Process: pointwise energy-angle data}{}\\
These data are always in the center-of-mass frame,\\
  \Input{Product Frame: centerOfMass}{}

The first lines in the data for Section~\ref{model-info} give the number~$K$
of incident energies along with the interpolation rules.  Currently,
\gettransfer\ handles only linear-linear unit-base interpolation with
respect to incident energy~$E$, linear-linear direct interpolation
with respect to outgoing energy~$\Ecm'$, and linear-linear interpolation
for the direction cosine~$\mucm$.  The input format is therefore of the form\\
  \Input{EEpMuPData:}{$n = K$}\\
  \Input{Incident energy interpolation: lin-lin unitbase}{}\\
  \Input{Outgoing energy interpolation: lin-lin direct}{}\\
  \Input{Outgoing cosine interpolation: lin-lin}{}\\

For each incident energy~$E$ there is a data section of the form\\
    \Input{ Ein:}{$E$: \quad \texttt{n = $N$}}\\
indicating that data are given for $N$ values of $\Ecm'$.  The block of
data corresponding to a value of $\Ecm'$ is of the form\\
    \Input{ Ep:}{$\Ecm'$: \quad \texttt{n = $J$}}\\
followed by $J$ pairs of values of direction cosine $\mucm$ and
probability density~$\pi(\Ecm', \mucm \mid E)$ .

An example of such data with energy in MeV is\\
  \Input{EEpMuPData: n = 15}{}\\
  \Input{Incident energy interpolation: lin-lin unitbase}{}\\
  \Input{Outgoing energy interpolation: lin-lin direct}{}\\
  \Input{Outgoing cosine interpolation: lin-lin}{}\\
 \Input{\# Start sub data}{}\\
  \Input{E:  1.002700000000e+01: n = 44}{}\\
  \Input{\# Start sub data}{}\\
  \Input{ Ep: 0.000000000000e+00 : n = 21}{}\\
   \Input{ \indent -1.000000000000e+00  1.078178702711e-01}{}\\
   \Input{ \indent -9.000000000000e-01  1.071148616947e-01}{}\\
  \Input{  \indent -8.000000000000e-01  1.064935902862e-01}{}\\
   \Input{ \indent -7.000000000000e-01  1.059535819660e-01}{}\\
  \Input{  \indent -6.000000000000e-01  1.054944246648e-01}{}\\
   \Input{ \indent -5.000000000000e-01  1.051157680091e-01}{}\\
   \Input{ \indent }{ $\cdots$}\\
  \Input{\# Start sub data}{}\\
  \Input{ Ep: 6.870524931290e-03 : n = 21}{}\\
  \Input{  \indent -1.000000000000e+00  1.078230802439e-01}{}\\
  \Input{  \indent -9.000000000000e-01  1.071184528032e-01}{}\\
   \Input{ \indent -8.000000000000e-01  1.064957460252e-01}{}\\
   \Input{ \indent -7.000000000000e-01  1.059544836844e-01}{}\\
  \Input{  \indent -6.000000000000e-01  1.054942518410e-01}{}\\
   \Input{ \indent }{ $\cdots$}\\
  \Input{\# Start sub data}{}\\
  \Input{ Ep: 4.122315000000e-01 : n = 21}{}\\
  \Input{  \indent -1.000000000000e+00  0.000000000000e+00}{}\\
  \Input{  \indent -9.000000000000e-01  0.000000000000e+00}{}\\
  \Input{  \indent -8.000000000000e-01  0.000000000000e+00}{}\\
  \Input{ $\cdots$}{}\\
  \Input{\# Start sub data}{}\\
 \Input{E =  1.100000000000e+01: n = 62}{}\\
  \Input{\# Start sub data}{}\\
  \Input{ Ep: 0.000000000000e+00 : n = 21}{}\\
  \Input{ \indent }{ $\cdots$}\


