\chapter{Transfer matrices}
\label{Sec:transfer}
Deterministic particle transport codes solve a
discrete version of the Boltzmann equation, and the
transfer matrix approximates the kernel of the integral
operator in this equation.
If $x$ denotes the position,
$t$ the time, $E'$ the particle energy, $\Omega'$ the
direction of motion, $v$ the magnitude of the velocity (speed), and
$n(x, t, E', \Omega')$ the number density, then the
flux $\phi = vn$ satisfies the Boltzmann 
equation~\cite{Lewis}
\begin{equation}
  \frac{1}{v} \, \partial_t \phi(E', \Omega') +
  \Omega'\cdot \nabla \phi(E', \Omega') +
      \rho\sigma_t \phi(E', \Omega') =
  \frac{\rho}{4\pi}
   \int_{\Omega} d\Omega \, \int_0^\infty dE \, 
      \calK(E', \Omega' \cdot \Omega \mid E) \phi(E, \Omega).
  \label{Boltzmann}
\end{equation}
The direction $\Omega'$ is relative to some given 
``north pole''~$\Omega_0$, and $\rho$ is the density of the material.
The dependence on $x$ and $t$ is suppressed.
The first two terms in Eq.~(\ref{Boltzmann}) give the derivative 
with respect to distance of
the flux in a coordinate system moving with the particles.
The parameter
$\sigma_t$ is the microscopic total cross section, so the term
$\rho\sigma_t \phi(E', \Omega')$ represents the rate of particle loss
per particle path length.

The kernel
$\calK(E', \Omega' \cdot \Omega \mid E)$ in Eq.~(\ref{Boltzmann}) gives the
rate of production of outgoing particles with energy $E'$ and 
direction~$\Omega'$ corresponding to incident particles
at energy $E$ and direction~$\Omega$.
Here, the energies $E$ and $E'$ and the directions $\Omega$
and $\Omega'$ are in the laboratory coordinate system.
From here on,  the notation
$$
  \mu = \Omega' \cdot \Omega
$$
is used.
It is significant that the dependence of
$\calK(E', \mu \mid E)$ on $\mu$
is axisymmetric, because the orientation of the
target nucleus is unknown.  The primes are placed where they are in
Eq.~(\ref{Boltzmann}), because the emphasis in this document is on
approximation of the right-hand side of the equation.  In that setting, it is
natural that $E$ denote the energy of the incident particle and $E'$
the outgoing particle energy.

For a given target,
the nuclear data in \xendl\ is given reaction by reaction,
e.g., elastic scattering, neutron capture, fission, etc.
The transfer matrix approximating $\calK$ is built
up by summing over the reactions~$r$
$$
  \calK = \sum_r \calK_r.
$$
The reaction kernels $\calK_r$ themselves are not given in \xendl,
but their component factors are given instead, namely,
\begin{enumerate}
 \item $\sigma_r(E)$: the cross section for the $r$-th reaction,
 \item  $M_r(E)$: the multiplicity of the outgoing particle,
 \item $w_r(E)$: the model weight for these data,
 \item $\pi_r(E', \mu \mid E)$: the double-differential probability
density of the energy and direction cosine
for one outgoing particle. 
\end{enumerate}
In terms of this notation, $\calK_r$ is the product
\begin{equation}
  \calK_r(E', \mu \mid E) = \sigma_r(E) M_r(E) w_r(E) \pi_r(E', \mu \mid E).
  \label{def_pi}
\end{equation}
The multiplicity $M_r(E)$ may be constant, e.g., 1 for elastic scattering
and 2 for $(n, 2n)$ reactions, but the number of fission neutrons
depends on the incident energy~$E$.  The default is $M_r(E) = 1$.

\paragraph{Model weight}\label{Sec:model-weight}
The model weight is usually $w_r(E) = 1$, and that is the
default.  One exception is that data for a single outgoing neutron in
an $(n, 2n)$ reaction may have $M_r(E) = 2$ and $w_r(E) = 0.5$.
The model weight is also used to handle the use of different interpolation
rules over different ranges of incident energy.  Thus, if the interpolation
for $E_1 < E < E_2$ is different from that for $E_2 < E < E_3$,
the data may be split into two sets, one with
$$
  w_r(E) = \begin{cases}
    1 & \text{for $E_1 \le E < E_2$,}\\
    0 & \text{for $E_2 \le E \le E_3$,}
  \end{cases}
$$
and the other with
$$
  w_r(E) = \begin{cases}
    0 & \text{for $E_1 \le E < E_2$,}\\
    1 & \text{for $E_2 \le E \le E_3$.}
  \end{cases}
$$

The \xendl\ nuclear data consist
of tables of $\sigma_r(E)$ and $\pi_r(E', \mu \mid E)$ and possibly
$M_r(E)$ and~$w_r(E)$.
The data for $\pi_r(E', \mu \mid E)$ take several forms, and
the various data representations are dealt with individually.

The discretization of Eq.~(\ref{Boltzmann}) is based, first,
on the specification of a set of energy groups $\{ \calE_g
\}$ for the incident particles and energy groups $\{ \calE'_h
\}$ for the emitted particles.  The energy groups for
neutrons are typically different from those for gammas,
and yet another set is usually used for charged particles.
The flux $\phi(E, \Omega)$ inside the integral in
Eq.~(\ref{Boltzmann}) is discretized according to the energy
groups of the incident particle, while $\phi(E', \Omega')$
on the left-hand side of Eq.~(\ref{Boltzmann}) is discretized
according the the energy groups of the outgoing particles.
These energy groups are also called energy bins.

According to the normalization for Legendre expansions used in \xendl,
the angular discretization of $\pi_r$ in Eq.~(\ref{def_pi}) is 
given by
\begin{equation}
  \pi_r( E', \mu \mid E) =
   \sum_{\ell = 0}^{\Lmax}
   \left(
     \ell + \frac{1}{2}
   \right)
   \pi_{r\ell}( E' \mid E) P_\ell( \mu)
  \label{def_pi_r}
\end{equation}
with $P_\ell( \mu)$ denoting the $\ell$-th Legendre polynomial
and
\begin{equation}
  \pi_{r\ell}( E' \mid E) =
  \int_{-1}^1 d\mu \, \pi_r( E', \mu \mid E) P_\ell( \mu)
  \label{def_piell}
\end{equation}
for $\ell = 0,$ 1, \ldots,~$\Lmax$.
The user may specify the order~$\Lmax$ with the command
given in Section~\ref{Sec:LegendreOrder}.

The flux $\phi(E, \Omega)$ in Eq.~(\ref{Boltzmann}) is expanded 
into spherical harmonics
\begin{equation}
  \phi(E, \Omega) =
  \sum_{\ell, m}
    C_{\ell, m} \phi_{\ell, m}(E) Y_{\ell, m} (\Omega)
 \label{sphHarmonics}
\end{equation}
with normalization
$$
  C_{\ell, m} = \frac{1}{\int d\Omega \, [Y_{\ell, m} (\Omega)]^2}.
$$

A discrete approximation to Eq.~(\ref{Boltzmann}) may be obtained
by expanding $\phi(E, \Omega)$ in spherical harmonics and integrating
over the outgoing energy group~$\calE'_h$.
This gives an
equation for the vector of values
$$
  \phi_{\ell, m}(E'_h).
$$
Note that $\phi_{\ell, m}(E'_h)$ is a histogram with respect to
the energy $E'$ of the outgoing particle, constant on each energy
group~$\calE'_h$.
Integration of the
right-hand side of Eq.~(\ref{Boltzmann}) over $\calE'_h$ gives
\begin{equation}
  \calI_{h,\ell} = \sum_r
  \int_0^\infty dE \, \phi_{\ell, 0}(E)
  \int_{\calE'_h } dE' \, \int_{-1}^1 d\mu \,
  \calK_r(E', \mu \mid E) P_\ell( \mu ).
  \label{binnedBoltzmann}
\end{equation}
The integral Eq.~(\ref{binnedBoltzmann}) contains only the spherical
harmonics with $m = 0$, because the kernel $\calK_r$ is axisymmetric.

The unknown flux $\phi$ appears in Eq.~(\ref{Boltzmann})
both on the left-hand side of the equation and under
the integral sign.  It is therefore convenient to
start the calculation using an assumed approximate value of
$\phi_{\ell, 0}(E)$ in the integral Eq.~(\ref{binnedBoltzmann}), namely,
\begin{equation}
  \phi_{\ell, 0}(E) \approx
    \widetilde\phi_{\ell}(E).
  \label{approx_phi}
\end{equation}

Upon inserting Eq.~(\ref{approx_phi}) into Eq.~(\ref{binnedBoltzmann})
and taking the incident energy groups $\calE_g$ one at a time,
it is found that Eq.~(\ref{binnedBoltzmann}) may be viewed as the product
of a matrix with a column vector.  Here, the column vector has the
components $\phi_{\ell, 0}(E'_h)$, and the components of the matrix
are given by
$$
  \calJ_{g,h,\ell} = \frac{ \calI_{g,h,\ell} }
       { \int_{\calE_g} dE \, \widetilde \phi_\ell(E) }
$$
with
$$
    \calI_{g,h,\ell} = \sum_r
     \int_{\calE_g} dE \, \widetilde \phi_\ell(E) 
    \int_{\calE'_h } dE' \, \int_\mu d\mu \, 
     \calK_r(E', \mu \mid E) P_\ell( \mu ).
$$
The quantities $\calJ_{g,h,\ell}$ constitute the entries of the \textit{transfer matrix}.

The above discussion gives one way of defining the transfer matrix,
but the \xndfgen\ code has three
different representations, depending
on whether one wants to conserve the number of particles,
the energy, or both.  Traditionally, conservation of particle 
number has been used for neutron transport, conservation of energy
for gammas, and conservation of both energy and number for
charged particles.  These cases are taken up in turn.

\section{Conservation of particle number}
With the approximate flux coefficient $\widetilde\phi_\ell$ in Eq.~(\ref{approx_phi})
and the representation Eq.~(\ref{def_pi}) of the kernel $\calK_r$, the
$\ell$-th Legendre coefficient of the contributions of energy
groups $\calE_g$ and $\calE'_h$ to the integral
in Eq.~(\ref{Boltzmann}) by reaction $r$ is given by
\begin{equation}
  \Inum_{r,g,h,\ell} =
     \int_{\calE_g} dE \, \sigma_r ( E ) M_r(E) w_r(E) \widetilde \phi_\ell(E) 
    \int_{\calE'_h } dE' \, \int_\mu d\mu \, 
     P_\ell( \mu )
     \pi_r(E', \mu \mid E).
  \label{Inum}
\end{equation}
For conservation of particle number the elements of the transfer matrix
are the sums over all reactions,
\begin{equation}
  \calJ_{g,h,\ell} = \frac{ \sum_r \Inum_{r,g,h,\ell} }
       { \int_{\calE_g} dE \, \widetilde \phi_\ell(E) }.
  \label{cons_num}
\end{equation}
The \gettransfer\ code computes the integrals $\Inum_{r,g,h,\ell}$
reaction by reaction, and the operation Eq.~(\ref{cons_num}) is performed
by \xndfgen.

Note that the number-preserving transfer matrices offer a
simple check.  Because the probability density
$\pi_r(E', \mu \mid E)$ has the normalization
$$
  \int_0^\infty dE' \, \int_{-1}^1 d\mu \, 
     \pi_r(E', \mu \mid E) = 1,
$$
it follows from Eq.~(\ref{Inum}) that
\begin{equation}
  \sum_h \Inum_{r,g,h,0} =
    \int_{\calE_g} dE \, \sigma_r ( E ) M_r(E) w_r(E) \widetilde \phi_0(E).
  \label{rowSum}
\end{equation}

\section{Conservation of energy}
When conservation of energy is desired, the integral
Eq.~(\ref{Inum}) is modified by insertion of $E'$ as a weight factor
\begin{equation}
  \Ien_{r,g,h,\ell} =
     \int_{\calE_g} dE \, \sigma_r ( E ) M_r(E) w_r(E) \widetilde \phi_\ell(E) 
    \int_{\calE'_h } dE' \, E' \int_\mu d\mu  \, 
     P_\ell ( \mu )
     \pi_r(E', \mu \mid E).
  \label{Ien}
\end{equation}
With the notation that $\overline {E'_h}$ denotes the midpoint of
energy group $\calE'_h$, the elements of the transfer matrix
for energy conservation are the sums over all reactions,
\begin{equation}
  \widehat \calJ_{g,h,\ell} = \frac{ \sum_r \Ien_{r,g,h,\ell} }
       { \overline {E'_h} \int_{\calE_g} dE \, \widetilde \phi_\ell(E) }.
  \label{cons_en}
\end{equation}
The computation of $\widehat \calJ_{g,h,\ell}$ in Eq.~(\ref{cons_en})
is done by \xndfgen\ using the integrals $\Ien_{r,g,h,\ell} $ calculated
by \gettransfer.

\section{Conservation of both particles and energy}
The \xndfgen\ code also has an option to combine the 
integrals $\Inum_{r,g,h,\ell}$ in Eq.~(\ref{Inum}) and $\Ien_{r,g,h,\ell}$ in Eq.~(\ref{Ien})
so as to construct a transfer matrix which conserves both energy
and particle number.  Energy conservation may be violated
in the lowest and highest outgoing energy groups, however.
The construction is based on the following ideas.

There are two ways to compute the average energy of particles
in the outgoing energy group~$\calE'_h$.  One such average is the midpoint
$\overline {E'_h}$ of this group.  Preferably, this value should be
the same as the average energy
derived from the the sums over the reactions~$r$ of the
integrals Eqs.~(\ref{Ien}) and~(\ref{Inum}),
\begin{equation}
  \langle E' \rangle_{g,h} =
  \frac{ \sum_r \Ien_{r,g,h,0}}{ \sum_r \Inum_{r,g,h,0}}.
  \label{av_E}
\end{equation}
This is accomplished, as much as possible, by properly defining
entries of the transfer matrix corresponding to adjacent outgoing
energy groups.

For each
incident energy group $\calE_g$ one iterates through the
outgoing energy groups~$\calE'_h$.
Note that the description of this process in
\cite{Omega} and \cite{ndfgen} assumes that the energy group
boundaries decrease with increasing index;  the energy
group boundaries are counted in increasing order here and in \xndfgen.

If $\langle E' \rangle_{g,h} < \overline {E'_h}$ and $\calE'_h$
is not the lowest energy group, make a fraction of the
sum
$$
\frac{ \sum_r \Ien_{r,g,h,\ell} }
       { \overline {E'_h} \int_{\calE_g} dE \, \widetilde \phi_\ell(E) }
$$
contribute to the transfer matrix element $\calJ_{g,h,\ell}$, and
make the remainder contribute to~$\calJ_{g,h-1,\ell}$.
Specifically, it is desired to find $j_{g,h}$
and $j_{g,h-1}$ which conserve particle number
$$
  j_{g,h} + j_{g,h-1} = 
  \frac{ \sum_r \Inum_{r,g,h,0} }
       { \int_{\calE_g} dE \, \widetilde \phi_0(E) }
$$
as well as average energy
$$
  \overline {E'_h}\, j_{g,h} + \overline {E'_{h-1}}\, j_{g,h-1} = 
    \sum_r \Ien_{r,g,h,0}.
$$
Therefore, set
$$
  f_{g,h} =
   \frac{ \langle E' \rangle_{g,h} - \overline {E'_{h-1}} }
    { \overline {E'_h} - \overline {E'_{h-1}} }.
$$
For each Legendre coefficient $\ell$ take as contribution to
$\calJ_{g,h,\ell}$ the quantity
$$
  j_{g,h} = \frac{ f_{g,h} \sum_r \Inum_{r,g,h,\ell} }
       { \int_{\calE_g} dE \, \widetilde \phi_\ell(E)  },
$$
and the contribution to $\calJ_{g,h-1,\ell}$ is
$$
  j_{g,h-1} = \frac{ (1 - f_{g,h}) \sum_r \Inum_{r,g,h,\ell} }
       { \int_{\calE_g} dE \, \widetilde \phi_\ell(E) }.
$$

If $\langle E' \rangle_{g,h} < \overline {E'_h}$ and $\calE'_h$
is the lowest energy group, the contribution to $\calJ_{g,h,\ell}$
is simply
$$
  \frac{ \sum_r \Inum_{r,g,h,\ell} }
       { \int_{\calE_g} dE \, \widetilde \phi_\ell(E) }.
$$
This maintains conservation of particle number.

If $\langle E' \rangle_{g,h} > \overline {E'_h}$ and $\calE'_h$
is not the highest energy group, these data are used to calculate
contributions to the components $\calJ_{g,h,\ell}$ and $\calJ_{g,h+1,\ell}$
of the transfer matrix.  Specifically, set
$$
  f_{g,h} =
   \frac{ \overline {E'_{h+1}} - \langle E' \rangle_{g,h} }
    { \overline {E'_{h+1}} - \overline {E'_h }}.
$$
For each Legendre coefficient $\ell$ take as contribution to
$\calJ_{g,h,\ell}$ the quantity
$$
  j_{g,h} = \frac{ f_{g,h} \sum_r \Inum_{r,g,h,\ell} }
       { \int_{\calE_g} dE \, \widetilde \phi_\ell(E) },
$$
and the contribution to $\calJ_{g,h+1,\ell}$ is
$$
  j_{g,h+1} = \frac{ (1 - f_{g,h}) \sum_r \Inum_{r,g,h,\ell} }
       { \int_{\calE_g} dE \, \widetilde \phi_\ell(E) }.
$$

If $\langle E' \rangle_{g,h} > \overline {E'_h}$ and $\calE'_h$
is the highest energy group, the contribution to $\calJ_{g,h,\ell}$
is
$$
  \frac{ \sum_r \Inum_{r,g,h,\ell} }
       { \int_{\calE_g} dE \, \widetilde \phi_\ell(E) }.
$$

The sum of all of these contributions produces the Legendre
coefficients $\calJ_{g,h,\ell}$ of a transfer matrix which conserves
particle number as well as usually conserving energy.

\section{Control of the conservation option}
The \gettransfer\ code computes the integrals Eq.~(\ref{Inum})
for the number-preserving transfer matrix or the
integrals Eq.~(\ref{Ien}) for the energy-preserving transfer matrix
or both, depending on the value of the \texttt{Conserve}
input parameter.  See Section~\ref{Sec:conserveFlag}.
 The default mode is to
compute both integrals.  The actual construction
of the transfer matrix is performed by \xndfgen.

\section{Numerical quadrature}
The integrals Eqs.~(\ref{Inum}) and~(\ref{Ien}) require some sort
of numerical quadrature, and the multiple integrals are computed as
a sequence of single integrals.  The user may specify Gaussian quadrature
of various orders, as explained in Section~\ref{Sec:QuadratureMethods}.
The reason for the use of Gaussian quadrature in place of, say, Simpson's
rule, is that in the calculations here,
one of the limits of integration may be a computed quantity
such as a threshold energy.  In such cases, computer arithmetic may
give rise to attempts to evaluate $\pi_r(E', \mu \mid E)$
where this function is undefined.

The user may also specify whether or not to employ an adaptive 
version of one of these Gaussian methods, using
a modification of the
adaptive quadrature method proposed by Gander and Gautschi~\cite{Gander}.
The default is to use adaptive quadrature---the non-adaptive version
is intended for debugging.  For an explanation of the quadrature options,
see Sections~\ref{Sec:QuadratureMethods} and~\ref{Sec:adaptive-quadrature}.

Another adaptation of the adaptive quadrature of the reference~\cite{Gander}
is that in the integrals Eqs.~(\ref{Inum}) and~(\ref{Ien}).

\textbf{Remark.} In the rest of this document the
subscript~$r$ is omitted from each of the terms in the kernel
Eq.~(\ref{def_pi}) and from the integrals
$\Inum_{r,g,h,\ell}$ and $\Ien_{r,g,h,\ell}$, because from now on the discussion will be
about the treatment of the data, reaction by reaction.

