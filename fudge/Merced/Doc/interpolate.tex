\chapter{Interpolation of the data}
\label{Sec:interpolate}
The data in \xendl\ representing the probability density
$\pi(E', \mu \mid E ) = \pi_r(E', \mu \mid E )$ in the integrals (\ref{Inum}) and~(\ref{Ien})
are given in various forms.  In the case of tabulated data, intermediate
values must be obtained via some sort of interpolation.  Interpolation
with respect to one independent variable is described first, followed by a discussion of
the 2-dimensional case.  In \xendl\ full 3-dimensional interpolation
of $\pi(E', \mu \mid E )$ data is reduced to a sequence of 2-dimensional
interpolations.

\section{Interpolation methods for a single variable}\label{Sec:1d-interp}
For the sake of having a specific application,
the discussion here is given in terms of tables of data $\{E_i, f(E_i)\}$,
with values $E_i$ of the energy of the outgoing particle as independent variable.
These ideas are applicable to one dimension for any tabular data.
The types of interpolation method used in \xendl\ 
for such tables are: histogram, linear-linear, 
log-linear, linear-log, and log-log.  The algorithms for interpolation
of $F(E)$ on an interval $E_0 < E < E_1$ with given $f(E_0)$ and~$f(E_1)$
are as follows.  In these definitions it is assumed that the argument of a
logarithm is positive.

\subsection{Histograms}
For histogram interpolation set
$$
  f(E) = f(E_0) \quad \text{for $E_0 \le E < E_1$}.
$$

\subsection{Linear-linear}
 For linear-linear interpolation set
\begin{equation}
   \alpha = \frac{E - E_0}{E_1 - E_0}
 \label{def-alpha}
\end{equation}
and take
$$
  f(E) = (1 - \alpha)f(E_0) + \alpha f(E_1)
   \quad \text{for $E_0 \le E \le E_1$}.
$$

\subsection{Log-linear}
For log-linear interpolation take $\alpha$ as in Eq.~(\ref{def-alpha}),
and set
$$
 \log f(E) = (1 - \alpha)\log f(E_0) + \alpha \log f(E_1)
   \quad \text{for $E_0 \le E \le E_1$}.
$$
This relation may also be written as
\begin{equation}
  f(E) = f(E_0)^{1 - \alpha} f(E_1)^\alpha.
 \label{loglin-interp}
\end{equation}

\subsection{Linear-log}
 For linear-log interpolation set
\begin{equation}
   \alpha' = \frac{\log(E / E_0)}{\log( E_1 / E_0 )}
 \label{def-alpha-prime}
\end{equation}
and take
$$
  f(E) = (1 - \alpha')f(E_0) + \alpha' f(E_1)
   \quad \text{for $E_0 \le E \le E_1$}.
$$

\subsection{Log-log}
For log-log interpolation take $\alpha'$ as in Eq.~(\ref{def-alpha-prime}),
and set
$$
 \log f(E) = (1 - \alpha')\log f(E_0) + \alpha' \log f(E_1)
   \quad \text{for $E_0 \le E \le E_1$}.
$$
This is equivalent to
\begin{equation}
  f(E) = f(E_0)^{1 - \alpha'} f(E_1)^{\alpha'}.
 \label{loglog-interp}
\end{equation}

\textbf{Remark.}
With log-linear interpolation written in the form of Eq.~(\ref{loglin-interp})
and log-log interpolation written as Eq.~(\ref{loglog-interp}), it is permitted
that $f(E_0) = 0$ or $f(E_1) = 0$.  These cases all lead to the result that
$f(E) = 0$ for $E_0 < E < E_1$, however.

\section{Interpolation methods for probability densities}\label{Sec:2d-interp}
In order to explain the methods for interpolation of probability densities, it suffices to
consider a table of values $\pi( E' \mid E )$
\begin{equation}
  \{ E'_{j,k}, \pi(E'_{j,k} \mid E_k )\} \quad \text{for $j = 0$, 1, $\ldots\,$, $J_k$}
  \label{EPtable}
\end{equation}
given at values of the incident energy~$E_k$, for
$k = 0$,~1, $\ldots\,$,~$K$.
In Eq.~(\ref{EPtable}) it is required that the outgoing energies
be ordered
\begin{equation}
  E'_{0,k} < E'_{1,k} \le E'_{2,k} \le \cdots \le E'_{J_k-1,k} < E'_{J_k,k}.
  \label{Eout-order}
\end{equation}

The condition Eq.~(\ref{Eout-order}) permits the data of
Eq.~(\ref{EPtable}) to have equal consecutive intermediate
outgoing energies $E'_{j-1,k} = E'_{j,k}$, so that the probability
density~$\pi(E' \mid E_k)$ may have a jump discontinuity there.
Jump discontinuities are not allowed at the end points $E' = E'_{0,k}$
and~$E' = E'_{J_k,k}$.  In Eq.~(\ref{EPtable}) the possibility 
of three or more consecutive equal outgoing energies may be ruled out,
because all but the first and last would be redundant. 
The convention adopted here is that the value of $\pi(E' \mid E_k)$
at a discontinuity is the second data value
$$
 \pi(E' \mid E_k) = \pi(E'_{j,k} \mid E_k) \quad
  \text{if $E' = E'_{j-1,k} = E'_{j,k}$.}
$$

For fixed incident energy $E_k$, the rules for interpolation of 
$\pi(E' \mid E_k )$ in outgoing energy~$E'$ are as given
in Section~\ref{Sec:1d-interp}.  The following types
of interpolation with respect to $E$ are discussed in subsequent subsections:
\begin{enumerate}
 \item direct interpolation.
 \item unit-base interpolation,
 \item interpolation using cumulative points.
\end{enumerate}

The method referred to here as ``interpolation by cumulative points''
is closely related to ``interpolation by corresponding energies'' as described
in the \ENDF\ manual~\cite{ENDFB}.  
For a more-detailed discussion of 2-dimensional interpolation methods,
see the reference~\cite{interpolation}.

For a discussion of interpolation of data Eq.~(\ref{EPtable}), it suffices
to consider interpolation between incident energies $E_0$ and~$E_1$.
Thus, it is desired to interpolate to incident energy~$E$ with
$E_0 < E < E_1$ the data
\begin{equation}
\begin{split}
  \{ E'_{j,0}, \pi(E'_{j,0} \mid E_0 )\} &\quad \text{for $j = 0$, 1, $\ldots\,$, $J_0$}, \\
  \{ E'_{j,1}, \pi(E'_{j,1} \mid E_1 )\} &\quad \text{for $j = 0$, 1, $\ldots\,$, $J_1$}.
  \label{EPtables}
\end{split}
\end{equation}
The ideas presented apply equally well to interpolation of data
in Eq.~(\ref{EPtable}) between any consecutive pair of
incident energies $E_{k-1} < E_k$.

The methods of 2-dimensional interpolation are described in turn.
For each of these methods the interpolated probability density $\pi(E' \mid E)$
for $E_0 < E < E_1$ has the proper norm
\begin{equation}
  \int
  \pi(E' \mid E) \ dE' = 1
  \label{piNorm}
\end{equation}
when the interpolation rule with respect to incident energy~$E$ for the
data Eq.~\eqref{EPtables} is linear-linear, histogram, or linear-log.
The norm condition Eq.~\eqref{piNorm} is not usually satisfied when
log-linear or log-log interpolation is used for the incident energy.

\subsection{Direct interpolation}\label{Sec:direct-interp}
It is common to do direct interpolation for interpolating tables of angular probability
density~$\pi( \mu \mid E)$ with respect to incident energy~$E$, because the
range of direction cosines is usually $-1 \le \mu \le 1$.  For example, in order
to determine the value of $\pi( \mu \mid E)$ for $E_0 < E < E_1$ from data
Eq.~(\ref{EPtables}), one first interpolates in $\mu$
at fixed incident energies to obtain $\pi( \mu \mid E_0)$ and~$\pi( \mu \mid E_1)$.
One then obtains the value of $\pi( \mu \mid E)$ by interpolating between
$\pi( \mu \mid E_0)$ and~$\pi( \mu \mid E_1)$.

The trouble with the application of direct interpolation to tables of 
energy distributions
 is that the range of outgoing energy~$E'$
usually depends on the incident energy~$E$.
Thus, for the data in Eq.~(\ref{EPtables}), the ranges of outgoing energies
are given by
\begin{equation}
\begin{split}
  \Eminzero' =  E'_{0,0} \quad \text{and} \quad
           \Emaxzero' = E'_{J_0,0} \quad \text{for $E = E_0$}, \\
  \Eminone' =  E'_{0,1} \quad \text{and} \quad
           \Emaxone' = E'_{J_1,1} \quad \text{for $E = E_1$}.
  \label{Eout-ranges}
\end{split}
\end{equation}

\textbf{Remark.}
In the definition of the range of outgoing energies Eq.~(\ref{Eout-ranges}),
it is natural to expect that the data in Eq.~(\ref{EPtables}) are such that
for each incident energy $E_k$ with $k = 0,$~1, $\ldots\,$,~$K$, the probability
density $\pi(E' \mid E_k )$ is not equal to zero on the entire lowest outgoing
energy range $E'_{0,k} < E' < E'_{1,k}$ or highest outgoing energy range
$E'_{J_k - 1,k} < E' < E'_{J_k,k}$.  That is, 
Eq.~(\ref{Eout-ranges}) ought to give the
actual range of outgoing energies.
Some nuclear data libraries, e.~g.,
\ENDFdata~\cite{ENDFdata}, have
data of the form Eq.~(\ref{EPtable}) which imply that
$\pi(E' \mid E_k ) = 0$ on the lowest or highest outgoing energy ranges.
The sample input data given in
Section~\ref{Sec:isotropic-table-lab} illustrates the problem.

It is convenient to describe the process of direct interpolation using
notation of set theory, with the sets
\begin{equation}
\begin{split}
  \calA_0 = \{ E': \Eminzero' \le E' \le
           \Emaxzero' \}, \\
  \calA_1 = \{ E':   \Eminone' \le E' \le
           \Emaxone' \}.
  \label{def-calA01}
\end{split}
\end{equation}
The union of these two sets is denoted by
\begin{equation}
   \calA_X = \calA_0 \cup \calA_1,
 \label{def-calA-X}
\end{equation}
and the intersection is denoted by
\begin{equation}
   \calA_T = \calA_0 \cap \calA_1,
 \label{def-calA-T}
\end{equation}

There are two obvious interpretations of direct interpolation of
the data in Eq.~(\ref{EPtables}) when the outgoing energy ranges
differ, $\calA_0 \ne \calA_1$.  One may do \textit{direct interpolation
with extrapolation} or \textit{direct interpolation
with truncation}.   Linear-linear versions of these methods are
described here. 

For direct interpolation with extrapolation the probability densities
$\pi(E' \mid E_0 )$ and $\pi(E' \mid E_1 )$ constructed from the
tables in Eq.~(\ref{EPtables})  are extrapolated to
\begin{equation}
  \pi_X(E' \mid E_0 ) = \begin{cases}
    \pi(E' \mid E_0 ) &\quad \text{for $E'$ in $\calA_0$,} \\
    0 &\quad \text{for $E'$ in $\calA_X \setminus \calA_0$,}
  \end{cases}
  \label{def-pi-X0}
\end{equation}
and
\begin{equation}
  \pi_X(E' \mid E_1 ) = \begin{cases}
    \pi(E' \mid E_1 ) &\quad \text{for $E'$ in $\calA_1$,} \\
    0 &\quad \text{for $E'$ in $\calA_X \setminus \calA_1$.}
  \end{cases}
  \label{def-pi-X1}
\end{equation}
For direct interpolation
to incident energy $E$ with $E_0 < E < E_1$,
the proportionality factor $q$ is  defined as
\begin{equation}
  q = \frac{ E - E_0}{ E_1 - E_0 }.
  \label{def-q}
\end{equation}
In linear-linear direct interpolation with extrapolation,
the interpolant is taken to be
\begin{equation}
  \pi_X(E' \mid E ) = ( 1 - q )\, \pi_X(E' \mid E_0 ) + q\, \pi_X(E' \mid E_1 )
  \label{def-pi-X}
\end{equation}
for $E'$ in the set~$\calA_X$.

The method of direct interpolation with truncation differs from
that using extrapolation, in that this method uses the truncated
probability densities
\begin{equation}
 \begin{split}
   \pi_T(E' \mid E_0 )& = C_0 \pi(E' \mid E_0 ), \\
   \pi_T(E' \mid E_1 )& = C_1 \pi(E' \mid E_1 )
  \end{split}
 \label{def-pi-T01}
\end{equation}
for outgoing energy $E'$ in the set~$\calA_T$.  Here, $C_0$ 
and~$C_1$ are normalization constants such that
$$
   \int_{\calA_T} dE' \, \pi_T(E' \mid E_0 ) = 1
   \quad \text{and} \quad
   \int_{\calA_T} dE' \, \pi_T(E' \mid E_1 ) = 1.
$$
For linear-linear direct interpolation with truncation of the data in Eq.~(\ref{EPtables})
to incident energy $E$ with $E_0 < E < E_1$, the factor~$q$
is chosen as in Eq.~(\ref{def-q}), and the interpolant is
$$
  \pi_T(E' \mid E ) = ( 1 - q )\, \pi_T(E' \mid E_0 ) + q\, \pi_T(E' \mid E_1 )
$$
for $E'$ in the set~$\calA_T$.

\textbf{Remarks.}
The \ENDFdata\ data~\cite{ENDFdata} contains many instances in which
linear-linear direct interpolation is specified, but the \ENDF\ manual~\cite{ENDFB}
says nothing about how to deal with differences in range of outgoing
energies.  Both versions can be expected to produce violation of
energy conservation.  The {\gettransfer} code
currently uses direct interpolation with extrapolation.

\subsection{Unit-base interpolation}\label{Sec:unitBase}
Only the linear-linear version of unit-base interpolation is discussed here.
The first step in unit-base interpolation is the construction of the
range of energies of the outgoing particle.  
The minimum and maximum
outgoing energies for the data in Eq.~(\ref{EPtables})
are given by Eq.~(\ref{Eout-ranges}).
For incident energy~$E$ with $E_0 < E < E_1$, the factor~$q$
is taken as in Eq.~(\ref{def-q}), and
 the minimum and maximum outgoing energies
are given by
\begin{equation}
 \begin{split}
   \Emin'& = ( 1 - q )\Eminzero' + q \Eminone', \\
   \Emax' & = ( 1 - q )\Emaxzero' + q \Emaxone'.
  \label{EoutRange}
 \end{split}
\end{equation}

The interpolated probability density $\pi(E' \mid E)$ must
satisfy the normalization condition
\begin{equation}
  \int_{\Emin'}^{\Emax'} dE' \, \pi(E' \mid E) = 1.
 \label{probabilityNorm}
\end{equation}
One way to ensure this is to first map the outgoing energy ranges
Eq.~(\ref{Eout-ranges}) to unit base $0 \le \Ehat' \le 1$
and to scale the probability densities Eq.~(\ref{EPtables})
accordingly.  Thus, for the data in Eq.~(\ref{EPtables}) at
incident energy~$E_0$, set
\begin{equation}
  \Ehat' = \frac{ E' - \Eminzero'}{ \Emaxzero' - \Eminzero'}
  \label{unit-base-map}
\end{equation}
and scale the probability density
\begin{equation}
  \pihat(\Ehat' \mid E_0) = ( \Emaxzero' - \Eminzero' )\pi(E' \mid E_0).
  \label{unitbaseMap}
\end{equation}
For incident energy $E_1$, the outgoing energy is scaled
as
\begin{equation}
    \Ehat' = \frac{ E' - \Eminone'}{ \Emaxone' - \Eminone'},
  \label{unit-base-map1}
\end{equation}
and the probability density
is scaled to define the unit-base probability
density 
\begin{equation}
  \pihat(\Ehat' \mid E_1) = ( \Emaxone' - \Eminone' )\pi(E' \mid E_1)
  \label{unitbaseMap1}
\end{equation}
for $0 \le \Ehat' \le 1$.

If linear-linear interpolation with respect to incident energy is
desired, the proportionality factor $q$ defined in
Eq.~(\ref{def-q}) is used to linearly interpolate between
$\pihat(\Ehat' \mid E_0)$ and $\pihat(\Ehat' \mid E_1)$
by setting
\begin{equation}
  \pihat(\Ehat' \mid E) = (1 - q)\,\pihat(\Ehat' \mid E_0) +
   q \,\pihat(\Ehat' \mid E_1)
 \label{unitbaseInterp}
\end{equation}
for $0 \le \Ehat' \le 1$. 

Finally, in order to define the interpolated probability
density $\pi(E' \mid E)$, invert the mappings Eq.~(\ref{unit-base-map})
and~Eq.~(\ref{unitbaseMap}).  Specifically, with $\Emin'$ and
$\Emax'$ as in Eq.~(\ref{EoutRange}), set
\begin{equation}
   E' = \Emin' + ( \Emax' - \Emin')\Ehat'
 \label{range-inv}
\end{equation}
and take
\begin{equation}
  \pi(E' \mid E) = \frac{\pihat(\Ehat' \mid E)}{ \Emax' - \Emin' }.
 \label{unitbaseInvert}
\end{equation}

Unit-base interpolation is ordinarily not used with tables of
angular probability densities~$\pi( \mu \mid E)$, because the
range of direction cosines is usually~$-1 \le \mu \le 1$.  One
may want to use it for a table with forward emission given in
the laboratory frame, however.

\subsection{Interpolation by cumulative points}\label{Sec:cumProb}
The method of interpolation by cumulative points that is
used in the code \gettransfer\ is proposed in~\cite{interpolation},
and it is a modification of interpolation by corresponding energies as described
in the \ENDF\ manual~\cite{ENDFB}.  
Interpolation by corresponding energies requires the selection of
$N$ equiprobable energy bins, so the result depends on the value of~$N$.
It is shown in~\cite{interpolation} that for data Eq.~(\ref{EPtable}) which are
histograms with respect to outgoing energy $E'$, interpolation by cumulative 
points is equivalent to interpolation by corresponding energies with~$N = \infty$.
The \gettransfer\ code therefore uses interpolation by cumulative points
whenever the data specify interpolation by corresponding energies.

One objection to unit-base interpolation is that the mapping
(\ref{unitbaseMap}) depends only
on the range of outgoing energies.
One can often get a better approximation to the physics if the
interpolation method incorporates
knowledge of the local behavior of each $\pi( E' \mid E_k)$
in Eq.~(\ref{EPtable}).
One method of doing so is based on the cumulative
probability function
\begin{equation}
  \Pi( E' \mid E_k) = \int_{E'_{k,\text{min}}}^{E'} dx\, \pi( x \mid E_k)
  \label{cumProb}
\end{equation}
for $k = 0$,~1, $\ldots\,$,~$K$.




Because the data $\pi( E' \mid E_k)$ consist of probability densities,
it follows that $\pi( E' \mid E_k) \ge 0$.  Hence, $\pi( E' \mid E_k)$ 
s a non-decreasing function of~$E'$.  There are
energy distributions in the data library
\xendl~\cite{GND} for which $\pi( E' \mid E_k ) = 0$
on an interval $E_{j - 1,k}' < E' < E_{j,k}$ for $k = 0$~or~1
and for some values of $j = 1$, 2, $\ldots\,$, $J_k$.
The corresponding cumulative probability is constant
on such intervals.  The method of cumulative points
depends on solutions of the equation
$$
  \Pi(E' \mid E_k ) = Y
$$
for given value of  $0 \le Y \le 1$.  Denote the
largest solution by
$$
  S = \max \left \{ E': \Pi(E' \mid E_k ) = Y \right \}
$$
and the smallest solution by
$$
  T = \min \left \{ E': \Pi(E' \mid E_k ) = Y \right \}.
$$
If $\pi( E' \mid E_k ) = 0$ only at discrete
points, then $\Pi(E' \mid E_k )$ is strictly increasing
and $S = T$.

The cumulative points method is defined as
follows.
\begin{enumerate}
\item For probability density data Eqs.~\eqref{EPtables}
at incident energy~$E_k$ with $k = 0$ and~1, compute the cumulative
probabilities $\Pi( E' \mid E_k )$ in Eqs.~\eqref{cumProb}
at the data points~$E_{j,k}'$ for $j = 0$, 1, $\ldots\,$, $J_k$.  
Denote the result as
\begin{equation}
  y_{j,k} = \Pi( E_{j,k}' \mid E_k ).
 \label{def-phijk}
\end{equation}
\item Form the union of these two sets
\begin{equation}
   \{Y_\ell\} = \{y_{j,0}\} \cup \{y_{j,1}\},
 \label{def-Phi-cup}
\end{equation}
remove duplicates, and arrange the remaining values
in increasing order,
\begin{equation}
  0 = Y_0 < Y_1 < \cdots < Y_L = 1.
 \label{Phi-ordered}
\end{equation}
\item For $k = 0$ and~1 and for each $Y = Y_\ell$ with
$\ell = 1$, 2, $\ldots\,$, $L$,
compute the outgoing energies
$S_{\ell,k}'$ and~$T_{\ell,k}'$ by solving the equations
\begin{equation}
 \begin{split}
     S_{\ell,k}' &= \max \left \{ E': \Pi(E' \mid E_k ) = Y_{\ell-1} \right \}, \\
      T_{\ell,k}' &= \min \left \{ E': \Pi(E' \mid E_k ) = Y_\ell \right \}.
 \end{split}
  \label{T-cum-prob-inv}
\end{equation}
\item For $k = 0$ and~1 form the intervals $\calB_\ell(E_k)$
\begin{equation}
 \begin{split}
   \calB_\ell(E_k) &= \{E' : S_{\ell,k}' \le E' < T_{\ell,k}' \} 
     \quad \text{for $\ell = 1$,~2, $\ldots\,$,~$L-1$,} \\
   \calB_L(E_k) &= \{E' : S_{L,k}' \le E' \le T_{L,k}' \}.
 \end{split}
  \label{def-B-ell}
\end{equation}
The reason for omitting right-hand endpoints of
intervals $\calB_\ell(E_k)$ for $\ell < L$ in Eqs.~\eqref{def-B-ell}
is that they may be points of discontinuity
of the original energy distributions in Eq.~\eqref{EPtables}.
\item For incident energy $E$ with $E_0 < E < E_1$ do unit-base
interpolation of $\pi( E' \mid E_0 )$ on $\calB_\ell(E_0)$ with
$\pi( E' \mid E_1 )$ on $\calB_\ell(E_1)$
for $\ell = 1$,~2, $\ldots\,$,~$L$. The interpolation with
respect to~$E$ may be histogram, linear-linear, or linear-log.
The linear-linear version is as follows.  
For $k = 0$ and~1 use
\begin{equation}
  \Ehat' = \frac{ E' -  S_{\ell,k}'}{  T_{\ell,k}' -  S_{\ell,k}'}
  \label{unit-base-map-cum}
\end{equation}
to map the interval $\calB_\ell(E_k)$
to $0 \le \Ehat' < 1$ for $\ell = 1$,~2, $\ldots\,$,~$L-1$ and to
$0 \le \Ehat' \le 1$ for $\ell = L$.  Scale the probability density
accordingly,
\begin{equation}
  \pihat(\Ehat' \mid E_k) = (T_{\ell,k}' -S_{\ell,k}' )\pi(E' \mid E_0).
  \label{unit-base-scale}
\end{equation}
Define the proportionality factor
\begin{equation}
  \alpha = \frac{ E - E_0 }{ E_1 - E_0 }
  \label{def-alpha-cum}
\end{equation}
and interpolate to obtain
\begin{equation}
  \pihat(\Ehat' \mid E) = (1 - \alpha)\pihat(\Ehat' \mid E_0) +
   \alpha \pihat(\Ehat' \mid E_1)
 \label{unitbaseInterpCum}
\end{equation}
for $0 \le \Ehat' < 1$ and $\ell < L$ and for $0 \le \Ehat' \le 1$ and $\ell = L$.
For $\ell = 1$,~2, $\ldots\,$,~$L$
interpolate the end points of the intervals $\calB_\ell(E_k)$ to
obtain
\begin{equation}
 \begin{split}
  S_{\ell,\alpha}' &= (1 - \alpha)S_{\ell,0}' + \alpha S_{\ell,1}', \\
  T_{\ell,\alpha}' &= (1 - \alpha)T_{\ell,0}' + \alpha T_{\ell,1}'.
 \end{split}
  \label{energy-range}
\end{equation}
At incident energy~$E$ 
the range of outgoing energies~$E'$ as taken as
\begin{equation}
 \begin{split}
   \calB_\ell(E) &= \{E' : S_{\ell,\alpha}' \le E' < T_{\ell,\alpha}' \}
     \quad \text{for $\ell = 1$,~2, $\ldots\,$,~$L-1$,} \\
   \calB_L(E) &= \{E' : S_{L,\alpha}' \le E' \le T_{L,\alpha}' \}.
 \end{split}
  \label{def-B-ell-alpha}
\end{equation}
Finally, the interpolated probability
density $\pi( E' \mid E )$ is obtained by inversion of the mapping
Eq.~\eqref{unit-base-scale}, giving
\begin{equation}
  \pi(E' \mid E) = \frac{\pihat(\Ehat' \mid E)}{  T_{\ell,\alpha}' -  S_{\ell,\alpha}'}
 \label{map-inv}
\end{equation}
for $E'$ in $\calB_\ell(E)$.
Set $\pi(E' \mid E) = 0$ for all outgoing energies~$E'$ which 
are not in any of the sets $\calB_\ell(E)$ for 
$\ell = 1$,~2, $\ldots\,$,~$L$.
\end{enumerate}

For tables of angular probability densities~$\pi( \mu \mid E)$, direct interpolation
is often used for interpolating with respect to incident energy~$E$,
because the direction cosines usually range over $-1 \le \mu \le 1$.
For angular distributions with strong local features, interpolation by cumulative points 
may be preferable.


\subsubsection{Practical considerations: intervals of zero length} 
\label{Sec:cum-points-trivial}
Experience with interpolation
by cumulative points as described above shows that it must be
modified to deal with the inaccuracy of
computer arithmetic.  In particular, it may happen that 
for $k=0$ or~1 and for some $\ell = \widehat \ell$
the values of $S'_{ \ell,k}$ and~$T'_{ \ell,k}$
as obtained from Eq.~(\ref{T-cum-prob-inv}) are computed to be equal.
The mapping Eq.~(\ref{unit-base-map-cum}) to unit base is then undefined.

This phenomenon may be understood in terms of the computer's
machine epsilon, which is the smallest number $\epsilon_{\text{mach}}$
such that $1 + \epsilon_{\text{mach}} > 1$ in the computer's arithmetic.
In double precision arithmetic a common value is
$$
  \epsilon_{\text{mach}}  = 2^{-52} \approx 2.22 \times 10^{-16}.
$$
In particular, if the exact values of $S_{\widehat \ell,k}' $ and $T_{\widehat \ell,k}' $
are such that $S_{\widehat \ell,k}' > 0$ and
$$
  T_{\widehat \ell,k}'  - S_{\widehat \ell,k}'  < \epsilon_{\text{mach}} S_{\widehat \ell,k}',
$$
then the computer will say that $S_{\widehat \ell,k}' = T_{\widehat \ell,k}'$.

Consider first the case that the probability density $\pi(E' \mid E_k )$ is
given as a histogram.  Then in exact arithmetic the cumulative points algorithm 
ensures the existence of $S_{\widehat \ell,k}' $ and $T_{\widehat \ell,k}' $
such that
\begin{equation}
  \Pi(S_{\widehat \ell,k}' \mid E_k ) = Y_{\widehat \ell-1}
  \quad \text{and} \quad
  \Pi(T_{\widehat \ell,k}' \mid E_k ) = Y_{\widehat \ell}.
 \label{get-S-T}
\end{equation}
Furthermore, it is assured that $S_{\widehat \ell,k}'  < T_{\widehat \ell,k}' $
and that
$$
  \pi(E' \mid E_k ) = \pi_{\widehat \ell,k}
  \quad \text{for} \quad
  S_{\widehat \ell,k}' \le E' < T_{\widehat \ell,k}',
$$
where $\pi_{\widehat \ell,k}$  is a positive constant.  Thus, if it happens
that
\begin{equation}
  Y_{\widehat \ell} - Y_{\widehat \ell-1} <
  \epsilon_{\text{mach}}  \pi_{\widehat \ell,k} S_{\widehat \ell,k}' ,
 \label{bad-cum-prob}
\end{equation}
the computer will conclude that $S_{\widehat \ell,k}' = T_{\widehat \ell,k}'$.
Because of the factor $\pi_{\widehat \ell,k}$ appearing on the right-hand
side of Eq.~{\ref{bad-cum-prob}), this phenomenon is usually associated
with narrow resonances in the data.  The size of $ \pi_{\widehat \ell,k}$ depends
on the units used for energy, but the product $ \pi_{\widehat \ell,k} S_{\widehat \ell,k}' $
is dimensionless, and it is typically large at a resonance.  For example, 
for $S_{\widehat \ell,k}'  = 1\,\text{MeV}$ the value $ \pi_{\widehat \ell,k} 
= 1.0 \times 10^8\,\text{MeV}^{-1}$ is reasonable, and Eq.~(\ref{bad-cum-prob})
could easily be satisfied.

The situation for a piecewise linear probability density $\pi(E' \mid E_k )$ is
a little different.  For one thing, the value $ \pi_{\widehat \ell,k}$ in
Eq.~(\ref{bad-cum-prob}) is replaced by the average
$$
  \frac{1}{2} \left(
    \pi( S_{\widehat \ell,k}' \mid E_k ) +  \pi( T_{\widehat \ell,k}' \mid E_k ) 
  \right).
$$
Another difference from the histogram case is that the determination of
$S_{\widehat \ell,k}' $ and $T_{\widehat \ell,k}' $ in Eq.~(\ref{get-S-T}) 
require the solution of quadratic equations, with additional inaccuracy
introduced by the computer arithmetic.

This raises the question of what to do when the solution of
Eq.~(\ref{T-cum-prob-inv}) gives values of $S_{ \widehat \ell,k}'$ and 
$T_{\widehat \ell,k}'$
which are nearly equal for some $\widehat \ell$ and~$k$.
The first requirement is a definition of equality for computer arithmetic.
In the \gettransfer\ code $S'_{ \widehat \ell,k}$ and~$T'_{ \widehat \ell,k}$
are considered to be essentially equal if
\begin{equation}
  \left| 
    S'_{ \widehat \ell,k} - T'_{ \widehat \ell,k}
  \right| <
  \frac{ \delta_r} {2} \left(
    S'_{ \widehat \ell,k} + T'_{ \widehat \ell,k}
  \right),
 \label{bad-unit-base}
\end{equation}
where $\delta_r$ is the parameter \texttt{tight\_tol}
described in Section~\ref{Sec:floatingPoint}.  

 The decision on what to do in \gettransfer\ about such near equality of
$S'_{ \widehat \ell,k}$ and~$T'_{ \widehat \ell,k}$  is based on the size of
$Y_{\widehat \ell} - Y_{\widehat \ell-1}$.  Let $\delta_c$
be the parameter \texttt{cum\_prob\_skip} described in
Section~\ref{Sec:cum-prob-skip}. 
If Eq~(\ref{bad-unit-base})
is satisfied and
\begin{equation}
  Y_{\widehat \ell} - Y_{\widehat \ell-1} < \delta_c,
 \label{drop-B-ell}
\end{equation}
then the sets $\calB_{\widehat \ell}(E_k)$ with $k = 0$,~1
in Eq.~(\ref{def-B-ell}) are omitted from the computation of the
transfer matrix.  This omission introduces an error of at most one part
in~$1/\delta_c$ in the transfer matrix.

On the other hand, if Eq~(\ref{bad-unit-base})
is satisfied and Eq~(\ref{drop-B-ell}) is violated for some
$\widehat \ell$ and~$k$, then in \gettransfer\ the interval
$\calB_{\widehat \ell}(E_k)$ is reduced to the point
\begin{equation}
   \calB_{\widehat \ell}(E_k) = \{E' : E' = \widetilde S_{\widehat \ell,k}' \} 
  \label{def-B-ell-point}
\end{equation}
with
\begin{equation}
    \widetilde S_{\widehat \ell,k}' =
     \frac{ 1} {2} \left(
    S'_{ \widehat \ell,k} + T'_{ \widehat \ell,k}
  \right),
  \label{def-S-ell-point}
\end{equation}
and $\pi( E' \mid E_k )$ on $\calB_{\widehat \ell}(E_k)$ is taken
to be a delta function
\begin{equation}
  \pi( E' \mid E_k ) = \left(
    Y_{\widehat \ell} - Y_{\widehat \ell - 1}
   \right)
  \delta( E' - \widetilde S_{\widehat \ell,k}'  ).
 \label{delta-data}
\end{equation}
Note that in exact arithmetic the intervals $\calB_{\ell}( E_k )$ defined
in Eq.~(\ref{def-B-ell}) are disjoint.  For the construction here,
however, there is no guarantee that the point $\calB_{\widehat \ell}(E_k)$
obtained from Eqs.~(\ref{def-B-ell-point}) and~(\ref{def-S-ell-point})
is not contained in $\calB_{\widehat \ell - 1}( E_k )$ or~$\calB_{\widehat \ell + 1}( E_k )$.

If it happens that the delta function Eq.~(\ref{delta-data}) is used
for both $k = 0$ and $k = 1$, then for incident energy~$E$ with
$E_0 < E < E_1$, the set  $\calB_{\widehat \ell}(E)$ in
Eq.~(\ref{def-B-ell-alpha}) is defined as the single point
$$
  \widetilde S_{\widehat \ell, \alpha}' = ( 1 - \alpha ) \widetilde S_{\widehat \ell, 0}' +
  \alpha \widetilde S_{\widehat \ell,1}'
$$
with $\alpha$ given by Eq.~(\ref{def-alpha-cum}).  The
probability density on $\calB_{\widehat \ell}(E)$ is taken to be
the delta function
$$
   \pi( E' \mid E ) = \left(
    Y_{\widehat \ell} - Y_{\widehat \ell - 1}
   \right)
  \delta( E' - \widetilde S_{\widehat \ell, \alpha}'  ).
$$

It remains to consider the case when Eq~(\ref{bad-unit-base})
is satisfied and Eq~(\ref{drop-B-ell}) is false for only one value
of~$k$, say $k = 0$.  In that case the set $ \calB_{\widehat \ell}(E_0)$
reduces to a point as in Eqs.~(\ref{def-B-ell-point}) and~(\ref{def-S-ell-point}), and
$\pi( E' \mid E_0 )$ on  $ \calB_{\widehat \ell}(E_0)$ is defined to be as
in Eq.~(\ref{delta-data}) with $k = 0$.  For the mapping of this
$\pi( E' \mid E_0 )$ on  $ \calB_{\widehat \ell}(E_0)$ to unit base,
Eq.~(\ref{unit-base-scale}) is replaced by in \gettransfer\ by
$$
  \pihat(\Ehat' \mid E_0) = Y_{\widehat \ell} - Y_{\widehat \ell - 1}
  \quad \text{for} \quad
  0 \le \Ehat'  \le 1.
$$
At incident energy~$E$ with $E_0 < E < E_1$ the interpolation
Eq.~(\ref{unitbaseInterpCum}) is performed using $\alpha$
defined by Eq.~(\ref{def-B-ell-alpha}).  For inversion of the
unit base map at incident energy~$E$, the range of outgoing
energies Eq.~(\ref{energy-range}) is replaced by
\begin{equation*}
 \begin{split}
  S_{\ell,\alpha}' &= (1 - \alpha) \widetilde S_{\widehat \ell, 0}'  + \alpha S_{\ell,1}', \\
  T_{\ell,\alpha}' &= (1 - \alpha) \widetilde S_{\widehat \ell, 0}'  + \alpha T_{\ell,1}'.
 \end{split}
  %\label
\end{equation*}
The probability density $\pi( E' \mid E )$ for $S_{\ell,\alpha}' \le E' <
T_{\ell,\alpha}'$ is then calculated using the scaling Eq.~(\ref{map-inv}).


\section{Unscaled interpolation of Kalbach-Mann data}\label{Sec:Kalbach-r-interp}
The above discussion pertains to the interpolation of tables of probability densities,
for which maintenance of the norm condition Eq.~(\ref{probabilityNorm})
is essential.  The parameter~$r(\Ecm', E)$ in Eq.~(\ref{Kalbach-eta}) for the
Kalbach-Mann model of double-differential data is given as tables
depending on the energy $E$ of the incident particle and the energy $\Ecm'$ of
the outgoing particle in the center-of-mass frame, and it has the different
constraint,
\begin{equation}
  0 \le r \le 1.
 \label{Kalbach-r-constraint}
\end{equation}

Again, it suffices to describe interpolation between Kalbach-Mann $r$
data between tables at incident energies $E_0$ and~$E_1$ with
$E_0 < E_1$.  As in Eq.~(\ref{def-calA01}),
consider the sets $\calA_0$ of outgoing energies
at $E = E_0$ and $\calA_1$ at $E = E_1$.  For unscaled direct interpolation
with extrapolation, take $\calA_X = \calA_0 \cup \calA_1$ as in
Eq.~(\ref{def-calA-X}), so that the extrapolated $r$ parameter is
\begin{equation*}
  r_X(E', E_0 ) = \begin{cases}
    r(E', E_0 ) &\quad \text{for $E'$ in $\calA_0$,} \\
    0 &\quad \text{for $E'$ in $\calA_X \setminus \calA_0$,}
  \end{cases}
%  \label{def-r-X0}
\end{equation*}
and
\begin{equation*}
  r_X(E', E_1 ) = \begin{cases}
    r(E', E_1 ) &\quad \text{for $E'$ in $\calA_1$,} \\
    0 &\quad \text{for $E'$ in $\calA_X \setminus \calA_1$.}
  \end{cases}
%  \label{def-r-X1}
\end{equation*}
Then for $E_0 < E < E_1$, for $q$ as in Eq.~(\ref{def-q}) and for
$E'$ in the set~$\calA_X$, the linear-linear form of unscaled direct interpolation
with extrapolation becomes as in Eq.~(\ref{def-pi-X}),
\begin{equation}
  r_X(E', E ) = ( 1 - q )\, r_X(E', E_0 ) + q\, r_X(E', E_1 ).
  \label{r-direct-extrapolation}
\end{equation}
The extrapolation version of direct interpolation of the Kallbach-Mann
$r$ parameter as in Eq.~(\ref{r-direct-extrapolation}) is implemented in
the \gettransfer\ code.

For unscaled direct interpolation of the Kalbach-Mann $r$ parameter with
truncation, the outgoing energy $E'$ is restricted to the common 
domain $\calA_T = \calA_0 \cap \calA_1$, and there is no change
of scale analogous to that used for probability densities in
Eq.~(\ref{def-pi-T01}).  Thus, the truncated Kalbach-Mann $r$ parameters
for incident energies $E_0$ and~$E_1$ are
\begin{equation*}
  r_T(E', E_0 ) = \begin{cases}
    r(E', E_0 ) &\quad \text{for $E'$ in $\calA_T$,} \\
    0 &\quad \text{for $E'$ in $\calA_0 \setminus \calA_T$,}
  \end{cases}
%  \label{def-r-T0}
\end{equation*}
and
\begin{equation*}
  r_X(E', E_1 ) = \begin{cases}
    r(E', E_1 ) &\quad \text{for $E'$ in $\calA_T$,} \\
    0 &\quad \text{for $E'$ in $\calA_1 \setminus \calA_T$.}
  \end{cases}
%  \label{def-r-T1}
\end{equation*}
The linear-linear version of unscaled direct
interpolation with truncation is
\begin{equation}
  r_T(E', E ) = ( 1 - q )\, r_T(E', E_0 ) + q\, r_T(E', E_1 )
  \label{r-direct-truncation}
\end{equation}
with $E'$ restricted to~$\calA_T$.
The \gettransfer\ code does not currently implement unscaled direct
interpolation with truncation of the Kalbach-Mann $r$ parameter
given in Eq.~(\ref{r-direct-truncation}).

There is also an unscaled version of unit-base interpolation with
Eqs.~(\ref{unitbaseMap}) and~(\ref{unitbaseMap1}) replaced by
\begin{equation*}
 \begin{split}
  \widehat r(\Ehat', E_0) &= r(E', E_0), \\
  \widehat r(\Ehat', E_1) &= r(E', E_1), \\
 \end{split}
\end{equation*}
for $0 \le \Ehat' \le 1$ with $\Ehat'$ as in Eq.~(\ref{unit-base-map-cum})
for $E = E_0$ and as in Eq.~(\ref{unit-base-map1})
for $E = E_1$.  For linear-linear unscaled unit-base interpolation
to incident energy $E$ with $E_0 < E < E_1$, interpolate the
minimal and maximal outgoing energies as in Eq.~(\ref{EoutRange}),
interpolate $\widehat r$ using
$$
    \widehat r(E', E ) = ( 1 - q )\, \widehat r(E', E_0 ) + q\, \widehat r(E', E_1 ),
$$
and invert the unit-base map using Eq.~(\ref{range-inv})
and 
$$
  r(E', E) = \widehat r(\Ehat', E)
$$
for $\Emin' \le E' \le \Emax'$.

When the energy probability density $\pi_E( E' \mid E )$ in Eq.~(\ref{Kalbach-prob})
is interpolated using the method of cumulative points, the interpolated values of
$r(E', E)$ in Eq.~(\ref{Kalbach-eta}) are obtained using the method of
unscaled cumulative points defined as follows.  Because the data for
 $r( E' , E )$ and  $\pi_E( E' \mid E )$ are given at the same energy points
 $E$ and~$E'$, it is natural to use for $r( E' , E_0 )$ and $r( E' , E_1 )$ the
 intervals  $\calB_{ \ell}(E_k )$, k = 0,~1, constructed for $\pi_E( E' \mid E_k )$
in Eq.~(\ref{def-B-ell}).  The \gettransfer\ code uses unscaled unit-base interpolation 
between $r( E', E_0)$ on $\calB_{\ell}(E_0 )$ and  $r( E', E_0)$ on $\calB_{ \ell}(E_1 )$
in sequence for $\ell = 1,$ 2, $\ldots\,$, $L$. This method is applicable
even when a set $\calB_{ \ell}(E_k )$ reduces to a point as discussed in
Section~\ref{Sec:cum-points-trivial}, as here is no problem of possible division by zero.


